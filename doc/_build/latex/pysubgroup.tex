%% Generated by Sphinx.
\def\sphinxdocclass{report}
\documentclass[letterpaper,10pt,english]{sphinxmanual}
\ifdefined\pdfpxdimen
   \let\sphinxpxdimen\pdfpxdimen\else\newdimen\sphinxpxdimen
\fi \sphinxpxdimen=.75bp\relax

\PassOptionsToPackage{warn}{textcomp}
\usepackage[utf8]{inputenc}
\ifdefined\DeclareUnicodeCharacter
% support both utf8 and utf8x syntaxes
  \ifdefined\DeclareUnicodeCharacterAsOptional
    \def\sphinxDUC#1{\DeclareUnicodeCharacter{"#1}}
  \else
    \let\sphinxDUC\DeclareUnicodeCharacter
  \fi
  \sphinxDUC{00A0}{\nobreakspace}
  \sphinxDUC{2500}{\sphinxunichar{2500}}
  \sphinxDUC{2502}{\sphinxunichar{2502}}
  \sphinxDUC{2514}{\sphinxunichar{2514}}
  \sphinxDUC{251C}{\sphinxunichar{251C}}
  \sphinxDUC{2572}{\textbackslash}
\fi
\usepackage{cmap}
\usepackage[T1]{fontenc}
\usepackage{amsmath,amssymb,amstext}
\usepackage{babel}



\usepackage{times}
\expandafter\ifx\csname T@LGR\endcsname\relax
\else
% LGR was declared as font encoding
  \substitutefont{LGR}{\rmdefault}{cmr}
  \substitutefont{LGR}{\sfdefault}{cmss}
  \substitutefont{LGR}{\ttdefault}{cmtt}
\fi
\expandafter\ifx\csname T@X2\endcsname\relax
  \expandafter\ifx\csname T@T2A\endcsname\relax
  \else
  % T2A was declared as font encoding
    \substitutefont{T2A}{\rmdefault}{cmr}
    \substitutefont{T2A}{\sfdefault}{cmss}
    \substitutefont{T2A}{\ttdefault}{cmtt}
  \fi
\else
% X2 was declared as font encoding
  \substitutefont{X2}{\rmdefault}{cmr}
  \substitutefont{X2}{\sfdefault}{cmss}
  \substitutefont{X2}{\ttdefault}{cmtt}
\fi


\usepackage[Bjarne]{fncychap}
\usepackage{sphinx}

\fvset{fontsize=\small}
\usepackage{geometry}


% Include hyperref last.
\usepackage{hyperref}
% Fix anchor placement for figures with captions.
\usepackage{hypcap}% it must be loaded after hyperref.
% Set up styles of URL: it should be placed after hyperref.
\urlstyle{same}
\addto\captionsenglish{\renewcommand{\contentsname}{Contents:}}

\usepackage{sphinxmessages}
\setcounter{tocdepth}{1}


% Jupyter Notebook code cell colors
\definecolor{nbsphinxin}{HTML}{307FC1}
\definecolor{nbsphinxout}{HTML}{BF5B3D}
\definecolor{nbsphinx-code-bg}{HTML}{F5F5F5}
\definecolor{nbsphinx-code-border}{HTML}{E0E0E0}
\definecolor{nbsphinx-stderr}{HTML}{FFDDDD}
% ANSI colors for output streams and traceback highlighting
\definecolor{ansi-black}{HTML}{3E424D}
\definecolor{ansi-black-intense}{HTML}{282C36}
\definecolor{ansi-red}{HTML}{E75C58}
\definecolor{ansi-red-intense}{HTML}{B22B31}
\definecolor{ansi-green}{HTML}{00A250}
\definecolor{ansi-green-intense}{HTML}{007427}
\definecolor{ansi-yellow}{HTML}{DDB62B}
\definecolor{ansi-yellow-intense}{HTML}{B27D12}
\definecolor{ansi-blue}{HTML}{208FFB}
\definecolor{ansi-blue-intense}{HTML}{0065CA}
\definecolor{ansi-magenta}{HTML}{D160C4}
\definecolor{ansi-magenta-intense}{HTML}{A03196}
\definecolor{ansi-cyan}{HTML}{60C6C8}
\definecolor{ansi-cyan-intense}{HTML}{258F8F}
\definecolor{ansi-white}{HTML}{C5C1B4}
\definecolor{ansi-white-intense}{HTML}{A1A6B2}
\definecolor{ansi-default-inverse-fg}{HTML}{FFFFFF}
\definecolor{ansi-default-inverse-bg}{HTML}{000000}

% Define an environment for non-plain-text code cell outputs (e.g. images)
\makeatletter
\newenvironment{nbsphinxfancyoutput}{%
    % Avoid fatal error with framed.sty if graphics too long to fit on one page
    \let\sphinxincludegraphics\nbsphinxincludegraphics
    \nbsphinx@image@maxheight\textheight
    \advance\nbsphinx@image@maxheight -2\fboxsep   % default \fboxsep 3pt
    \advance\nbsphinx@image@maxheight -2\fboxrule  % default \fboxrule 0.4pt
    \advance\nbsphinx@image@maxheight -\baselineskip
\def\nbsphinxfcolorbox{\spx@fcolorbox{nbsphinx-code-border}{white}}%
\def\FrameCommand{\nbsphinxfcolorbox\nbsphinxfancyaddprompt\@empty}%
\def\FirstFrameCommand{\nbsphinxfcolorbox\nbsphinxfancyaddprompt\sphinxVerbatim@Continues}%
\def\MidFrameCommand{\nbsphinxfcolorbox\sphinxVerbatim@Continued\sphinxVerbatim@Continues}%
\def\LastFrameCommand{\nbsphinxfcolorbox\sphinxVerbatim@Continued\@empty}%
\MakeFramed{\advance\hsize-\width\@totalleftmargin\z@\linewidth\hsize\@setminipage}%
\lineskip=1ex\lineskiplimit=1ex\raggedright%
}{\par\unskip\@minipagefalse\endMakeFramed}
\makeatother
\newbox\nbsphinxpromptbox
\def\nbsphinxfancyaddprompt{\ifvoid\nbsphinxpromptbox\else
    \kern\fboxrule\kern\fboxsep
    \copy\nbsphinxpromptbox
    \kern-\ht\nbsphinxpromptbox\kern-\dp\nbsphinxpromptbox
    \kern-\fboxsep\kern-\fboxrule\nointerlineskip
    \fi}
\newlength\nbsphinxcodecellspacing
\setlength{\nbsphinxcodecellspacing}{0pt}

% Define support macros for attaching opening and closing lines to notebooks
\newsavebox\nbsphinxbox
\makeatletter
\newcommand{\nbsphinxstartnotebook}[1]{%
    \par
    % measure needed space
    \setbox\nbsphinxbox\vtop{{#1\par}}
    % reserve some space at bottom of page, else start new page
    \needspace{\dimexpr2.5\baselineskip+\ht\nbsphinxbox+\dp\nbsphinxbox}
    % mimick vertical spacing from \section command
      \addpenalty\@secpenalty
      \@tempskipa 3.5ex \@plus 1ex \@minus .2ex\relax
      \addvspace\@tempskipa
      {\Large\@tempskipa\baselineskip
             \advance\@tempskipa-\prevdepth
             \advance\@tempskipa-\ht\nbsphinxbox
             \ifdim\@tempskipa>\z@
               \vskip \@tempskipa
             \fi}
    \unvbox\nbsphinxbox
    % if notebook starts with a \section, prevent it from adding extra space
    \@nobreaktrue\everypar{\@nobreakfalse\everypar{}}%
    % compensate the parskip which will get inserted by next paragraph
    \nobreak\vskip-\parskip
    % do not break here
    \nobreak
}% end of \nbsphinxstartnotebook

\newcommand{\nbsphinxstopnotebook}[1]{%
    \par
    % measure needed space
    \setbox\nbsphinxbox\vbox{{#1\par}}
    \nobreak % it updates page totals
    \dimen@\pagegoal
    \advance\dimen@-\pagetotal \advance\dimen@-\pagedepth
    \advance\dimen@-\ht\nbsphinxbox \advance\dimen@-\dp\nbsphinxbox
    \ifdim\dimen@<\z@
      % little space left
      \unvbox\nbsphinxbox
      \kern-.8\baselineskip
      \nobreak\vskip\z@\@plus1fil
      \penalty100
      \vskip\z@\@plus-1fil
      \kern.8\baselineskip
    \else
      \unvbox\nbsphinxbox
    \fi
}% end of \nbsphinxstopnotebook

% Ensure height of an included graphics fits in nbsphinxfancyoutput frame
\newdimen\nbsphinx@image@maxheight % set in nbsphinxfancyoutput environment
\newcommand*{\nbsphinxincludegraphics}[2][]{%
    \gdef\spx@includegraphics@options{#1}%
    \setbox\spx@image@box\hbox{\includegraphics[#1,draft]{#2}}%
    \in@false
    \ifdim \wd\spx@image@box>\linewidth
      \g@addto@macro\spx@includegraphics@options{,width=\linewidth}%
      \in@true
    \fi
    % no rotation, no need to worry about depth
    \ifdim \ht\spx@image@box>\nbsphinx@image@maxheight
      \g@addto@macro\spx@includegraphics@options{,height=\nbsphinx@image@maxheight}%
      \in@true
    \fi
    \ifin@
      \g@addto@macro\spx@includegraphics@options{,keepaspectratio}%
    \fi
    \setbox\spx@image@box\box\voidb@x % clear memory
    \expandafter\includegraphics\expandafter[\spx@includegraphics@options]{#2}%
}% end of "\MakeFrame"-safe variant of \sphinxincludegraphics
\makeatother

\makeatletter
\renewcommand*\sphinx@verbatim@nolig@list{\do\'\do\`}
\begingroup
\catcode`'=\active
\let\nbsphinx@noligs\@noligs
\g@addto@macro\nbsphinx@noligs{\let'\PYGZsq}
\endgroup
\makeatother
\renewcommand*\sphinxbreaksbeforeactivelist{\do\<\do\"\do\'}
\renewcommand*\sphinxbreaksafteractivelist{\do\.\do\,\do\:\do\;\do\?\do\!\do\/\do\>\do\-}
\makeatletter
\fvset{codes*=\sphinxbreaksattexescapedchars\do\^\^\let\@noligs\nbsphinx@noligs}
\makeatother



\title{pysubgroup Documentation}
\date{Apr 24, 2020}
\release{0.0.2}
\author{Florian Lemmerich}
\newcommand{\sphinxlogo}{\vbox{}}
\renewcommand{\releasename}{Release}
\makeindex
\begin{document}

\pagestyle{empty}
\sphinxmaketitle
\pagestyle{plain}
\sphinxtableofcontents
\pagestyle{normal}
\phantomsection\label{\detokenize{index::doc}}


\sphinxstylestrong{pysubgroup} is a Python package that enables subgroup discovery in
Python+pandas (scipy stack) data analysis environment.
It provides for a lightweight, easy\sphinxhyphen{}to\sphinxhyphen{}use, extensible and freely available
implementation of state\sphinxhyphen{}of\sphinxhyphen{}the\sphinxhyphen{}art algorithms, interestingness measures and presentation options.

As of 2018, this library is still in a prototype phase.
It has, however, been already successfully employed in active application projects.


\chapter{Pysubgroup}
\label{\detokenize{tutorials/introduction:Pysubgroup}}\label{\detokenize{tutorials/introduction::doc}}
\sphinxstylestrong{pysubgroup} is a Python package that enables subgroup discovery in Python+pandas (scipy stack) data analysis environment. It provides for a lightweight, easy\sphinxhyphen{}to\sphinxhyphen{}use, extensible and freely available implementation of state\sphinxhyphen{}of\sphinxhyphen{}the\sphinxhyphen{}art algorithms, interestingness measures and presentation options.

As of 2018, this library is still in a prototype phase. It has, however, been already succeesfully employed in active application projects.


\section{Subgroup Discovery}
\label{\detokenize{tutorials/introduction:Subgroup-Discovery}}
Subgroup Discovery is a well established data mining technique that allows you to identify patterns in your data. More precisely, the goal of subgroup discovery is to identify descriptions of data subsets that show an interesting distribution with respect to a pre\sphinxhyphen{}specified target concept. For example, given a dataset of patients in a hospital, we could be interested in subgroups of patients, for which a certain treatment X was successful. One example result could then be stated as:

\sphinxstyleemphasis{“While in general the operation is successful in only 60\% of the cases”, for the subgroup of female patients under 50 that also have been treated with drug d, the successrate was 82\%.”}

Here, a variable \sphinxstyleemphasis{operation success} is the target concept, the identified subgroup has the interpretable description \sphinxstyleemphasis{female=True AND age\textless{}50 AND drug\_D = True}. We call these single conditions (such as \sphinxstyleemphasis{female=True}) selection expressions or short \sphinxstyleemphasis{selectors}. The interesting behavior for this subgroup is that the distribution of the target concept differs significantly from the distribution in the overall general dataset. A discovered subgroup could also be seen as a rule:

\begin{sphinxVerbatim}[commandchars=\\\{\}]
\PYG{n}{female}\PYG{o}{=}\PYG{k+kc}{True} \PYG{n}{AND} \PYG{n}{age}\PYG{o}{\PYGZlt{}}\PYG{l+m+mi}{50} \PYG{n}{AND} \PYG{n}{drug\PYGZus{}D} \PYG{o}{=} \PYG{k+kc}{True} \PYG{o}{==}\PYG{o}{\PYGZgt{}} \PYG{n}{Operation\PYGZus{}outcome}\PYG{o}{=}\PYG{n}{SUCCESS}
\end{sphinxVerbatim}

Computationally, subgroup discovery is challenging since a large number of such conjunctive subgroup descriptions have to be considered. Of course, finding computable criteria, which subgroups are likely interesting to a user is also an eternal struggle. Therefore, a lot of literature has been devoted to the topic of subgroup discovery (including some of my own work). Recent overviews on the topic are for example:
\begin{itemize}
\item {} 
Herrera, Franciso, et al. “\sphinxhref{https://scholar.google.de/scholar?q=Herrera\%2C+Franciso\%2C+et+al.+\%E2\%80\%9CAn+overview+on+subgroup+discovery\%3A+foundations+and+applications.\%E2\%80\%9D+Knowledge+and+information+systems+29.3+(2011)\%3A+495-525.}{An overview on subgroup discovery: foundations and applications.}” Knowledge and information systems 29.3 (2011): 495\sphinxhyphen{}525.

\item {} 
Atzmueller, Martin. “\sphinxhref{https://scholar.google.de/scholar?q=Atzmueller\%2C+Martin.+\%E2\%80\%9CSubgroup+discovery.\%E2\%80\%9D+Wiley+Interdisciplinary+Reviews\%3A+Data+Mining+and+Knowledge+Discovery+5.1+(2015)\%3A+35-49.}{Subgroup discovery.}” Wiley Interdisciplinary Reviews: Data Mining and Knowledge Discovery 5.1 (2015): 35\sphinxhyphen{}49.

\item {} 
And of course, my point of view on the topic is \sphinxhref{https://opus.bibliothek.uni-wuerzburg.de/files/9781/Dissertation-Lemmerich.pdf}{summarized in my dissertation}:

\end{itemize}


\section{Prerequisites and Installation}
\label{\detokenize{tutorials/introduction:Prerequisites-and-Installation}}
pysubgroup is built to fit in the standard Python data analysis environment from the scipy\sphinxhyphen{}stack. Thus, it can be used just having pandas (including its dependencies numpy, scipy, and matplotlib) installed. Visualizations are carried out with the matplotlib library.

pysubgroup consists of pure Python code. Thus, you can simply download the code from the repository and copy it in your \sphinxcode{\sphinxupquote{site\sphinxhyphen{}packages}} directory. pysubgroup is also on PyPI and should be installable using:

\begin{sphinxVerbatim}[commandchars=\\\{\}]
\PYG{n}{pip} \PYG{n}{install} \PYG{n}{pysubgroup}
\end{sphinxVerbatim}


\section{How to use:}
\label{\detokenize{tutorials/introduction:How-to-use:}}
A simple use case (here using the well known \sphinxstyleemphasis{titanic} data) can be created in just a few lines of code:

{
\sphinxsetup{VerbatimColor={named}{nbsphinx-code-bg}}
\sphinxsetup{VerbatimBorderColor={named}{nbsphinx-code-border}}
\begin{sphinxVerbatim}[commandchars=\\\{\}]
\llap{\color{nbsphinxin}[1]:\,\hspace{\fboxrule}\hspace{\fboxsep}}\PYG{k+kn}{import} \PYG{n+nn}{pysubgroup} \PYG{k}{as} \PYG{n+nn}{ps}

\PYG{c+c1}{\PYGZsh{} Load the example dataset}
\PYG{k+kn}{from} \PYG{n+nn}{pysubgroup}\PYG{n+nn}{.}\PYG{n+nn}{tests}\PYG{n+nn}{.}\PYG{n+nn}{DataSets} \PYG{k}{import} \PYG{n}{get\PYGZus{}titanic\PYGZus{}data}
\PYG{n}{data} \PYG{o}{=} \PYG{n}{get\PYGZus{}titanic\PYGZus{}data}\PYG{p}{(}\PYG{p}{)}

\PYG{n}{target} \PYG{o}{=} \PYG{n}{ps}\PYG{o}{.}\PYG{n}{BinaryTarget} \PYG{p}{(}\PYG{l+s+s1}{\PYGZsq{}}\PYG{l+s+s1}{Survived}\PYG{l+s+s1}{\PYGZsq{}}\PYG{p}{,} \PYG{k+kc}{True}\PYG{p}{)}
\PYG{n}{searchspace} \PYG{o}{=} \PYG{n}{ps}\PYG{o}{.}\PYG{n}{create\PYGZus{}selectors}\PYG{p}{(}\PYG{n}{data}\PYG{p}{,} \PYG{n}{ignore}\PYG{o}{=}\PYG{p}{[}\PYG{l+s+s1}{\PYGZsq{}}\PYG{l+s+s1}{Survived}\PYG{l+s+s1}{\PYGZsq{}}\PYG{p}{]}\PYG{p}{)}
\PYG{n}{task} \PYG{o}{=} \PYG{n}{ps}\PYG{o}{.}\PYG{n}{SubgroupDiscoveryTask} \PYG{p}{(}
    \PYG{n}{data}\PYG{p}{,}
    \PYG{n}{target}\PYG{p}{,}
    \PYG{n}{searchspace}\PYG{p}{,}
    \PYG{n}{result\PYGZus{}set\PYGZus{}size}\PYG{o}{=}\PYG{l+m+mi}{5}\PYG{p}{,}
    \PYG{n}{depth}\PYG{o}{=}\PYG{l+m+mi}{2}\PYG{p}{,}
    \PYG{n}{qf}\PYG{o}{=}\PYG{n}{ps}\PYG{o}{.}\PYG{n}{WRAccQF}\PYG{p}{(}\PYG{p}{)}\PYG{p}{)}
\PYG{n}{result} \PYG{o}{=} \PYG{n}{ps}\PYG{o}{.}\PYG{n}{BeamSearch}\PYG{p}{(}\PYG{p}{)}\PYG{o}{.}\PYG{n}{execute}\PYG{p}{(}\PYG{n}{task}\PYG{p}{)}

\end{sphinxVerbatim}
}

The first two lines imports \sphinxstyleemphasis{pysubgroup} package. The following lines load an example dataset (the popular titanic dataset).

Therafter, we define a target, i.e., the property we are mainly interested in (\_‘survived’\}. Then, we define the searchspace as a list of basic selectors. Descriptions are built from this searchspace. We can create this list manually, or use an utility function. Next, we create a SubgroupDiscoveryTask object that encapsulates what we want to find in our search. In particular, that comprises the target, the search space, the depth of the search (maximum numbers of selectors combined in a
subgroup description), and the interestingness measure for candidate scoring (here, the Weighted Relative Accuracy measure).

The last line executes the defined task by performing a search with an algorithm—in this case beam search. The result of this algorithm execution is stored in a SubgroupDiscoveryResults object.

To just print the result, we could for example do:

{
\sphinxsetup{VerbatimColor={named}{nbsphinx-code-bg}}
\sphinxsetup{VerbatimBorderColor={named}{nbsphinx-code-border}}
\begin{sphinxVerbatim}[commandchars=\\\{\}]
\llap{\color{nbsphinxin}[5]:\,\hspace{\fboxrule}\hspace{\fboxsep}}\PYG{n}{result}\PYG{o}{.}\PYG{n}{to\PYGZus{}dataframe}\PYG{p}{(}\PYG{p}{)}
\end{sphinxVerbatim}
}

{

\kern-\sphinxverbatimsmallskipamount\kern-\baselineskip
\kern+\FrameHeightAdjust\kern-\fboxrule
\vspace{\nbsphinxcodecellspacing}

\sphinxsetup{VerbatimColor={named}{white}}
\sphinxsetup{VerbatimBorderColor={named}{nbsphinx-code-border}}
\begin{sphinxVerbatim}[commandchars=\\\{\}]
\llap{\color{nbsphinxout}[5]:\,\hspace{\fboxrule}\hspace{\fboxsep}}    quality                       description
0  0.132150                     Sex=='female'
1  0.101331        Parch==0 AND Sex=='female'
2  0.079142    Sex=='female' AND SibSp: [0:1[
3  0.077663  Cabin.isnull() AND Sex=='female'
4  0.071746   Embarked=='S' AND Sex=='female'
\end{sphinxVerbatim}
}


\chapter{How pysubgroup works}
\label{\detokenize{tutorials/introduction2:How-pysubgroup-works}}\label{\detokenize{tutorials/introduction2::doc}}
just ignore the following code block for non technical people, but have a look at the example dataFrame (table) that we created

{
\sphinxsetup{VerbatimColor={named}{nbsphinx-code-bg}}
\sphinxsetup{VerbatimBorderColor={named}{nbsphinx-code-border}}
\begin{sphinxVerbatim}[commandchars=\\\{\}]
\llap{\color{nbsphinxin}[1]:\,\hspace{\fboxrule}\hspace{\fboxsep}}\PYG{k+kn}{import} \PYG{n+nn}{pysubgroup} \PYG{k}{as} \PYG{n+nn}{ps}
\PYG{k+kn}{import} \PYG{n+nn}{pandas} \PYG{k}{as} \PYG{n+nn}{pd}
\PYG{n}{pd}\PYG{o}{.}\PYG{n}{set\PYGZus{}option}\PYG{p}{(}\PYG{l+s+s1}{\PYGZsq{}}\PYG{l+s+s1}{display.width}\PYG{l+s+s1}{\PYGZsq{}}\PYG{p}{,} \PYG{l+m+mi}{1000}\PYG{p}{)}
\PYG{n}{pd}\PYG{o}{.}\PYG{n}{set\PYGZus{}option}\PYG{p}{(}\PYG{l+s+s1}{\PYGZsq{}}\PYG{l+s+s1}{display.max\PYGZus{}colwidth}\PYG{l+s+s1}{\PYGZsq{}}\PYG{p}{,} \PYG{l+m+mi}{300}\PYG{p}{)}
\PYG{n}{products} \PYG{o}{=} \PYG{p}{[}\PYG{l+s+s1}{\PYGZsq{}}\PYG{l+s+s1}{toast}\PYG{l+s+s1}{\PYGZsq{}}\PYG{p}{,} \PYG{l+s+s1}{\PYGZsq{}}\PYG{l+s+s1}{bread}\PYG{l+s+s1}{\PYGZsq{}}\PYG{p}{,} \PYG{l+s+s1}{\PYGZsq{}}\PYG{l+s+s1}{bread}\PYG{l+s+s1}{\PYGZsq{}}\PYG{p}{,} \PYG{l+s+s1}{\PYGZsq{}}\PYG{l+s+s1}{toast}\PYG{l+s+s1}{\PYGZsq{}}\PYG{p}{,} \PYG{l+s+s1}{\PYGZsq{}}\PYG{l+s+s1}{bread}\PYG{l+s+s1}{\PYGZsq{}}\PYG{p}{,} \PYG{l+s+s1}{\PYGZsq{}}\PYG{l+s+s1}{bread}\PYG{l+s+s1}{\PYGZsq{}}\PYG{p}{,}\PYG{l+s+s1}{\PYGZsq{}}\PYG{l+s+s1}{toast}\PYG{l+s+s1}{\PYGZsq{}}\PYG{p}{,} \PYG{l+s+s1}{\PYGZsq{}}\PYG{l+s+s1}{bread}\PYG{l+s+s1}{\PYGZsq{}}\PYG{p}{,} \PYG{l+s+s1}{\PYGZsq{}}\PYG{l+s+s1}{bread}\PYG{l+s+s1}{\PYGZsq{}}\PYG{p}{,} \PYG{l+s+s1}{\PYGZsq{}}\PYG{l+s+s1}{pizza}\PYG{l+s+s1}{\PYGZsq{}}\PYG{p}{]}
\PYG{n}{amount} \PYG{o}{=}  \PYG{p}{[}\PYG{l+m+mi}{100}\PYG{p}{,}\PYG{l+m+mi}{1000}\PYG{p}{,}\PYG{l+m+mi}{100}\PYG{p}{,}\PYG{l+m+mi}{1000}\PYG{p}{,}\PYG{l+m+mi}{10000}\PYG{p}{,}\PYG{l+m+mi}{540}\PYG{p}{,}\PYG{l+m+mi}{750}\PYG{p}{,}\PYG{l+m+mi}{860}\PYG{p}{,}\PYG{l+m+mi}{350}\PYG{p}{,}\PYG{l+m+mi}{400}\PYG{p}{]}
\PYG{n}{was\PYGZus{}fraud} \PYG{o}{=}  \PYG{p}{[}\PYG{l+m+mi}{1}\PYG{p}{,}\PYG{l+m+mi}{0}\PYG{p}{,}\PYG{l+m+mi}{0}\PYG{p}{,}\PYG{l+m+mi}{0}\PYG{p}{,}\PYG{l+m+mi}{1}\PYG{p}{,}\PYG{l+m+mi}{1}\PYG{p}{,}\PYG{l+m+mi}{1}\PYG{p}{,}\PYG{l+m+mi}{0}\PYG{p}{,}\PYG{l+m+mi}{0}\PYG{p}{,}\PYG{l+m+mi}{0}\PYG{p}{]}
\PYG{n}{day\PYGZus{}of\PYGZus{}month} \PYG{o}{=} \PYG{p}{[}\PYG{l+m+mi}{15}\PYG{p}{,}\PYG{l+m+mi}{20}\PYG{p}{,}\PYG{l+m+mi}{30}\PYG{p}{,}\PYG{l+m+mi}{17}\PYG{p}{,}\PYG{l+m+mi}{12}\PYG{p}{,}\PYG{l+m+mi}{7}\PYG{p}{,}\PYG{l+m+mi}{11}\PYG{p}{,}\PYG{l+m+mi}{14}\PYG{p}{,}\PYG{l+m+mi}{20}\PYG{p}{,}\PYG{l+m+mi}{27}\PYG{p}{]}
\PYG{n}{is\PYGZus{}gold} \PYG{o}{=} \PYG{p}{[}\PYG{l+m+mi}{1}\PYG{p}{,}\PYG{l+m+mi}{0}\PYG{p}{,}\PYG{l+m+mi}{1}\PYG{p}{,}\PYG{l+m+mi}{0}\PYG{p}{,}\PYG{l+m+mi}{0}\PYG{p}{,}\PYG{l+m+mi}{0}\PYG{p}{,}\PYG{l+m+mi}{1}\PYG{p}{,}\PYG{l+m+mi}{1}\PYG{p}{,}\PYG{l+m+mi}{1}\PYG{p}{,}\PYG{l+m+mi}{0}\PYG{p}{]}
\PYG{n}{df} \PYG{o}{=} \PYG{n}{pd}\PYG{o}{.}\PYG{n}{DataFrame}\PYG{o}{.}\PYG{n}{from\PYGZus{}dict}\PYG{p}{(}\PYG{p}{\PYGZob{}}\PYG{l+s+s1}{\PYGZsq{}}\PYG{l+s+s1}{product}\PYG{l+s+s1}{\PYGZsq{}}\PYG{p}{:} \PYG{n}{products}\PYG{p}{,} \PYG{l+s+s1}{\PYGZsq{}}\PYG{l+s+s1}{was\PYGZus{}fraud}\PYG{l+s+s1}{\PYGZsq{}}\PYG{p}{:} \PYG{n}{was\PYGZus{}fraud}\PYG{p}{,} \PYG{l+s+s1}{\PYGZsq{}}\PYG{l+s+s1}{amount}\PYG{l+s+s1}{\PYGZsq{}}\PYG{p}{:} \PYG{n}{amount}\PYG{p}{,} \PYG{l+s+s1}{\PYGZsq{}}\PYG{l+s+s1}{day\PYGZus{}of\PYGZus{}month}\PYG{l+s+s1}{\PYGZsq{}} \PYG{p}{:} \PYG{n}{day\PYGZus{}of\PYGZus{}month}\PYG{p}{,} \PYG{l+s+s1}{\PYGZsq{}}\PYG{l+s+s1}{is\PYGZus{}gold}\PYG{l+s+s1}{\PYGZsq{}}\PYG{p}{:}\PYG{n}{is\PYGZus{}gold}\PYG{p}{\PYGZcb{}}\PYG{p}{)}
\PYG{n}{df}
\end{sphinxVerbatim}
}

{

\kern-\sphinxverbatimsmallskipamount\kern-\baselineskip
\kern+\FrameHeightAdjust\kern-\fboxrule
\vspace{\nbsphinxcodecellspacing}

\sphinxsetup{VerbatimColor={named}{white}}
\sphinxsetup{VerbatimBorderColor={named}{nbsphinx-code-border}}
\begin{sphinxVerbatim}[commandchars=\\\{\}]
\llap{\color{nbsphinxout}[1]:\,\hspace{\fboxrule}\hspace{\fboxsep}}  product  was\_fraud  amount  day\_of\_month  is\_gold
0   toast          1     100            15        1
1   bread          0    1000            20        0
2   bread          0     100            30        1
3   toast          0    1000            17        0
4   bread          1   10000            12        0
5   bread          1     540             7        0
6   toast          1     750            11        1
7   bread          0     860            14        1
8   bread          0     350            20        1
9   pizza          0     400            27        0
\end{sphinxVerbatim}
}

{
\sphinxsetup{VerbatimColor={named}{nbsphinx-code-bg}}
\sphinxsetup{VerbatimBorderColor={named}{nbsphinx-code-border}}
\begin{sphinxVerbatim}[commandchars=\\\{\}]
\llap{\color{nbsphinxin}[2]:\,\hspace{\fboxrule}\hspace{\fboxsep}}\PYG{n}{target} \PYG{o}{=} \PYG{n}{ps}\PYG{o}{.}\PYG{n}{BinaryTarget}\PYG{p}{(}\PYG{l+s+s1}{\PYGZsq{}}\PYG{l+s+s1}{was\PYGZus{}fraud}\PYG{l+s+s1}{\PYGZsq{}}\PYG{p}{,} \PYG{l+m+mi}{1}\PYG{p}{)} \PYG{c+c1}{\PYGZsh{} we are looking for fraud=1}
\PYG{n}{search\PYGZus{}space} \PYG{o}{=} \PYG{n}{ps}\PYG{o}{.}\PYG{n}{create\PYGZus{}selectors}\PYG{p}{(}\PYG{n}{df}\PYG{p}{,} \PYG{n}{ignore}\PYG{o}{=}\PYG{l+s+s1}{\PYGZsq{}}\PYG{l+s+s1}{was\PYGZus{}fraud}\PYG{l+s+s1}{\PYGZsq{}}\PYG{p}{)} \PYG{c+c1}{\PYGZsh{} define what we are looking for}
\PYG{n}{search\PYGZus{}space}
\end{sphinxVerbatim}
}

{

\kern-\sphinxverbatimsmallskipamount\kern-\baselineskip
\kern+\FrameHeightAdjust\kern-\fboxrule
\vspace{\nbsphinxcodecellspacing}

\sphinxsetup{VerbatimColor={named}{white}}
\sphinxsetup{VerbatimBorderColor={named}{nbsphinx-code-border}}
\begin{sphinxVerbatim}[commandchars=\\\{\}]
\llap{\color{nbsphinxout}[2]:\,\hspace{\fboxrule}\hspace{\fboxsep}}[product=='toast',
 product=='bread',
 product=='pizza',
 amount<350,
 amount: [350:540[,
 amount: [540:860[,
 amount: [860:1000[,
 amount>=1000,
 day\_of\_month<12,
 day\_of\_month: [12:15[,
 day\_of\_month: [15:20[,
 day\_of\_month: [20:27[,
 day\_of\_month>=27,
 is\_gold==0,
 is\_gold==1]
\end{sphinxVerbatim}
}

As you can see, pysubgroup created selectors for us, treating nominal columns (product, is\_gold) different from numeric columns (where it uses intervals selectors) you can also create your own selectors see below

{
\sphinxsetup{VerbatimColor={named}{nbsphinx-code-bg}}
\sphinxsetup{VerbatimBorderColor={named}{nbsphinx-code-border}}
\begin{sphinxVerbatim}[commandchars=\\\{\}]
\llap{\color{nbsphinxin}[3]:\,\hspace{\fboxrule}\hspace{\fboxsep}}\PYG{n}{my\PYGZus{}selector} \PYG{o}{=} \PYG{n}{ps}\PYG{o}{.}\PYG{n}{IntervalSelector}\PYG{p}{(}\PYG{l+s+s1}{\PYGZsq{}}\PYG{l+s+s1}{amount}\PYG{l+s+s1}{\PYGZsq{}}\PYG{p}{,} \PYG{l+m+mi}{500}\PYG{p}{,} \PYG{l+m+mi}{25000}\PYG{p}{)}
\PYG{n+nb}{print}\PYG{p}{(}\PYG{n}{my\PYGZus{}selector}\PYG{p}{)}
\PYG{n}{search\PYGZus{}space}\PYG{o}{.}\PYG{n}{append}\PYG{p}{(}\PYG{n}{my\PYGZus{}selector}\PYG{p}{)} \PYG{c+c1}{\PYGZsh{} add my selector to searchspace}
\end{sphinxVerbatim}
}

{

\kern-\sphinxverbatimsmallskipamount\kern-\baselineskip
\kern+\FrameHeightAdjust\kern-\fboxrule
\vspace{\nbsphinxcodecellspacing}

\sphinxsetup{VerbatimColor={named}{white}}
\sphinxsetup{VerbatimBorderColor={named}{nbsphinx-code-border}}
\begin{sphinxVerbatim}[commandchars=\\\{\}]
amount: [500:25000[
\end{sphinxVerbatim}
}

Now that we have defined where we want to search we write all that information into a task object:

{
\sphinxsetup{VerbatimColor={named}{nbsphinx-code-bg}}
\sphinxsetup{VerbatimBorderColor={named}{nbsphinx-code-border}}
\begin{sphinxVerbatim}[commandchars=\\\{\}]
\llap{\color{nbsphinxin}[4]:\,\hspace{\fboxrule}\hspace{\fboxsep}}\PYG{n}{quality\PYGZus{}function} \PYG{o}{=} \PYG{n}{ps}\PYG{o}{.}\PYG{n}{StandardQF}\PYG{p}{(}\PYG{l+m+mi}{0}\PYG{p}{)} \PYG{c+c1}{\PYGZsh{} Looks for subgroups with highest true positives ratio}
\PYG{n}{min\PYGZus{}quality} \PYG{o}{=} \PYG{l+m+mf}{0.2} \PYG{c+c1}{\PYGZsh{} Minimum required quality = min true positive ratio}
\PYG{n}{task} \PYG{o}{=} \PYG{n}{ps}\PYG{o}{.}\PYG{n}{SubgroupDiscoveryTask}\PYG{p}{(}\PYG{n}{df}\PYG{p}{,} \PYG{n}{target}\PYG{p}{,} \PYG{n}{search\PYGZus{}space}\PYG{p}{,} \PYG{n}{quality\PYGZus{}function}\PYG{p}{,} \PYG{n}{result\PYGZus{}set\PYGZus{}size}\PYG{o}{=}\PYG{l+m+mi}{10}\PYG{p}{,} \PYG{n}{depth}\PYG{o}{=}\PYG{l+m+mi}{3}\PYG{p}{,} \PYG{n}{min\PYGZus{}quality}\PYG{o}{=}\PYG{n}{min\PYGZus{}quality}\PYG{p}{)}
\end{sphinxVerbatim}
}

now that we have that task object we can run the algorithm (Here Depth first search)

{
\sphinxsetup{VerbatimColor={named}{nbsphinx-code-bg}}
\sphinxsetup{VerbatimBorderColor={named}{nbsphinx-code-border}}
\begin{sphinxVerbatim}[commandchars=\\\{\}]
\llap{\color{nbsphinxin}[5]:\,\hspace{\fboxrule}\hspace{\fboxsep}}\PYG{n}{result} \PYG{o}{=} \PYG{n}{ps}\PYG{o}{.}\PYG{n}{SimpleDFS}\PYG{p}{(}\PYG{p}{)}\PYG{o}{.}\PYG{n}{execute}\PYG{p}{(}\PYG{n}{task}\PYG{p}{)} \PYG{c+c1}{\PYGZsh{} Run the algorithm}
\PYG{n}{result}\PYG{o}{.}\PYG{n}{to\PYGZus{}dataframe}\PYG{p}{(}\PYG{n}{include\PYGZus{}info}\PYG{o}{=}\PYG{k+kc}{True}\PYG{p}{)} \PYG{c+c1}{\PYGZsh{} Show the output}
\end{sphinxVerbatim}
}

{

\kern-\sphinxverbatimsmallskipamount\kern-\baselineskip
\kern+\FrameHeightAdjust\kern-\fboxrule
\vspace{\nbsphinxcodecellspacing}

\sphinxsetup{VerbatimColor={named}{white}}
\sphinxsetup{VerbatimBorderColor={named}{nbsphinx-code-border}}
\begin{sphinxVerbatim}[commandchars=\\\{\}]
\llap{\color{nbsphinxout}[5]:\,\hspace{\fboxrule}\hspace{\fboxsep}}   quality                                                     description  size\_sg  size\_dataset  positives\_sg  positives\_dataset  size\_complement  relative\_size\_sg  relative\_size\_complement  coverage\_sg  coverage\_complement  target\_share\_sg  target\_share\_complement  target\_share\_dataset  lift
0      0.6  amount: [500:25000[ AND amount: [540:860[ AND product=='toast'        1            10             1                  4                9               0.1                       0.9         0.25                 0.75              1.0                 0.333333                   0.4   2.5
1      0.6    amount: [500:25000[ AND day\_of\_month<12 AND product=='toast'        1            10             1                  4                9               0.1                       0.9         0.25                 0.75              1.0                 0.333333                   0.4   2.5
2      0.6      amount: [540:860[ AND day\_of\_month<12 AND product=='toast'        1            10             1                  4                9               0.1                       0.9         0.25                 0.75              1.0                 0.333333                   0.4   2.5
3      0.6                                 amount<350 AND product=='toast'        1            10             1                  4                9               0.1                       0.9         0.25                 0.75              1.0                 0.333333                   0.4   2.5
4      0.6                          amount: [540:860[ AND product=='toast'        1            10             1                  4                9               0.1                       0.9         0.25                 0.75              1.0                 0.333333                   0.4   2.5
5      0.6       amount<350 AND day\_of\_month: [15:20[ AND product=='toast'        1            10             1                  4                9               0.1                       0.9         0.25                 0.75              1.0                 0.333333                   0.4   2.5
6      0.6           amount: [540:860[ AND is\_gold==1 AND product=='toast'        1            10             1                  4                9               0.1                       0.9         0.25                 0.75              1.0                 0.333333                   0.4   2.5
7      0.6             day\_of\_month<12 AND is\_gold==1 AND product=='toast'        1            10             1                  4                9               0.1                       0.9         0.25                 0.75              1.0                 0.333333                   0.4   2.5
8      0.6                            day\_of\_month<12 AND product=='toast'        1            10             1                  4                9               0.1                       0.9         0.25                 0.75              1.0                 0.333333                   0.4   2.5
9      0.6                  amount<350 AND is\_gold==1 AND product=='toast'        1            10             1                  4                9               0.1                       0.9         0.25                 0.75              1.0                 0.333333                   0.4   2.5
\end{sphinxVerbatim}
}

{
\sphinxsetup{VerbatimColor={named}{nbsphinx-code-bg}}
\sphinxsetup{VerbatimBorderColor={named}{nbsphinx-code-border}}
\begin{sphinxVerbatim}[commandchars=\\\{\}]
\llap{\color{nbsphinxin}[6]:\,\hspace{\fboxrule}\hspace{\fboxsep}}\PYG{c+c1}{\PYGZsh{} Visualize resultset, see there is quite some redundancy}
\PYG{k+kn}{import} \PYG{n+nn}{matplotlib}\PYG{n+nn}{.}\PYG{n+nn}{pyplot} \PYG{k}{as} \PYG{n+nn}{plt}
\PYG{n}{plt}\PYG{o}{.}\PYG{n}{matshow}\PYG{p}{(}\PYG{n}{result}\PYG{o}{.}\PYG{n}{supportSetVisualization}\PYG{p}{(}\PYG{p}{)}\PYG{p}{)}
\end{sphinxVerbatim}
}

{

\kern-\sphinxverbatimsmallskipamount\kern-\baselineskip
\kern+\FrameHeightAdjust\kern-\fboxrule
\vspace{\nbsphinxcodecellspacing}

\sphinxsetup{VerbatimColor={named}{white}}
\sphinxsetup{VerbatimBorderColor={named}{nbsphinx-code-border}}
\begin{sphinxVerbatim}[commandchars=\\\{\}]
Discarding 8 entities that are not covered
\end{sphinxVerbatim}
}

{

\kern-\sphinxverbatimsmallskipamount\kern-\baselineskip
\kern+\FrameHeightAdjust\kern-\fboxrule
\vspace{\nbsphinxcodecellspacing}

\sphinxsetup{VerbatimColor={named}{white}}
\sphinxsetup{VerbatimBorderColor={named}{nbsphinx-code-border}}
\begin{sphinxVerbatim}[commandchars=\\\{\}]
\llap{\color{nbsphinxout}[6]:\,\hspace{\fboxrule}\hspace{\fboxsep}}<matplotlib.image.AxesImage at 0x2c8420d7b48>
\end{sphinxVerbatim}
}

\hrule height -\fboxrule\relax
\vspace{\nbsphinxcodecellspacing}

\makeatletter\setbox\nbsphinxpromptbox\box\voidb@x\makeatother

\begin{nbsphinxfancyoutput}

\noindent\sphinxincludegraphics[width=206\sphinxpxdimen,height=927\sphinxpxdimen]{{tutorials_introduction2_9_2}.png}

\end{nbsphinxfancyoutput}

{
\sphinxsetup{VerbatimColor={named}{nbsphinx-code-bg}}
\sphinxsetup{VerbatimBorderColor={named}{nbsphinx-code-border}}
\begin{sphinxVerbatim}[commandchars=\\\{\}]
\llap{\color{nbsphinxin}[7]:\,\hspace{\fboxrule}\hspace{\fboxsep}}\PYG{c+c1}{\PYGZsh{} we can already avoid redundancy a little but its quite slow!}
\PYG{n}{quality\PYGZus{}function} \PYG{o}{=} \PYG{n}{ps}\PYG{o}{.}\PYG{n}{GeneralizationAwareQF}\PYG{p}{(}\PYG{n}{ps}\PYG{o}{.}\PYG{n}{StandardQF}\PYG{p}{(}\PYG{l+m+mi}{0}\PYG{p}{)}\PYG{p}{)} \PYG{c+c1}{\PYGZsh{} Looks for subgroups with highest true positives ratio, trying to avoid redundancy}
\PYG{n}{task} \PYG{o}{=} \PYG{n}{ps}\PYG{o}{.}\PYG{n}{SubgroupDiscoveryTask}\PYG{p}{(}\PYG{n}{df}\PYG{p}{,} \PYG{n}{target}\PYG{p}{,} \PYG{n}{search\PYGZus{}space}\PYG{p}{,} \PYG{n}{quality\PYGZus{}function}\PYG{p}{,} \PYG{n}{result\PYGZus{}set\PYGZus{}size}\PYG{o}{=}\PYG{l+m+mi}{10}\PYG{p}{,} \PYG{n}{depth}\PYG{o}{=}\PYG{l+m+mi}{3}\PYG{p}{,} \PYG{n}{min\PYGZus{}quality}\PYG{o}{=}\PYG{n}{min\PYGZus{}quality}\PYG{p}{)}
\PYG{n}{result} \PYG{o}{=} \PYG{n}{ps}\PYG{o}{.}\PYG{n}{SimpleDFS}\PYG{p}{(}\PYG{p}{)}\PYG{o}{.}\PYG{n}{execute}\PYG{p}{(}\PYG{n}{task}\PYG{p}{)}
\PYG{n}{plt}\PYG{o}{.}\PYG{n}{matshow}\PYG{p}{(}\PYG{n}{result}\PYG{o}{.}\PYG{n}{supportSetVisualization}\PYG{p}{(}\PYG{p}{)}\PYG{p}{)}
\end{sphinxVerbatim}
}

{

\kern-\sphinxverbatimsmallskipamount\kern-\baselineskip
\kern+\FrameHeightAdjust\kern-\fboxrule
\vspace{\nbsphinxcodecellspacing}

\sphinxsetup{VerbatimColor={named}{white}}
\sphinxsetup{VerbatimBorderColor={named}{nbsphinx-code-border}}
\begin{sphinxVerbatim}[commandchars=\\\{\}]
Discarding 4 entities that are not covered
\end{sphinxVerbatim}
}

{

\kern-\sphinxverbatimsmallskipamount\kern-\baselineskip
\kern+\FrameHeightAdjust\kern-\fboxrule
\vspace{\nbsphinxcodecellspacing}

\sphinxsetup{VerbatimColor={named}{white}}
\sphinxsetup{VerbatimBorderColor={named}{nbsphinx-code-border}}
\begin{sphinxVerbatim}[commandchars=\\\{\}]
\llap{\color{nbsphinxout}[7]:\,\hspace{\fboxrule}\hspace{\fboxsep}}<matplotlib.image.AxesImage at 0x2c84212da48>
\end{sphinxVerbatim}
}

\hrule height -\fboxrule\relax
\vspace{\nbsphinxcodecellspacing}

\makeatletter\setbox\nbsphinxpromptbox\box\voidb@x\makeatother

\begin{nbsphinxfancyoutput}

\noindent\sphinxincludegraphics[width=250\sphinxpxdimen,height=406\sphinxpxdimen]{{tutorials_introduction2_10_2}.png}

\end{nbsphinxfancyoutput}

{
\sphinxsetup{VerbatimColor={named}{nbsphinx-code-bg}}
\sphinxsetup{VerbatimBorderColor={named}{nbsphinx-code-border}}
\begin{sphinxVerbatim}[commandchars=\\\{\}]
\llap{\color{nbsphinxin}[8]:\,\hspace{\fboxrule}\hspace{\fboxsep}}\PYG{n}{result}\PYG{o}{.}\PYG{n}{to\PYGZus{}dataframe}\PYG{p}{(}\PYG{n}{include\PYGZus{}info}\PYG{o}{=}\PYG{k+kc}{True}\PYG{p}{)} \PYG{c+c1}{\PYGZsh{} Show the output, avoiding redundancy}
\end{sphinxVerbatim}
}

{

\kern-\sphinxverbatimsmallskipamount\kern-\baselineskip
\kern+\FrameHeightAdjust\kern-\fboxrule
\vspace{\nbsphinxcodecellspacing}

\sphinxsetup{VerbatimColor={named}{white}}
\sphinxsetup{VerbatimBorderColor={named}{nbsphinx-code-border}}
\begin{sphinxVerbatim}[commandchars=\\\{\}]
\llap{\color{nbsphinxout}[8]:\,\hspace{\fboxrule}\hspace{\fboxsep}}    quality                             description  size\_sg  size\_dataset  positives\_sg  positives\_dataset  size\_complement  relative\_size\_sg  relative\_size\_complement  coverage\_sg  coverage\_complement  target\_share\_sg  target\_share\_complement  target\_share\_dataset      lift
0  0.600000                       amount: [540:860[        2            10             2                  4                8               0.2                       0.8         0.50                 0.50         1.000000                 0.250000                   0.4  2.500000
1  0.600000                         day\_of\_month<12        2            10             2                  4                8               0.2                       0.8         0.50                 0.50         1.000000                 0.250000                   0.4  2.500000
2  0.500000    amount<350 AND day\_of\_month: [15:20[        1            10             1                  4                9               0.1                       0.9         0.25                 0.75         1.000000                 0.333333                   0.4  2.500000
3  0.500000  amount>=1000 AND day\_of\_month: [12:15[        1            10             1                  4                9               0.1                       0.9         0.25                 0.75         1.000000                 0.333333                   0.4  2.500000
4  0.500000    day\_of\_month: [12:15[ AND is\_gold==0        1            10             1                  4                9               0.1                       0.9         0.25                 0.75         1.000000                 0.333333                   0.4  2.500000
5  0.500000    day\_of\_month: [15:20[ AND is\_gold==1        1            10             1                  4                9               0.1                       0.9         0.25                 0.75         1.000000                 0.333333                   0.4  2.500000
6  0.333333         is\_gold==1 AND product=='toast'        2            10             2                  4                8               0.2                       0.8         0.50                 0.50         1.000000                 0.250000                   0.4  2.500000
7  0.333333         amount<350 AND product=='toast'        1            10             1                  4                9               0.1                       0.9         0.25                 0.75         1.000000                 0.333333                   0.4  2.500000
8  0.266667         is\_gold==0 AND product=='bread'        3            10             2                  4                7               0.3                       0.7         0.50                 0.50         0.666667                 0.285714                   0.4  1.666667
9  0.266667                        product=='toast'        3            10             2                  4                7               0.3                       0.7         0.50                 0.50         0.666667                 0.285714                   0.4  1.666667
\end{sphinxVerbatim}
}

{
\sphinxsetup{VerbatimColor={named}{nbsphinx-code-bg}}
\sphinxsetup{VerbatimBorderColor={named}{nbsphinx-code-border}}
\begin{sphinxVerbatim}[commandchars=\\\{\}]
\llap{\color{nbsphinxin}[ ]:\,\hspace{\fboxrule}\hspace{\fboxsep}}
\end{sphinxVerbatim}
}


\chapter{Selectors}
\label{\detokenize{Selectors:selectors}}\label{\detokenize{Selectors::doc}}
Selectors are objects that if applied to a dataset yield a set of instances. If an instance is retured from a selector we say that the selectors covers that instance.
While the term selectors usually only refers to basic selectors, conjunctions and disjunctions as well as negated selectors are also in a general sense selectors. Broadly speaking anything that implements the code:\sphinxtitleref{covers} function is a selector.
We will first introduce the frequently used basic selectors and thereafter the more general selectors that are the conjunction and disjunction. We conclude the chapter by showing how to implement a selectors yourself.


\section{Basic Selectors}
\label{\detokenize{Selectors:basic-selectors}}
The pysubgroup package provides two basic selectors: The EqualitySelector and the IntervalSelector.
Lets start by exploring the EqualitySelector:

\begin{sphinxVerbatim}[commandchars=\\\{\}]
\PYG{k+kn}{import} \PYG{n+nn}{pysubgroup} \PYG{k}{as} \PYG{n+nn}{ps}
\PYG{k+kn}{import} \PYG{n+nn}{pandas} \PYG{k}{as} \PYG{n+nn}{pd}

\PYG{c+c1}{\PYGZsh{} create dataset}
\PYG{n}{first\PYGZus{}names} \PYG{o}{=} \PYG{p}{[}\PYG{l+s+s1}{\PYGZsq{}}\PYG{l+s+s1}{Alex}\PYG{l+s+s1}{\PYGZsq{}}\PYG{p}{,} \PYG{l+s+s1}{\PYGZsq{}}\PYG{l+s+s1}{Anna}\PYG{l+s+s1}{\PYGZsq{}}\PYG{p}{,} \PYG{l+s+s1}{\PYGZsq{}}\PYG{l+s+s1}{Alex}\PYG{l+s+s1}{\PYGZsq{}}\PYG{p}{]}
\PYG{n}{sur\PYGZus{}names} \PYG{o}{=} \PYG{p}{[}\PYG{l+s+s1}{\PYGZsq{}}\PYG{l+s+s1}{Smith}\PYG{l+s+s1}{\PYGZsq{}}\PYG{p}{,} \PYG{l+s+s1}{\PYGZsq{}}\PYG{l+s+s1}{Johnson}\PYG{l+s+s1}{\PYGZsq{}}\PYG{p}{,} \PYG{l+s+s1}{\PYGZsq{}}\PYG{l+s+s1}{Williams}\PYG{l+s+s1}{\PYGZsq{}}\PYG{p}{]}
\PYG{n}{ages} \PYG{o}{=}  \PYG{p}{[}\PYG{l+m+mi}{40}\PYG{p}{,} \PYG{l+m+mi}{25}\PYG{p}{,} \PYG{l+m+mi}{32}\PYG{p}{]}
\PYG{n}{df} \PYG{o}{=} \PYG{n}{pd}\PYG{o}{.}\PYG{n}{DataFrame}\PYG{o}{.}\PYG{n}{from\PYGZus{}dict}\PYG{p}{(}\PYG{p}{\PYGZob{}}\PYG{l+s+s1}{\PYGZsq{}}\PYG{l+s+s1}{First\PYGZus{}name}\PYG{l+s+s1}{\PYGZsq{}}\PYG{p}{:}\PYG{n}{first\PYGZus{}names}\PYG{p}{,} \PYG{l+s+s1}{\PYGZsq{}}\PYG{l+s+s1}{Sur\PYGZus{}name}\PYG{l+s+s1}{\PYGZsq{}}\PYG{p}{:} \PYG{n}{sur\PYGZus{}names}\PYG{p}{,} \PYG{l+s+s1}{\PYGZsq{}}\PYG{l+s+s1}{age}\PYG{l+s+s1}{\PYGZsq{}}\PYG{p}{:}\PYG{n}{ages}\PYG{p}{\PYGZcb{}}\PYG{p}{)}

\PYG{c+c1}{\PYGZsh{} create selector}
\PYG{n}{alex\PYGZus{}selector} \PYG{o}{=} \PYG{n}{ps}\PYG{o}{.}\PYG{n}{EqualitySelector}\PYG{p}{(}\PYG{l+s+s1}{\PYGZsq{}}\PYG{l+s+s1}{First\PYGZus{}name}\PYG{l+s+s1}{\PYGZsq{}}\PYG{p}{,} \PYG{l+s+s1}{\PYGZsq{}}\PYG{l+s+s1}{Alex}\PYG{l+s+s1}{\PYGZsq{}}\PYG{p}{)}
\PYG{n}{age\PYGZus{}selector} \PYG{o}{=} \PYG{n}{ps}\PYG{o}{.}\PYG{n}{EqualitySelector}\PYG{p}{(}\PYG{l+s+s1}{\PYGZsq{}}\PYG{l+s+s1}{age}\PYG{l+s+s1}{\PYGZsq{}}\PYG{p}{,} \PYG{l+m+mi}{22}\PYG{p}{)}
\PYG{c+c1}{\PYGZsh{} apply selectors to dataframe}
\PYG{n+nb}{print}\PYG{p}{(}\PYG{l+s+s1}{\PYGZsq{}}\PYG{l+s+s1}{instances with }\PYG{l+s+s1}{\PYGZsq{}}\PYG{p}{,} \PYG{n+nb}{str}\PYG{p}{(}\PYG{n}{alex\PYGZus{}selector}\PYG{p}{)}\PYG{p}{,} \PYG{n}{alex\PYGZus{}selector}\PYG{o}{.}\PYG{n}{covers}\PYG{p}{(}\PYG{n}{df}\PYG{p}{)}\PYG{p}{)}
\PYG{n+nb}{print}\PYG{p}{(}\PYG{l+s+s1}{\PYGZsq{}}\PYG{l+s+s1}{instances with}\PYG{l+s+s1}{\PYGZsq{}}\PYG{p}{,} \PYG{n+nb}{str}\PYG{p}{(}\PYG{n}{age\PYGZus{}selector}\PYG{p}{)}\PYG{p}{,} \PYG{n}{age\PYGZus{}selector}\PYG{o}{.}\PYG{n}{covers}\PYG{p}{(}\PYG{n}{df}\PYG{p}{)}\PYG{p}{)}
\end{sphinxVerbatim}

\begin{sphinxVerbatim}[commandchars=\\\{\}]
instances with  First\PYGZus{}name==\PYGZsq{}Alex\PYGZsq{} [ True False  True]
instances with age==22 [False False False]
\end{sphinxVerbatim}

The output indicates that the first and third instance in the dataset have a first name that is equal to \sphinxcode{\sphinxupquote{\textquotesingle{}Alex\textquotesingle{}}}.
The second output shows that no instances in our dataset is of age 22.
The EqualitySelector selector can be used on many different datatypes, but is most useful on binary, string and categorical data.
In addition to the EqualitySelector the pysubgroup package also provides the IntervalSelector. The following codes selects all instances from the database, which are in the age range 18 (included) to 40 (excluded).

\begin{sphinxVerbatim}[commandchars=\\\{\}]
\PYG{n}{interval\PYGZus{}selector} \PYG{o}{=} \PYG{n}{ps}\PYG{o}{.}\PYG{n}{IntervalSelector}\PYG{p}{(}\PYG{l+s+s1}{\PYGZsq{}}\PYG{l+s+s1}{age}\PYG{l+s+s1}{\PYGZsq{}}\PYG{p}{,} \PYG{l+m+mi}{18}\PYG{p}{,} \PYG{l+m+mi}{40}\PYG{p}{)}
\PYG{n+nb}{print}\PYG{p}{(}\PYG{n}{interval\PYGZus{}selector}\PYG{o}{.}\PYG{n}{covers}\PYG{p}{(}\PYG{n}{df}\PYG{p}{)}\PYG{p}{)}
\end{sphinxVerbatim}

\begin{sphinxVerbatim}[commandchars=\\\{\}]
[False  True  True]
\end{sphinxVerbatim}

The outpu shows that the second and third instance in our dataset have an age within the interval \([18,40)\).

Selectors are the building block of all rules generated with the pysubgroup package. If you want to write your own custom selector that is not a problem see \DUrole{xref,std,std-ref}{customselector} for references.


\section{Negations}
\label{\detokenize{Selectors:negations}}
The pysubgroup package also provides the NegatedSelector class that takes any selector (not just basic ones) and inverts it.

\begin{sphinxVerbatim}[commandchars=\\\{\}]
\PYG{n}{inverted\PYGZus{}selector} \PYG{o}{=} \PYG{n}{ps}\PYG{o}{.}\PYG{n}{NegatedSelector}\PYG{p}{(}\PYG{n}{alex\PYGZus{}selector}\PYG{p}{)}
\PYG{n+nb}{print}\PYG{p}{(}\PYG{l+s+s1}{\PYGZsq{}}\PYG{l+s+s1}{instances with first name not equal to Alex}\PYG{l+s+s1}{\PYGZsq{}}\PYG{p}{,} \PYG{n}{inverted\PYGZus{}selector}\PYG{o}{.}\PYG{n}{covers}\PYG{p}{(}\PYG{n}{df}\PYG{p}{)}\PYG{p}{)}
\end{sphinxVerbatim}

\begin{sphinxVerbatim}[commandchars=\\\{\}]
instances with first name not equal to Alex [False  True False]
\end{sphinxVerbatim}

The output is: \sphinxcode{\sphinxupquote{instances with first name not equal to Alex  {[}False, True, False{]}}}.


\section{Conjunctions}
\label{\detokenize{Selectors:conjunctions}}
Most of the rules that are generated with the pysubgroup package use conjunctions to form more complex queries. Continuing the running example from above we can find all persons whose name is Alex \sphinxstyleemphasis{and} which have an age in the interval \([18,40)\) like so:

\begin{sphinxVerbatim}[commandchars=\\\{\}]
\PYG{n}{conj} \PYG{o}{=} \PYG{n}{ps}\PYG{o}{.}\PYG{n}{Conjunction}\PYG{p}{(}\PYG{p}{[}\PYG{n}{interval\PYGZus{}selector}\PYG{p}{,} \PYG{n}{alex\PYGZus{}selector}\PYG{p}{]}\PYG{p}{)}
\PYG{n+nb}{print}\PYG{p}{(}\PYG{l+s+s1}{\PYGZsq{}}\PYG{l+s+s1}{instances with}\PYG{l+s+s1}{\PYGZsq{}}\PYG{p}{,} \PYG{n+nb}{str}\PYG{p}{(}\PYG{n}{conj}\PYG{p}{)}\PYG{p}{,} \PYG{n}{conj}\PYG{o}{.}\PYG{n}{covers}\PYG{p}{(}\PYG{n}{df}\PYG{p}{)}\PYG{p}{)}
\end{sphinxVerbatim}

\begin{sphinxVerbatim}[commandchars=\\\{\}]
instances with First\PYGZus{}name==\PYGZsq{}Alex\PYGZsq{} AND age: [18:40[ [False False  True]
\end{sphinxVerbatim}

The output shows that only the last instance is covered by our conjunction.


\section{Disjunctions}
\label{\detokenize{Selectors:disjunctions}}
The pysubgroup package also provides disjunctions with the \sphinxcode{\sphinxupquote{Disjunction}} class. Continuing the running example we can find all persons whose name is Alex \sphinxstyleemphasis{or} which have an age in the interval \([18,40)\) like so:

\begin{sphinxVerbatim}[commandchars=\\\{\}]
\PYG{n}{disj} \PYG{o}{=} \PYG{n}{ps}\PYG{o}{.}\PYG{n}{Disjunction}\PYG{p}{(}\PYG{p}{[}\PYG{n}{interval\PYGZus{}selector}\PYG{p}{,} \PYG{n}{alex\PYGZus{}selector}\PYG{p}{]}\PYG{p}{)}
\PYG{n+nb}{print}\PYG{p}{(}\PYG{l+s+s1}{\PYGZsq{}}\PYG{l+s+s1}{instances with}\PYG{l+s+s1}{\PYGZsq{}}\PYG{p}{,} \PYG{n+nb}{str}\PYG{p}{(}\PYG{n}{disj}\PYG{p}{)}\PYG{p}{,} \PYG{n}{disj}\PYG{o}{.}\PYG{n}{covers}\PYG{p}{(}\PYG{n}{df}\PYG{p}{)}\PYG{p}{)}
\end{sphinxVerbatim}

\begin{sphinxVerbatim}[commandchars=\\\{\}]
instances with First\PYGZus{}name==\PYGZsq{}Alex\PYGZsq{} OR age: [18:40[ [ True  True  True]
\end{sphinxVerbatim}

We can see that all instances are covered by our conjunction.


\section{Implementing your own}
\label{\detokenize{Selectors:implementing-your-own}}
As already mentioned in the introduction on selectors, anything that provides a cover function is a selector. In this case we will show how to implement a custom basic selector that checks whether a string contains a given substring:

\begin{sphinxVerbatim}[commandchars=\\\{\}]
\PYG{k}{class} \PYG{n+nc}{StrContainsSelector}\PYG{p}{:}
    \PYG{k}{def} \PYG{n+nf}{\PYGZus{}\PYGZus{}init\PYGZus{}\PYGZus{}}\PYG{p}{(}\PYG{n+nb+bp}{self}\PYG{p}{,} \PYG{n}{column}\PYG{p}{,} \PYG{n}{substr}\PYG{p}{)}\PYG{p}{:}
        \PYG{n+nb+bp}{self}\PYG{o}{.}\PYG{n}{column} \PYG{o}{=} \PYG{n}{column}
        \PYG{n+nb+bp}{self}\PYG{o}{.}\PYG{n}{substr} \PYG{o}{=} \PYG{n}{substr}

    \PYG{k}{def} \PYG{n+nf}{covers}\PYG{p}{(}\PYG{n+nb+bp}{self}\PYG{p}{,} \PYG{n}{df}\PYG{p}{)}\PYG{p}{:}
        \PYG{k}{return} \PYG{n}{df}\PYG{p}{[}\PYG{n+nb+bp}{self}\PYG{o}{.}\PYG{n}{column}\PYG{p}{]}\PYG{o}{.}\PYG{n}{str}\PYG{o}{.}\PYG{n}{contains}\PYG{p}{(}\PYG{n+nb+bp}{self}\PYG{o}{.}\PYG{n}{substr}\PYG{p}{)}\PYG{o}{.}\PYG{n}{to\PYGZus{}numpy}\PYG{p}{(}\PYG{p}{)}

\PYG{n}{contains\PYGZus{}selector} \PYG{o}{=} \PYG{n}{StrContainsSelector}\PYG{p}{(}\PYG{l+s+s1}{\PYGZsq{}}\PYG{l+s+s1}{Sur\PYGZus{}name}\PYG{l+s+s1}{\PYGZsq{}}\PYG{p}{,}\PYG{l+s+s1}{\PYGZsq{}}\PYG{l+s+s1}{m}\PYG{l+s+s1}{\PYGZsq{}}\PYG{p}{)}
\PYG{n+nb}{print}\PYG{p}{(}\PYG{n}{contains\PYGZus{}selector}\PYG{o}{.}\PYG{n}{covers}\PYG{p}{(}\PYG{n}{df}\PYG{p}{)}\PYG{p}{)}
\end{sphinxVerbatim}

\begin{sphinxVerbatim}[commandchars=\\\{\}]
[ True False  True]
\end{sphinxVerbatim}

The output shows that only the first and last instance contain an m in their name.
In addition to the covers function it is certainly advised to also implement the \sphinxcode{\sphinxupquote{\_\_str\_\_}} and \sphinxcode{\sphinxupquote{\_\_repr\_\_}} functions. This selector can now be added to the searchspace for any algorithm execution.


\chapter{Targets and Quality Functions}
\label{\detokenize{targets:targets-and-quality-functions}}\label{\detokenize{targets::doc}}
To define the goal of our subgroup discovery task, we use targets and quality functions. Targets are used to define which attributes play a significant role and can provide common statistics for a subgroup in question. Quality functions assign a score to each subgroup.
These scores are used by all the algorithms to determine the most interesting subgroups.


\section{Frequent Itemset Targets}
\label{\detokenize{targets:frequent-itemset-targets}}\label{\detokenize{targets:countqf}}
The most simple target is the \sphinxstyleemphasis{FITarget} with its associated quality functions \sphinxstyleemphasis{CountQF} and \sphinxstyleemphasis{AreaQf}.
The CountQF simple counts the number of instances covered by the subgroup in question.
The AreaQF multiplies the depth or length of the subgroup description with the number of instances covered by that description.


\section{Binary Targets}
\label{\detokenize{targets:binary-targets}}
For Boolean or Binary Targets we provide the \sphinxstyleemphasis{ChiSquaredQF} as well as the \sphinxstyleemphasis{StandardQF} quality functions.
The \sphinxstyleemphasis{StandardQF} quality function uses a parameter \(\alpha\) to weight the relative size \(\frac{N_{SG}}{N}\) of a subgroup and
multiplies it with the differences in relations of positive instances \(p\) to the number of instances \(N\)
\begin{equation*}
\begin{split}\left ( \frac{N_{SG}}{N} \right ) ^\alpha \left(\frac{p_{SG}}{N_{SG}} - \frac{p}{N} \right)\end{split}
\end{equation*}
The \sphinxstyleemphasis{StandardQF} also supports an optimistic estimate.

The \sphinxstyleemphasis{ChiSquaredQF} is calculated based on the following contigency table which is then passed to the scipy \sphinxhref{https://docs.scipy.org/doc/scipy/reference/generated/scipy.stats.chi2\_contingency.html}{chi2\_contigency} function.
The small \(n\) represents the number of negative instances and should not be confused with the capital \(N\) which represents the total number of instances.


\begin{savenotes}\sphinxattablestart
\centering
\begin{tabulary}{\linewidth}[t]{|T|T|}
\hline

\(p_{SG}\)
&
\(p-p_{SG}\)
\\
\hline
\(n_{SG}\)
&
\(n-n_{SG}\)
\\
\hline
\end{tabulary}
\par
\sphinxattableend\end{savenotes}


\section{Nominal Targets}
\label{\detokenize{targets:nominal-targets}}
Currently pysubgroup only supports nominal targets as binary targets.
So you can look for deviations of one nominal value with respect to all othe nominal values.


\section{Numeric Targets}
\label{\detokenize{targets:numeric-targets}}
For numeric targets pysubgroup offers the \sphinxstyleemphasis{StandardQFNumeric} which is defined similar to the StandardQF
\begin{equation*}
\begin{split}\left ( \frac{N_{SG}}{N}  \right ) ^\alpha \left (\mu_{SG} - \mu \right )\end{split}
\end{equation*}
where \(\mu_{SG}\) and \(\mu\) are the mean value for the subgroup and entire dataset respectively.
For the \sphinxstyleemphasis{StandardQFNumeric} we offer three optimistic estimates:  Average, Summation and Ordering. These are in detail described in Florian Lemmerich’s dissertation.
You can choose between the different optimistic estimates by using the keyword argument \sphinxcode{\sphinxupquote{estimator}} the different options are \sphinxcode{\sphinxupquote{\textquotesingle{}sum\textquotesingle{}}}, \sphinxcode{\sphinxupquote{\textquotesingle{}average\textquotesingle{}}}, and \sphinxcode{\sphinxupquote{\textquotesingle{}order\textquotesingle{}}}


\section{Custom Quality Function}
\label{\detokenize{targets:custom-quality-function}}\label{\detokenize{targets:customtarget}}
To create a custom quality function that works will all algorithms except gp\_growth.

\begin{sphinxVerbatim}[commandchars=\\\{\}]
\PYG{k}{class} \PYG{n+nc}{MyQualityFunction}\PYG{p}{:}
    \PYG{k}{def} \PYG{n+nf}{calculate\PYGZus{}constant\PYGZus{}statistics}\PYG{p}{(}\PYG{n+nb+bp}{self}\PYG{p}{,} \PYG{n}{task}\PYG{p}{)}\PYG{p}{:}
        \PYG{l+s+sd}{\PYGZdq{}\PYGZdq{}\PYGZdq{} calculate\PYGZus{}constant\PYGZus{}statistics}
\PYG{l+s+sd}{            This function is called once for every execution,}
\PYG{l+s+sd}{            it should do any preparation that is necessary prior to an execution.}
\PYG{l+s+sd}{        \PYGZdq{}\PYGZdq{}\PYGZdq{}}
        \PYG{k}{pass}

    \PYG{k}{def} \PYG{n+nf}{calculate\PYGZus{}statistics}\PYG{p}{(}\PYG{n+nb+bp}{self}\PYG{p}{,} \PYG{n}{subgroup}\PYG{p}{,} \PYG{n}{data}\PYG{o}{=}\PYG{n+nb+bp}{None}\PYG{p}{)}\PYG{p}{:}
        \PYG{l+s+sd}{\PYGZdq{}\PYGZdq{}\PYGZdq{} calculates necessary statistics}
\PYG{l+s+sd}{            this statistics object is passed on to the evaluate}
\PYG{l+s+sd}{            and optimistic\PYGZus{}estimate functions}
\PYG{l+s+sd}{        \PYGZdq{}\PYGZdq{}\PYGZdq{}}
        \PYG{k}{pass}

    \PYG{k}{def} \PYG{n+nf}{evaluate}\PYG{p}{(}\PYG{n+nb+bp}{self}\PYG{p}{,} \PYG{n}{subgroup}\PYG{p}{,} \PYG{n}{statistics\PYGZus{}or\PYGZus{}data}\PYG{o}{=}\PYG{n+nb+bp}{None}\PYG{p}{)}\PYG{p}{:}
        \PYG{l+s+sd}{\PYGZdq{}\PYGZdq{}\PYGZdq{} return the quality calculated from the statistics \PYGZdq{}\PYGZdq{}\PYGZdq{}}
        \PYG{k}{pass}

    \PYG{k}{def} \PYG{n+nf}{optimistic\PYGZus{}estimate}\PYG{p}{(}\PYG{n+nb+bp}{self}\PYG{p}{,} \PYG{n}{subgroup}\PYG{p}{,} \PYG{n}{statistics}\PYG{o}{=}\PYG{n+nb+bp}{None}\PYG{p}{)}\PYG{p}{:}
        \PYG{l+s+sd}{\PYGZdq{}\PYGZdq{}\PYGZdq{} returns optimistic estimate}
\PYG{l+s+sd}{            if one is available return it otherwise infinity\PYGZdq{}\PYGZdq{}\PYGZdq{}}
        \PYG{k}{pass}
\end{sphinxVerbatim}


\chapter{GP\sphinxhyphen{}Growth}
\label{\detokenize{gp_growth:gp-growth}}\label{\detokenize{gp_growth::doc}}
This tree based algorithm uses a condensed representation (a so called valuation basis) to find interesting subgroups. The main advantage of this approach is, that the (potentially large) database has to be scanned only twice and thereafter all the necessary information is represented as more compact pattern\sphinxhyphen{}tree.
Gp\sphinxhyphen{}growth is a generalisation of the popular \sphinxhref{https://en.wikipedia.org/wiki/Association\_rule\_learning\#FP-growth\_algorithm}{fp\sphinxhyphen{}growth} algorithm. So refer to instructional material on fp\sphinxhyphen{}growth for more in depth knowledge on the workings of this tree based algorithm.

\begin{sphinxShadowBox}
\sphinxstyletopictitle{Contents}
\begin{itemize}
\item {} 
\phantomsection\label{\detokenize{gp_growth:id1}}{\hyperref[\detokenize{gp_growth:gp-growth}]{\sphinxcrossref{GP\sphinxhyphen{}Growth}}}
\begin{itemize}
\item {} 
\phantomsection\label{\detokenize{gp_growth:id2}}{\hyperref[\detokenize{gp_growth:basic-usage}]{\sphinxcrossref{Basic usage}}}

\item {} 
\phantomsection\label{\detokenize{gp_growth:id3}}{\hyperref[\detokenize{gp_growth:create-a-custom-target}]{\sphinxcrossref{Create a custom target}}}

\end{itemize}

\end{itemize}
\end{sphinxShadowBox}


\section{Basic usage}
\label{\detokenize{gp_growth:basic-usage}}
The basic usage of the gp\sphinxhyphen{}growth algorithm is not very different from the usage of any other algorithm in this package.

\begin{sphinxVerbatim}[commandchars=\\\{\}]
\PYG{k+kn}{import} \PYG{n+nn}{pysubgroup} \PYG{k+kn}{as} \PYG{n+nn}{ps}

\PYG{c+c1}{\PYGZsh{} Load the example dataset}
\PYG{k+kn}{from} \PYG{n+nn}{pysubgroup.tests.DataSets} \PYG{k+kn}{import} \PYG{n}{get\PYGZus{}titanic\PYGZus{}data}
\PYG{n}{data} \PYG{o}{=} \PYG{n}{get\PYGZus{}titanic\PYGZus{}data}\PYG{p}{(}\PYG{p}{)}

\PYG{n}{target} \PYG{o}{=} \PYG{n}{ps}\PYG{o}{.}\PYG{n}{NominalSelector} \PYG{p}{(}\PYG{l+s+s1}{\PYGZsq{}}\PYG{l+s+s1}{Survived}\PYG{l+s+s1}{\PYGZsq{}}\PYG{p}{,} \PYG{n+nb+bp}{True}\PYG{p}{)}
\PYG{n}{searchspace} \PYG{o}{=} \PYG{n}{ps}\PYG{o}{.}\PYG{n}{create\PYGZus{}selectors}\PYG{p}{(}\PYG{n}{data}\PYG{p}{,} \PYG{n}{ignore}\PYG{o}{=}\PYG{p}{[}\PYG{l+s+s1}{\PYGZsq{}}\PYG{l+s+s1}{Survived}\PYG{l+s+s1}{\PYGZsq{}}\PYG{p}{]}\PYG{p}{)}
\PYG{n}{task} \PYG{o}{=} \PYG{n}{ps}\PYG{o}{.}\PYG{n}{SubgroupDiscoveryTask} \PYG{p}{(}\PYG{n}{data}\PYG{p}{,} \PYG{n}{target}\PYG{p}{,} \PYG{n}{dearchspace}\PYG{p}{,} \PYG{n}{result\PYGZus{}set\PYGZus{}size}\PYG{o}{=}\PYG{l+m+mi}{5}\PYG{p}{,} \PYG{n}{depth}\PYG{o}{=}\PYG{l+m+mi}{2}\PYG{p}{,} \PYG{n}{qf}\PYG{o}{=}\PYG{n}{ps}\PYG{o}{.}\PYG{n}{WRAccQF}\PYG{p}{(}\PYG{p}{)}\PYG{p}{)}
\PYG{n}{GpGrowth}\PYG{o}{.}\PYG{n}{execute}\PYG{p}{(}\PYG{n}{task}\PYG{p}{)}
\end{sphinxVerbatim}

But beware that gp\sphinxhyphen{}growth is using an exhaustive search strategy! This can greatly increase the runtime for high search depth.
You can specify the \sphinxcode{\sphinxupquote{mode}} argument in the constructor of GpGrowth to run gp\sphinxhyphen{}growth either bottom up (\sphinxcode{\sphinxupquote{mode=\textquotesingle{}b\_u\textquotesingle{}}}) or top down (\sphinxcode{\sphinxupquote{mode=\textquotesingle{}b\_u\textquotesingle{}}}).
As gp growth is a generalisation of fp\sphinxhyphen{}growth you can also perform standard fp\sphinxhyphen{}growth using gp\_growth by using the CountQF ({\hyperref[\detokenize{targets:countqf}]{\sphinxcrossref{\DUrole{std,std-ref}{Frequent Itemset Targets}}}}) quality function.


\section{Create a custom target}
\label{\detokenize{gp_growth:create-a-custom-target}}
If you consider to use the gp\sphinxhyphen{}growth algorithm for your custom target that is totally possible if you find a valuation basis.
We will now first introduce the concept of a valuation basis and thereafter outline the gp\sphinxhyphen{}growth interface that you have to support to use your quality function with our gp\sphinxhyphen{}growth implementation.


\subsection{Valuation Basis}
\label{\detokenize{gp_growth:valuation-basis}}
Think of a valuation basis as a codensed representation of a subgroup that allows to quickly compute the same representation for a union of two disjoint subgroups.

We call the function which takes the valuation basis of two disjoint sets and computes the valuation basis for the unified set \sphinxcode{\sphinxupquote{merge}}. The function that compute the necessary statistics from a valuation basis \sphinxcode{\sphinxupquote{stats\_from\_basis}}.

Now we can formulate: Given two disjoint sets \(A\) and \(B\) with \(A \cap B = \varnothing\) and their valuation bases \(v(A)\) and \(v(B)\) with their functions \sphinxcode{\sphinxupquote{stats\_from\_basis}} and \sphinxcode{\sphinxupquote{merge}} as defined above, we can compute the properties of \(A \cup B\) instead of from the union of the instances from the merged valuation basis.
This can be summarized through the following equation:
\begin{equation*}
\begin{split}props\_from\_instances(A\cup B) = props\_from\_basis(merge(v(A), v(B)))\end{split}
\end{equation*}

\subsection{Required Methods}
\label{\detokenize{gp_growth:required-methods}}
To make a target and quality function suitable for gp\sphinxhyphen{}growth you have to provide several methods (all methods start with \sphinxcode{\sphinxupquote{gp\_}} to indicate that they are used in the gp\sphinxhyphen{}growth algorithm). In addition to the standard quality function methods (see {\hyperref[\detokenize{targets:customtarget}]{\sphinxcrossref{\DUrole{std,std-ref}{Custom Quality Function}}}}) the following methods should be implemented to make a quality funciton usable with gp\_growth.

\begin{sphinxVerbatim}[commandchars=\\\{\}]
\PYG{k}{class} \PYG{n+nc}{MyGpQualityFunction}
    \PYG{k}{def} \PYG{n+nf}{gp\PYGZus{}get\PYGZus{}basis}\PYG{p}{(}\PYG{n+nb+bp}{self}\PYG{p}{,} \PYG{n}{row\PYGZus{}index}\PYG{p}{)}\PYG{p}{:}
    \PYG{l+s+sd}{\PYGZdq{}\PYGZdq{}\PYGZdq{} returns the valuation basis of the element at this row\PYGZus{}index \PYGZdq{}\PYGZdq{}\PYGZdq{}}
        \PYG{k}{pass}

    \PYG{k}{def} \PYG{n+nf}{gp\PYGZus{}get\PYGZus{}null\PYGZus{}vector}\PYG{p}{(}\PYG{n+nb+bp}{self}\PYG{p}{)}\PYG{p}{:}
    \PYG{l+s+sd}{\PYGZdq{}\PYGZdq{}\PYGZdq{} returns the zero element of the valuation basis \PYGZdq{}\PYGZdq{}\PYGZdq{}}
        \PYG{k}{pass}

    \PYG{n+nd}{@staticmethod}
    \PYG{k}{def} \PYG{n+nf}{gp\PYGZus{}merge}\PYG{p}{(}\PYG{n}{v\PYGZus{}l}\PYG{p}{,} \PYG{n}{v\PYGZus{}r}\PYG{p}{)}\PYG{p}{:}
    \PYG{l+s+sd}{\PYGZdq{}\PYGZdq{}\PYGZdq{} merges the v\PYGZus{}r valuation basis into the v\PYGZus{}l valuation basis inplace! \PYGZdq{}\PYGZdq{}\PYGZdq{}}
        \PYG{k}{pass}

    \PYG{k}{def} \PYG{n+nf}{gp\PYGZus{}get\PYGZus{}statistics}\PYG{p}{(}\PYG{n+nb+bp}{self}\PYG{p}{,} \PYG{n}{cover\PYGZus{}arr}\PYG{p}{,} \PYG{n}{v}\PYG{p}{)}\PYG{p}{:}
    \PYG{l+s+sd}{\PYGZdq{}\PYGZdq{}\PYGZdq{} computes the statistics for this quality function from the valuation basis v \PYGZdq{}\PYGZdq{}\PYGZdq{}}
        \PYG{k}{pass}

    \PYG{n+nd}{@property}
    \PYG{k}{def} \PYG{n+nf}{gp\PYGZus{}requires\PYGZus{}cover\PYGZus{}arr}\PYG{p}{(}\PYG{n+nb+bp}{self}\PYG{p}{)} \PYG{o}{\PYGZhy{}}\PYG{o}{\PYGZgt{}} \PYG{n+nb}{bool}\PYG{p}{:}
    \PYG{l+s+sd}{\PYGZdq{}\PYGZdq{}\PYGZdq{} returns a boolean value that indicates whether a cover array is required when calling the gp\PYGZus{}get\PYGZus{}statistics function}

\PYG{l+s+sd}{        usually this value is False}
\PYG{l+s+sd}{    \PYGZdq{}\PYGZdq{}\PYGZdq{}}
        \PYG{k}{pass}
\end{sphinxVerbatim}


\subsection{Saving a gp\_tree}
\label{\detokenize{gp_growth:saving-a-gp-tree}}
It is possible to save a gp tree to a txt file for e.g. debugging purpose. You therefor have to implementd the gp\_to\_str function which takes a valuation basis and returns a string representation.
It is an intentional choide to not call the  \sphinxcode{\sphinxupquote{str}} function on the valuation basis directly.

\begin{sphinxVerbatim}[commandchars=\\\{\}]
\PYG{k}{def} \PYG{n+nf}{gp\PYGZus{}to\PYGZus{}str}\PYG{p}{(}\PYG{n+nb+bp}{self}\PYG{p}{,} \PYG{n}{basis}\PYG{p}{)} \PYG{o}{\PYGZhy{}}\PYG{o}{\PYGZgt{}} \PYG{n+nb}{str}\PYG{p}{:}
\PYG{l+s+sd}{\PYGZdq{}\PYGZdq{}\PYGZdq{} returns a string representation of the valuation basis \PYGZdq{}\PYGZdq{}\PYGZdq{}}
    \PYG{k}{pass}
\end{sphinxVerbatim}


\chapter{pysubgroup}
\label{\detokenize{source/modules:pysubgroup}}\label{\detokenize{source/modules::doc}}

\section{pysubgroup package}
\label{\detokenize{source/pysubgroup:pysubgroup-package}}\label{\detokenize{source/pysubgroup::doc}}

\subsection{Submodules}
\label{\detokenize{source/pysubgroup:submodules}}

\subsection{pysubgroup.algorithms module}
\label{\detokenize{source/pysubgroup:module-pysubgroup.algorithms}}\label{\detokenize{source/pysubgroup:pysubgroup-algorithms-module}}\index{pysubgroup.algorithms (module)@\spxentry{pysubgroup.algorithms}\spxextra{module}}
Created on 29.04.2016

@author: lemmerfn
\index{Apriori (class in pysubgroup.algorithms)@\spxentry{Apriori}\spxextra{class in pysubgroup.algorithms}}

\begin{fulllineitems}
\phantomsection\label{\detokenize{source/pysubgroup:pysubgroup.algorithms.Apriori}}\pysiglinewithargsret{\sphinxbfcode{\sphinxupquote{class }}\sphinxbfcode{\sphinxupquote{Apriori}}}{\emph{representation\_type=None}, \emph{combination\_name=\textquotesingle{}Conjunction\textquotesingle{}}, \emph{use\_numba=True}}{}~\index{execute() (Apriori method)@\spxentry{execute()}\spxextra{Apriori method}}

\begin{fulllineitems}
\phantomsection\label{\detokenize{source/pysubgroup:pysubgroup.algorithms.Apriori.execute}}\pysiglinewithargsret{\sphinxbfcode{\sphinxupquote{execute}}}{\emph{task}}{}
\end{fulllineitems}

\index{get\_next\_level() (Apriori method)@\spxentry{get\_next\_level()}\spxextra{Apriori method}}

\begin{fulllineitems}
\phantomsection\label{\detokenize{source/pysubgroup:pysubgroup.algorithms.Apriori.get_next_level}}\pysiglinewithargsret{\sphinxbfcode{\sphinxupquote{get\_next\_level}}}{\emph{promising\_candidates}}{}
\end{fulllineitems}

\index{get\_next\_level\_candidates() (Apriori method)@\spxentry{get\_next\_level\_candidates()}\spxextra{Apriori method}}

\begin{fulllineitems}
\phantomsection\label{\detokenize{source/pysubgroup:pysubgroup.algorithms.Apriori.get_next_level_candidates}}\pysiglinewithargsret{\sphinxbfcode{\sphinxupquote{get\_next\_level\_candidates}}}{\emph{task}, \emph{result}, \emph{next\_level\_candidates}}{}
\end{fulllineitems}

\index{get\_next\_level\_candidates\_vectorized() (Apriori method)@\spxentry{get\_next\_level\_candidates\_vectorized()}\spxextra{Apriori method}}

\begin{fulllineitems}
\phantomsection\label{\detokenize{source/pysubgroup:pysubgroup.algorithms.Apriori.get_next_level_candidates_vectorized}}\pysiglinewithargsret{\sphinxbfcode{\sphinxupquote{get\_next\_level\_candidates\_vectorized}}}{\emph{task}, \emph{result}, \emph{next\_level\_candidates}}{}
\end{fulllineitems}

\index{get\_next\_level\_numba() (Apriori method)@\spxentry{get\_next\_level\_numba()}\spxextra{Apriori method}}

\begin{fulllineitems}
\phantomsection\label{\detokenize{source/pysubgroup:pysubgroup.algorithms.Apriori.get_next_level_numba}}\pysiglinewithargsret{\sphinxbfcode{\sphinxupquote{get\_next\_level\_numba}}}{\emph{promising\_candidates}}{}
\end{fulllineitems}

\index{reprune\_lower\_levels() (Apriori method)@\spxentry{reprune\_lower\_levels()}\spxextra{Apriori method}}

\begin{fulllineitems}
\phantomsection\label{\detokenize{source/pysubgroup:pysubgroup.algorithms.Apriori.reprune_lower_levels}}\pysiglinewithargsret{\sphinxbfcode{\sphinxupquote{reprune\_lower\_levels}}}{\emph{promising\_candidates}, \emph{depth}}{}
\end{fulllineitems}


\end{fulllineitems}

\index{BeamSearch (class in pysubgroup.algorithms)@\spxentry{BeamSearch}\spxextra{class in pysubgroup.algorithms}}

\begin{fulllineitems}
\phantomsection\label{\detokenize{source/pysubgroup:pysubgroup.algorithms.BeamSearch}}\pysiglinewithargsret{\sphinxbfcode{\sphinxupquote{class }}\sphinxbfcode{\sphinxupquote{BeamSearch}}}{\emph{beam\_width=20}, \emph{beam\_width\_adaptive=False}}{}
Implements the BeamSearch algorithm. Its a basic implementation
\index{execute() (BeamSearch method)@\spxentry{execute()}\spxextra{BeamSearch method}}

\begin{fulllineitems}
\phantomsection\label{\detokenize{source/pysubgroup:pysubgroup.algorithms.BeamSearch.execute}}\pysiglinewithargsret{\sphinxbfcode{\sphinxupquote{execute}}}{\emph{task}}{}
\end{fulllineitems}


\end{fulllineitems}

\index{BestFirstSearch (class in pysubgroup.algorithms)@\spxentry{BestFirstSearch}\spxextra{class in pysubgroup.algorithms}}

\begin{fulllineitems}
\phantomsection\label{\detokenize{source/pysubgroup:pysubgroup.algorithms.BestFirstSearch}}\pysigline{\sphinxbfcode{\sphinxupquote{class }}\sphinxbfcode{\sphinxupquote{BestFirstSearch}}}~\index{execute() (BestFirstSearch method)@\spxentry{execute()}\spxextra{BestFirstSearch method}}

\begin{fulllineitems}
\phantomsection\label{\detokenize{source/pysubgroup:pysubgroup.algorithms.BestFirstSearch.execute}}\pysiglinewithargsret{\sphinxbfcode{\sphinxupquote{execute}}}{\emph{task}}{}
\end{fulllineitems}


\end{fulllineitems}

\index{DFS (class in pysubgroup.algorithms)@\spxentry{DFS}\spxextra{class in pysubgroup.algorithms}}

\begin{fulllineitems}
\phantomsection\label{\detokenize{source/pysubgroup:pysubgroup.algorithms.DFS}}\pysiglinewithargsret{\sphinxbfcode{\sphinxupquote{class }}\sphinxbfcode{\sphinxupquote{DFS}}}{\emph{apply\_representation}}{}
Implementation of a depth\sphinxhyphen{}first\sphinxhyphen{}search with look\sphinxhyphen{}ahead using a provided datastructure.
\index{execute() (DFS method)@\spxentry{execute()}\spxextra{DFS method}}

\begin{fulllineitems}
\phantomsection\label{\detokenize{source/pysubgroup:pysubgroup.algorithms.DFS.execute}}\pysiglinewithargsret{\sphinxbfcode{\sphinxupquote{execute}}}{\emph{task}}{}
\end{fulllineitems}

\index{search\_internal() (DFS method)@\spxentry{search\_internal()}\spxextra{DFS method}}

\begin{fulllineitems}
\phantomsection\label{\detokenize{source/pysubgroup:pysubgroup.algorithms.DFS.search_internal}}\pysiglinewithargsret{\sphinxbfcode{\sphinxupquote{search\_internal}}}{\emph{task}, \emph{result}, \emph{sg}}{}
\end{fulllineitems}


\end{fulllineitems}

\index{DFSNumeric (class in pysubgroup.algorithms)@\spxentry{DFSNumeric}\spxextra{class in pysubgroup.algorithms}}

\begin{fulllineitems}
\phantomsection\label{\detokenize{source/pysubgroup:pysubgroup.algorithms.DFSNumeric}}\pysigline{\sphinxbfcode{\sphinxupquote{class }}\sphinxbfcode{\sphinxupquote{DFSNumeric}}}~\index{execute() (DFSNumeric method)@\spxentry{execute()}\spxextra{DFSNumeric method}}

\begin{fulllineitems}
\phantomsection\label{\detokenize{source/pysubgroup:pysubgroup.algorithms.DFSNumeric.execute}}\pysiglinewithargsret{\sphinxbfcode{\sphinxupquote{execute}}}{\emph{task}}{}
\end{fulllineitems}

\index{search\_internal() (DFSNumeric method)@\spxentry{search\_internal()}\spxextra{DFSNumeric method}}

\begin{fulllineitems}
\phantomsection\label{\detokenize{source/pysubgroup:pysubgroup.algorithms.DFSNumeric.search_internal}}\pysiglinewithargsret{\sphinxbfcode{\sphinxupquote{search\_internal}}}{\emph{task}, \emph{prefix}, \emph{modification\_set}, \emph{result}, \emph{bitset}}{}
\end{fulllineitems}

\index{tpl (DFSNumeric attribute)@\spxentry{tpl}\spxextra{DFSNumeric attribute}}

\begin{fulllineitems}
\phantomsection\label{\detokenize{source/pysubgroup:pysubgroup.algorithms.DFSNumeric.tpl}}\pysigline{\sphinxbfcode{\sphinxupquote{tpl}}}
alias of \sphinxcode{\sphinxupquote{size\_mean\_parameters}}

\end{fulllineitems}


\end{fulllineitems}

\index{GeneralisingBFS (class in pysubgroup.algorithms)@\spxentry{GeneralisingBFS}\spxextra{class in pysubgroup.algorithms}}

\begin{fulllineitems}
\phantomsection\label{\detokenize{source/pysubgroup:pysubgroup.algorithms.GeneralisingBFS}}\pysigline{\sphinxbfcode{\sphinxupquote{class }}\sphinxbfcode{\sphinxupquote{GeneralisingBFS}}}~\index{execute() (GeneralisingBFS method)@\spxentry{execute()}\spxextra{GeneralisingBFS method}}

\begin{fulllineitems}
\phantomsection\label{\detokenize{source/pysubgroup:pysubgroup.algorithms.GeneralisingBFS.execute}}\pysiglinewithargsret{\sphinxbfcode{\sphinxupquote{execute}}}{\emph{task}}{}
\end{fulllineitems}


\end{fulllineitems}

\index{SimpleDFS (class in pysubgroup.algorithms)@\spxentry{SimpleDFS}\spxextra{class in pysubgroup.algorithms}}

\begin{fulllineitems}
\phantomsection\label{\detokenize{source/pysubgroup:pysubgroup.algorithms.SimpleDFS}}\pysigline{\sphinxbfcode{\sphinxupquote{class }}\sphinxbfcode{\sphinxupquote{SimpleDFS}}}~\index{execute() (SimpleDFS method)@\spxentry{execute()}\spxextra{SimpleDFS method}}

\begin{fulllineitems}
\phantomsection\label{\detokenize{source/pysubgroup:pysubgroup.algorithms.SimpleDFS.execute}}\pysiglinewithargsret{\sphinxbfcode{\sphinxupquote{execute}}}{\emph{task}, \emph{use\_optimistic\_estimates=True}}{}
\end{fulllineitems}

\index{search\_internal() (SimpleDFS method)@\spxentry{search\_internal()}\spxextra{SimpleDFS method}}

\begin{fulllineitems}
\phantomsection\label{\detokenize{source/pysubgroup:pysubgroup.algorithms.SimpleDFS.search_internal}}\pysiglinewithargsret{\sphinxbfcode{\sphinxupquote{search\_internal}}}{\emph{task}, \emph{prefix}, \emph{modification\_set}, \emph{result}, \emph{use\_optimistic\_estimates}}{}
\end{fulllineitems}


\end{fulllineitems}

\index{SimpleSearch (class in pysubgroup.algorithms)@\spxentry{SimpleSearch}\spxextra{class in pysubgroup.algorithms}}

\begin{fulllineitems}
\phantomsection\label{\detokenize{source/pysubgroup:pysubgroup.algorithms.SimpleSearch}}\pysiglinewithargsret{\sphinxbfcode{\sphinxupquote{class }}\sphinxbfcode{\sphinxupquote{SimpleSearch}}}{\emph{show\_progress=True}}{}~\index{execute() (SimpleSearch method)@\spxentry{execute()}\spxextra{SimpleSearch method}}

\begin{fulllineitems}
\phantomsection\label{\detokenize{source/pysubgroup:pysubgroup.algorithms.SimpleSearch.execute}}\pysiglinewithargsret{\sphinxbfcode{\sphinxupquote{execute}}}{\emph{task}}{}
\end{fulllineitems}


\end{fulllineitems}

\index{SubgroupDiscoveryTask (class in pysubgroup.algorithms)@\spxentry{SubgroupDiscoveryTask}\spxextra{class in pysubgroup.algorithms}}

\begin{fulllineitems}
\phantomsection\label{\detokenize{source/pysubgroup:pysubgroup.algorithms.SubgroupDiscoveryTask}}\pysiglinewithargsret{\sphinxbfcode{\sphinxupquote{class }}\sphinxbfcode{\sphinxupquote{SubgroupDiscoveryTask}}}{\emph{data}, \emph{target}, \emph{search\_space}, \emph{qf}, \emph{result\_set\_size=10}, \emph{depth=3}, \emph{min\_quality=0}, \emph{weighting\_attribute=None}}{}
Capsulates all parameters required to perform standard subgroup discovery

\end{fulllineitems}



\subsection{pysubgroup.boolean\_expressions module}
\label{\detokenize{source/pysubgroup:module-pysubgroup.boolean_expressions}}\label{\detokenize{source/pysubgroup:pysubgroup-boolean-expressions-module}}\index{pysubgroup.boolean\_expressions (module)@\spxentry{pysubgroup.boolean\_expressions}\spxextra{module}}\index{BooleanExpressionBase (class in pysubgroup.boolean\_expressions)@\spxentry{BooleanExpressionBase}\spxextra{class in pysubgroup.boolean\_expressions}}

\begin{fulllineitems}
\phantomsection\label{\detokenize{source/pysubgroup:pysubgroup.boolean_expressions.BooleanExpressionBase}}\pysigline{\sphinxbfcode{\sphinxupquote{class }}\sphinxbfcode{\sphinxupquote{BooleanExpressionBase}}}~\index{append\_and() (BooleanExpressionBase method)@\spxentry{append\_and()}\spxextra{BooleanExpressionBase method}}

\begin{fulllineitems}
\phantomsection\label{\detokenize{source/pysubgroup:pysubgroup.boolean_expressions.BooleanExpressionBase.append_and}}\pysiglinewithargsret{\sphinxbfcode{\sphinxupquote{abstract }}\sphinxbfcode{\sphinxupquote{append\_and}}}{\emph{to\_append}}{}
\end{fulllineitems}

\index{append\_or() (BooleanExpressionBase method)@\spxentry{append\_or()}\spxextra{BooleanExpressionBase method}}

\begin{fulllineitems}
\phantomsection\label{\detokenize{source/pysubgroup:pysubgroup.boolean_expressions.BooleanExpressionBase.append_or}}\pysiglinewithargsret{\sphinxbfcode{\sphinxupquote{abstract }}\sphinxbfcode{\sphinxupquote{append\_or}}}{\emph{to\_append}}{}
\end{fulllineitems}


\end{fulllineitems}

\index{Conjunction (class in pysubgroup.boolean\_expressions)@\spxentry{Conjunction}\spxextra{class in pysubgroup.boolean\_expressions}}

\begin{fulllineitems}
\phantomsection\label{\detokenize{source/pysubgroup:pysubgroup.boolean_expressions.Conjunction}}\pysiglinewithargsret{\sphinxbfcode{\sphinxupquote{class }}\sphinxbfcode{\sphinxupquote{Conjunction}}}{\emph{selectors}}{}~\index{append\_and() (Conjunction method)@\spxentry{append\_and()}\spxextra{Conjunction method}}

\begin{fulllineitems}
\phantomsection\label{\detokenize{source/pysubgroup:pysubgroup.boolean_expressions.Conjunction.append_and}}\pysiglinewithargsret{\sphinxbfcode{\sphinxupquote{append\_and}}}{\emph{to\_append}}{}
\end{fulllineitems}

\index{append\_or() (Conjunction method)@\spxentry{append\_or()}\spxextra{Conjunction method}}

\begin{fulllineitems}
\phantomsection\label{\detokenize{source/pysubgroup:pysubgroup.boolean_expressions.Conjunction.append_or}}\pysiglinewithargsret{\sphinxbfcode{\sphinxupquote{append\_or}}}{\emph{to\_append}}{}
\end{fulllineitems}

\index{covers() (Conjunction method)@\spxentry{covers()}\spxextra{Conjunction method}}

\begin{fulllineitems}
\phantomsection\label{\detokenize{source/pysubgroup:pysubgroup.boolean_expressions.Conjunction.covers}}\pysiglinewithargsret{\sphinxbfcode{\sphinxupquote{covers}}}{\emph{instance}}{}
\end{fulllineitems}

\index{depth() (Conjunction property)@\spxentry{depth()}\spxextra{Conjunction property}}

\begin{fulllineitems}
\phantomsection\label{\detokenize{source/pysubgroup:pysubgroup.boolean_expressions.Conjunction.depth}}\pysigline{\sphinxbfcode{\sphinxupquote{property }}\sphinxbfcode{\sphinxupquote{depth}}}
\end{fulllineitems}

\index{pop\_and() (Conjunction method)@\spxentry{pop\_and()}\spxextra{Conjunction method}}

\begin{fulllineitems}
\phantomsection\label{\detokenize{source/pysubgroup:pysubgroup.boolean_expressions.Conjunction.pop_and}}\pysiglinewithargsret{\sphinxbfcode{\sphinxupquote{pop\_and}}}{}{}
\end{fulllineitems}

\index{pop\_or() (Conjunction method)@\spxentry{pop\_or()}\spxextra{Conjunction method}}

\begin{fulllineitems}
\phantomsection\label{\detokenize{source/pysubgroup:pysubgroup.boolean_expressions.Conjunction.pop_or}}\pysiglinewithargsret{\sphinxbfcode{\sphinxupquote{pop\_or}}}{}{}
\end{fulllineitems}


\end{fulllineitems}

\index{DNF (class in pysubgroup.boolean\_expressions)@\spxentry{DNF}\spxextra{class in pysubgroup.boolean\_expressions}}

\begin{fulllineitems}
\phantomsection\label{\detokenize{source/pysubgroup:pysubgroup.boolean_expressions.DNF}}\pysiglinewithargsret{\sphinxbfcode{\sphinxupquote{class }}\sphinxbfcode{\sphinxupquote{DNF}}}{\emph{selectors=None}}{}~\index{append\_and() (DNF method)@\spxentry{append\_and()}\spxextra{DNF method}}

\begin{fulllineitems}
\phantomsection\label{\detokenize{source/pysubgroup:pysubgroup.boolean_expressions.DNF.append_and}}\pysiglinewithargsret{\sphinxbfcode{\sphinxupquote{append\_and}}}{\emph{to\_append}}{}
\end{fulllineitems}

\index{append\_or() (DNF method)@\spxentry{append\_or()}\spxextra{DNF method}}

\begin{fulllineitems}
\phantomsection\label{\detokenize{source/pysubgroup:pysubgroup.boolean_expressions.DNF.append_or}}\pysiglinewithargsret{\sphinxbfcode{\sphinxupquote{append\_or}}}{\emph{to\_append}}{}
\end{fulllineitems}

\index{pop\_and() (DNF method)@\spxentry{pop\_and()}\spxextra{DNF method}}

\begin{fulllineitems}
\phantomsection\label{\detokenize{source/pysubgroup:pysubgroup.boolean_expressions.DNF.pop_and}}\pysiglinewithargsret{\sphinxbfcode{\sphinxupquote{pop\_and}}}{}{}
\end{fulllineitems}


\end{fulllineitems}

\index{Disjunction (class in pysubgroup.boolean\_expressions)@\spxentry{Disjunction}\spxextra{class in pysubgroup.boolean\_expressions}}

\begin{fulllineitems}
\phantomsection\label{\detokenize{source/pysubgroup:pysubgroup.boolean_expressions.Disjunction}}\pysiglinewithargsret{\sphinxbfcode{\sphinxupquote{class }}\sphinxbfcode{\sphinxupquote{Disjunction}}}{\emph{selectors}}{}~\index{append\_and() (Disjunction method)@\spxentry{append\_and()}\spxextra{Disjunction method}}

\begin{fulllineitems}
\phantomsection\label{\detokenize{source/pysubgroup:pysubgroup.boolean_expressions.Disjunction.append_and}}\pysiglinewithargsret{\sphinxbfcode{\sphinxupquote{append\_and}}}{\emph{to\_append}}{}
\end{fulllineitems}

\index{append\_or() (Disjunction method)@\spxentry{append\_or()}\spxextra{Disjunction method}}

\begin{fulllineitems}
\phantomsection\label{\detokenize{source/pysubgroup:pysubgroup.boolean_expressions.Disjunction.append_or}}\pysiglinewithargsret{\sphinxbfcode{\sphinxupquote{append\_or}}}{\emph{to\_append}}{}
\end{fulllineitems}

\index{covers() (Disjunction method)@\spxentry{covers()}\spxextra{Disjunction method}}

\begin{fulllineitems}
\phantomsection\label{\detokenize{source/pysubgroup:pysubgroup.boolean_expressions.Disjunction.covers}}\pysiglinewithargsret{\sphinxbfcode{\sphinxupquote{covers}}}{\emph{instance}}{}
\end{fulllineitems}


\end{fulllineitems}



\subsection{pysubgroup.fi\_target module}
\label{\detokenize{source/pysubgroup:module-pysubgroup.fi_target}}\label{\detokenize{source/pysubgroup:pysubgroup-fi-target-module}}\index{pysubgroup.fi\_target (module)@\spxentry{pysubgroup.fi\_target}\spxextra{module}}
Created on 29.09.2017

@author: lemmerfn
\index{AreaQF (class in pysubgroup.fi\_target)@\spxentry{AreaQF}\spxextra{class in pysubgroup.fi\_target}}

\begin{fulllineitems}
\phantomsection\label{\detokenize{source/pysubgroup:pysubgroup.fi_target.AreaQF}}\pysigline{\sphinxbfcode{\sphinxupquote{class }}\sphinxbfcode{\sphinxupquote{AreaQF}}}~\index{evaluate() (AreaQF method)@\spxentry{evaluate()}\spxextra{AreaQF method}}

\begin{fulllineitems}
\phantomsection\label{\detokenize{source/pysubgroup:pysubgroup.fi_target.AreaQF.evaluate}}\pysiglinewithargsret{\sphinxbfcode{\sphinxupquote{evaluate}}}{\emph{subgroup}, \emph{statistics=None}}{}
\end{fulllineitems}

\index{is\_applicable() (AreaQF method)@\spxentry{is\_applicable()}\spxextra{AreaQF method}}

\begin{fulllineitems}
\phantomsection\label{\detokenize{source/pysubgroup:pysubgroup.fi_target.AreaQF.is_applicable}}\pysiglinewithargsret{\sphinxbfcode{\sphinxupquote{is\_applicable}}}{\emph{subgroup}}{}
\end{fulllineitems}

\index{supports\_weights() (AreaQF method)@\spxentry{supports\_weights()}\spxextra{AreaQF method}}

\begin{fulllineitems}
\phantomsection\label{\detokenize{source/pysubgroup:pysubgroup.fi_target.AreaQF.supports_weights}}\pysiglinewithargsret{\sphinxbfcode{\sphinxupquote{supports\_weights}}}{}{}
\end{fulllineitems}


\end{fulllineitems}

\index{CountQF (class in pysubgroup.fi\_target)@\spxentry{CountQF}\spxextra{class in pysubgroup.fi\_target}}

\begin{fulllineitems}
\phantomsection\label{\detokenize{source/pysubgroup:pysubgroup.fi_target.CountQF}}\pysigline{\sphinxbfcode{\sphinxupquote{class }}\sphinxbfcode{\sphinxupquote{CountQF}}}~\index{evaluate() (CountQF method)@\spxentry{evaluate()}\spxextra{CountQF method}}

\begin{fulllineitems}
\phantomsection\label{\detokenize{source/pysubgroup:pysubgroup.fi_target.CountQF.evaluate}}\pysiglinewithargsret{\sphinxbfcode{\sphinxupquote{evaluate}}}{\emph{subgroup}, \emph{statistics=None}}{}
\end{fulllineitems}

\index{is\_applicable() (CountQF method)@\spxentry{is\_applicable()}\spxextra{CountQF method}}

\begin{fulllineitems}
\phantomsection\label{\detokenize{source/pysubgroup:pysubgroup.fi_target.CountQF.is_applicable}}\pysiglinewithargsret{\sphinxbfcode{\sphinxupquote{is\_applicable}}}{\emph{subgroup}}{}
\end{fulllineitems}

\index{optimistic\_estimate() (CountQF method)@\spxentry{optimistic\_estimate()}\spxextra{CountQF method}}

\begin{fulllineitems}
\phantomsection\label{\detokenize{source/pysubgroup:pysubgroup.fi_target.CountQF.optimistic_estimate}}\pysiglinewithargsret{\sphinxbfcode{\sphinxupquote{optimistic\_estimate}}}{\emph{subgroup}, \emph{statistics=None}}{}
\end{fulllineitems}

\index{supports\_weights() (CountQF method)@\spxentry{supports\_weights()}\spxextra{CountQF method}}

\begin{fulllineitems}
\phantomsection\label{\detokenize{source/pysubgroup:pysubgroup.fi_target.CountQF.supports_weights}}\pysiglinewithargsret{\sphinxbfcode{\sphinxupquote{supports\_weights}}}{}{}
\end{fulllineitems}


\end{fulllineitems}

\index{FITarget (class in pysubgroup.fi\_target)@\spxentry{FITarget}\spxextra{class in pysubgroup.fi\_target}}

\begin{fulllineitems}
\phantomsection\label{\detokenize{source/pysubgroup:pysubgroup.fi_target.FITarget}}\pysigline{\sphinxbfcode{\sphinxupquote{class }}\sphinxbfcode{\sphinxupquote{FITarget}}}~\index{calculate\_statistics() (FITarget method)@\spxentry{calculate\_statistics()}\spxextra{FITarget method}}

\begin{fulllineitems}
\phantomsection\label{\detokenize{source/pysubgroup:pysubgroup.fi_target.FITarget.calculate_statistics}}\pysiglinewithargsret{\sphinxbfcode{\sphinxupquote{calculate\_statistics}}}{\emph{subgroup}, \emph{data}, \emph{weighting\_attribute=None}}{}
\end{fulllineitems}

\index{get\_attributes() (FITarget method)@\spxentry{get\_attributes()}\spxextra{FITarget method}}

\begin{fulllineitems}
\phantomsection\label{\detokenize{source/pysubgroup:pysubgroup.fi_target.FITarget.get_attributes}}\pysiglinewithargsret{\sphinxbfcode{\sphinxupquote{get\_attributes}}}{}{}
\end{fulllineitems}

\index{get\_base\_statistics() (FITarget method)@\spxentry{get\_base\_statistics()}\spxextra{FITarget method}}

\begin{fulllineitems}
\phantomsection\label{\detokenize{source/pysubgroup:pysubgroup.fi_target.FITarget.get_base_statistics}}\pysiglinewithargsret{\sphinxbfcode{\sphinxupquote{get\_base\_statistics}}}{\emph{data}, \emph{subgroup}, \emph{weighting\_attribute=None}}{}
\end{fulllineitems}


\end{fulllineitems}

\index{SimpleCountQF (class in pysubgroup.fi\_target)@\spxentry{SimpleCountQF}\spxextra{class in pysubgroup.fi\_target}}

\begin{fulllineitems}
\phantomsection\label{\detokenize{source/pysubgroup:pysubgroup.fi_target.SimpleCountQF}}\pysigline{\sphinxbfcode{\sphinxupquote{class }}\sphinxbfcode{\sphinxupquote{SimpleCountQF}}}~\index{calculate\_constant\_statistics() (SimpleCountQF method)@\spxentry{calculate\_constant\_statistics()}\spxextra{SimpleCountQF method}}

\begin{fulllineitems}
\phantomsection\label{\detokenize{source/pysubgroup:pysubgroup.fi_target.SimpleCountQF.calculate_constant_statistics}}\pysiglinewithargsret{\sphinxbfcode{\sphinxupquote{calculate\_constant\_statistics}}}{\emph{task}}{}
\end{fulllineitems}

\index{calculate\_statistics() (SimpleCountQF method)@\spxentry{calculate\_statistics()}\spxextra{SimpleCountQF method}}

\begin{fulllineitems}
\phantomsection\label{\detokenize{source/pysubgroup:pysubgroup.fi_target.SimpleCountQF.calculate_statistics}}\pysiglinewithargsret{\sphinxbfcode{\sphinxupquote{calculate\_statistics}}}{\emph{subgroup}, \emph{data=None}}{}
\end{fulllineitems}

\index{gp\_get\_null\_vector() (SimpleCountQF method)@\spxentry{gp\_get\_null\_vector()}\spxextra{SimpleCountQF method}}

\begin{fulllineitems}
\phantomsection\label{\detokenize{source/pysubgroup:pysubgroup.fi_target.SimpleCountQF.gp_get_null_vector}}\pysiglinewithargsret{\sphinxbfcode{\sphinxupquote{gp\_get\_null\_vector}}}{}{}
\end{fulllineitems}

\index{gp\_get\_params() (SimpleCountQF method)@\spxentry{gp\_get\_params()}\spxextra{SimpleCountQF method}}

\begin{fulllineitems}
\phantomsection\label{\detokenize{source/pysubgroup:pysubgroup.fi_target.SimpleCountQF.gp_get_params}}\pysiglinewithargsret{\sphinxbfcode{\sphinxupquote{gp\_get\_params}}}{\emph{\_cover\_arr}, \emph{v}}{}
\end{fulllineitems}

\index{gp\_get\_stats() (SimpleCountQF method)@\spxentry{gp\_get\_stats()}\spxextra{SimpleCountQF method}}

\begin{fulllineitems}
\phantomsection\label{\detokenize{source/pysubgroup:pysubgroup.fi_target.SimpleCountQF.gp_get_stats}}\pysiglinewithargsret{\sphinxbfcode{\sphinxupquote{gp\_get\_stats}}}{\emph{\_}}{}
\end{fulllineitems}

\index{gp\_merge() (SimpleCountQF method)@\spxentry{gp\_merge()}\spxextra{SimpleCountQF method}}

\begin{fulllineitems}
\phantomsection\label{\detokenize{source/pysubgroup:pysubgroup.fi_target.SimpleCountQF.gp_merge}}\pysiglinewithargsret{\sphinxbfcode{\sphinxupquote{gp\_merge}}}{\emph{l}, \emph{r}}{}
\end{fulllineitems}

\index{gp\_requires\_cover\_arr() (SimpleCountQF property)@\spxentry{gp\_requires\_cover\_arr()}\spxextra{SimpleCountQF property}}

\begin{fulllineitems}
\phantomsection\label{\detokenize{source/pysubgroup:pysubgroup.fi_target.SimpleCountQF.gp_requires_cover_arr}}\pysigline{\sphinxbfcode{\sphinxupquote{property }}\sphinxbfcode{\sphinxupquote{gp\_requires\_cover\_arr}}}
\end{fulllineitems}

\index{gp\_to\_str() (SimpleCountQF method)@\spxentry{gp\_to\_str()}\spxextra{SimpleCountQF method}}

\begin{fulllineitems}
\phantomsection\label{\detokenize{source/pysubgroup:pysubgroup.fi_target.SimpleCountQF.gp_to_str}}\pysiglinewithargsret{\sphinxbfcode{\sphinxupquote{gp\_to\_str}}}{\emph{stats}}{}
\end{fulllineitems}

\index{tpl (SimpleCountQF attribute)@\spxentry{tpl}\spxextra{SimpleCountQF attribute}}

\begin{fulllineitems}
\phantomsection\label{\detokenize{source/pysubgroup:pysubgroup.fi_target.SimpleCountQF.tpl}}\pysigline{\sphinxbfcode{\sphinxupquote{tpl}}}
alias of \sphinxcode{\sphinxupquote{CountQF\_parameters}}

\end{fulllineitems}


\end{fulllineitems}



\subsection{pysubgroup.gp\_growth module}
\label{\detokenize{source/pysubgroup:module-pysubgroup.gp_growth}}\label{\detokenize{source/pysubgroup:pysubgroup-gp-growth-module}}\index{pysubgroup.gp\_growth (module)@\spxentry{pysubgroup.gp\_growth}\spxextra{module}}\index{GpGrowth (class in pysubgroup.gp\_growth)@\spxentry{GpGrowth}\spxextra{class in pysubgroup.gp\_growth}}

\begin{fulllineitems}
\phantomsection\label{\detokenize{source/pysubgroup:pysubgroup.gp_growth.GpGrowth}}\pysiglinewithargsret{\sphinxbfcode{\sphinxupquote{class }}\sphinxbfcode{\sphinxupquote{GpGrowth}}}{\emph{mode=\textquotesingle{}b\_u\textquotesingle{}}}{}~\index{calculate\_quality\_function\_for\_patterns() (GpGrowth method)@\spxentry{calculate\_quality\_function\_for\_patterns()}\spxextra{GpGrowth method}}

\begin{fulllineitems}
\phantomsection\label{\detokenize{source/pysubgroup:pysubgroup.gp_growth.GpGrowth.calculate_quality_function_for_patterns}}\pysiglinewithargsret{\sphinxbfcode{\sphinxupquote{calculate\_quality\_function\_for\_patterns}}}{\emph{patterns}, \emph{selectors\_sorted}, \emph{arrs}}{}
\end{fulllineitems}

\index{check\_constraints() (GpGrowth method)@\spxentry{check\_constraints()}\spxextra{GpGrowth method}}

\begin{fulllineitems}
\phantomsection\label{\detokenize{source/pysubgroup:pysubgroup.gp_growth.GpGrowth.check_constraints}}\pysiglinewithargsret{\sphinxbfcode{\sphinxupquote{check\_constraints}}}{\emph{node}}{}
\end{fulllineitems}

\index{create\_copy\_of\_path() (GpGrowth method)@\spxentry{create\_copy\_of\_path()}\spxextra{GpGrowth method}}

\begin{fulllineitems}
\phantomsection\label{\detokenize{source/pysubgroup:pysubgroup.gp_growth.GpGrowth.create_copy_of_path}}\pysiglinewithargsret{\sphinxbfcode{\sphinxupquote{create\_copy\_of\_path}}}{\emph{nodes}, \emph{new\_nodes}, \emph{stats}}{}
\end{fulllineitems}

\index{create\_copy\_of\_tree\_top\_down() (GpGrowth method)@\spxentry{create\_copy\_of\_tree\_top\_down()}\spxextra{GpGrowth method}}

\begin{fulllineitems}
\phantomsection\label{\detokenize{source/pysubgroup:pysubgroup.gp_growth.GpGrowth.create_copy_of_tree_top_down}}\pysiglinewithargsret{\sphinxbfcode{\sphinxupquote{create\_copy\_of\_tree\_top\_down}}}{\emph{root}, \emph{nodes=None}, \emph{parent=None}}{}
\end{fulllineitems}

\index{create\_new\_tree\_from\_nodes() (GpGrowth method)@\spxentry{create\_new\_tree\_from\_nodes()}\spxextra{GpGrowth method}}

\begin{fulllineitems}
\phantomsection\label{\detokenize{source/pysubgroup:pysubgroup.gp_growth.GpGrowth.create_new_tree_from_nodes}}\pysiglinewithargsret{\sphinxbfcode{\sphinxupquote{create\_new\_tree\_from\_nodes}}}{\emph{nodes}}{}
\end{fulllineitems}

\index{execute() (GpGrowth method)@\spxentry{execute()}\spxextra{GpGrowth method}}

\begin{fulllineitems}
\phantomsection\label{\detokenize{source/pysubgroup:pysubgroup.gp_growth.GpGrowth.execute}}\pysiglinewithargsret{\sphinxbfcode{\sphinxupquote{execute}}}{\emph{task}}{}
\end{fulllineitems}

\index{get\_nodes\_upwards() (GpGrowth method)@\spxentry{get\_nodes\_upwards()}\spxextra{GpGrowth method}}

\begin{fulllineitems}
\phantomsection\label{\detokenize{source/pysubgroup:pysubgroup.gp_growth.GpGrowth.get_nodes_upwards}}\pysiglinewithargsret{\sphinxbfcode{\sphinxupquote{get\_nodes\_upwards}}}{\emph{node}}{}
\end{fulllineitems}

\index{get\_prefixes\_top\_down() (GpGrowth method)@\spxentry{get\_prefixes\_top\_down()}\spxextra{GpGrowth method}}

\begin{fulllineitems}
\phantomsection\label{\detokenize{source/pysubgroup:pysubgroup.gp_growth.GpGrowth.get_prefixes_top_down}}\pysiglinewithargsret{\sphinxbfcode{\sphinxupquote{get\_prefixes\_top\_down}}}{\emph{alpha}, \emph{max\_length}}{}
\end{fulllineitems}

\index{get\_stats\_for\_class() (GpGrowth method)@\spxentry{get\_stats\_for\_class()}\spxextra{GpGrowth method}}

\begin{fulllineitems}
\phantomsection\label{\detokenize{source/pysubgroup:pysubgroup.gp_growth.GpGrowth.get_stats_for_class}}\pysiglinewithargsret{\sphinxbfcode{\sphinxupquote{get\_stats\_for\_class}}}{\emph{cls\_nodes}}{}
\end{fulllineitems}

\index{get\_top\_down\_tree\_for\_class() (GpGrowth method)@\spxentry{get\_top\_down\_tree\_for\_class()}\spxextra{GpGrowth method}}

\begin{fulllineitems}
\phantomsection\label{\detokenize{source/pysubgroup:pysubgroup.gp_growth.GpGrowth.get_top_down_tree_for_class}}\pysiglinewithargsret{\sphinxbfcode{\sphinxupquote{get\_top\_down\_tree\_for\_class}}}{\emph{cls\_nodes}, \emph{cls}}{}
\end{fulllineitems}

\index{insert\_into\_tree() (GpGrowth method)@\spxentry{insert\_into\_tree()}\spxextra{GpGrowth method}}

\begin{fulllineitems}
\phantomsection\label{\detokenize{source/pysubgroup:pysubgroup.gp_growth.GpGrowth.insert_into_tree}}\pysiglinewithargsret{\sphinxbfcode{\sphinxupquote{insert\_into\_tree}}}{\emph{root}, \emph{nodes}, \emph{new\_stats}, \emph{classes}, \emph{max\_depth}}{}
Creates a tree of a maximum depth = depth

\end{fulllineitems}

\index{merge\_trees\_top\_down() (GpGrowth method)@\spxentry{merge\_trees\_top\_down()}\spxextra{GpGrowth method}}

\begin{fulllineitems}
\phantomsection\label{\detokenize{source/pysubgroup:pysubgroup.gp_growth.GpGrowth.merge_trees_top_down}}\pysiglinewithargsret{\sphinxbfcode{\sphinxupquote{merge\_trees\_top\_down}}}{\emph{nodes}, \emph{mutable\_root}, \emph{other\_root}}{}
\end{fulllineitems}

\index{nodes\_to\_cls\_nodes() (GpGrowth method)@\spxentry{nodes\_to\_cls\_nodes()}\spxextra{GpGrowth method}}

\begin{fulllineitems}
\phantomsection\label{\detokenize{source/pysubgroup:pysubgroup.gp_growth.GpGrowth.nodes_to_cls_nodes}}\pysiglinewithargsret{\sphinxbfcode{\sphinxupquote{nodes\_to\_cls\_nodes}}}{\emph{nodes}}{}
\end{fulllineitems}

\index{normal\_insert() (GpGrowth method)@\spxentry{normal\_insert()}\spxextra{GpGrowth method}}

\begin{fulllineitems}
\phantomsection\label{\detokenize{source/pysubgroup:pysubgroup.gp_growth.GpGrowth.normal_insert}}\pysiglinewithargsret{\sphinxbfcode{\sphinxupquote{normal\_insert}}}{\emph{root}, \emph{nodes}, \emph{new\_stats}, \emph{classes}}{}
\end{fulllineitems}

\index{prepare\_selectors() (GpGrowth method)@\spxentry{prepare\_selectors()}\spxextra{GpGrowth method}}

\begin{fulllineitems}
\phantomsection\label{\detokenize{source/pysubgroup:pysubgroup.gp_growth.GpGrowth.prepare_selectors}}\pysiglinewithargsret{\sphinxbfcode{\sphinxupquote{prepare\_selectors}}}{\emph{search\_space}}{}
\end{fulllineitems}

\index{recurse() (GpGrowth method)@\spxentry{recurse()}\spxextra{GpGrowth method}}

\begin{fulllineitems}
\phantomsection\label{\detokenize{source/pysubgroup:pysubgroup.gp_growth.GpGrowth.recurse}}\pysiglinewithargsret{\sphinxbfcode{\sphinxupquote{recurse}}}{\emph{cls\_nodes}, \emph{prefix}, \emph{is\_single\_path=False}}{}
\end{fulllineitems}

\index{recurse\_top\_down() (GpGrowth method)@\spxentry{recurse\_top\_down()}\spxextra{GpGrowth method}}

\begin{fulllineitems}
\phantomsection\label{\detokenize{source/pysubgroup:pysubgroup.gp_growth.GpGrowth.recurse_top_down}}\pysiglinewithargsret{\sphinxbfcode{\sphinxupquote{recurse\_top\_down}}}{\emph{cls\_nodes}, \emph{root}, \emph{depth\_in=0}}{}
\end{fulllineitems}

\index{remove\_infrequent\_class() (GpGrowth method)@\spxentry{remove\_infrequent\_class()}\spxextra{GpGrowth method}}

\begin{fulllineitems}
\phantomsection\label{\detokenize{source/pysubgroup:pysubgroup.gp_growth.GpGrowth.remove_infrequent_class}}\pysiglinewithargsret{\sphinxbfcode{\sphinxupquote{remove\_infrequent\_class}}}{\emph{nodes}, \emph{cls\_nodes}, \emph{stats\_dict}}{}
\end{fulllineitems}

\index{remove\_infrequent\_nodes() (GpGrowth method)@\spxentry{remove\_infrequent\_nodes()}\spxextra{GpGrowth method}}

\begin{fulllineitems}
\phantomsection\label{\detokenize{source/pysubgroup:pysubgroup.gp_growth.GpGrowth.remove_infrequent_nodes}}\pysiglinewithargsret{\sphinxbfcode{\sphinxupquote{remove\_infrequent\_nodes}}}{\emph{new\_nodes}}{}
\end{fulllineitems}

\index{to\_file() (GpGrowth method)@\spxentry{to\_file()}\spxextra{GpGrowth method}}

\begin{fulllineitems}
\phantomsection\label{\detokenize{source/pysubgroup:pysubgroup.gp_growth.GpGrowth.to_file}}\pysiglinewithargsret{\sphinxbfcode{\sphinxupquote{to\_file}}}{\emph{task}, \emph{path}}{}
\end{fulllineitems}


\end{fulllineitems}



\subsection{pysubgroup.measures module}
\label{\detokenize{source/pysubgroup:module-pysubgroup.measures}}\label{\detokenize{source/pysubgroup:pysubgroup-measures-module}}\index{pysubgroup.measures (module)@\spxentry{pysubgroup.measures}\spxextra{module}}
Created on 28.04.2016

@author: lemmerfn
\index{AbstractInterestingnessMeasure (class in pysubgroup.measures)@\spxentry{AbstractInterestingnessMeasure}\spxextra{class in pysubgroup.measures}}

\begin{fulllineitems}
\phantomsection\label{\detokenize{source/pysubgroup:pysubgroup.measures.AbstractInterestingnessMeasure}}\pysigline{\sphinxbfcode{\sphinxupquote{class }}\sphinxbfcode{\sphinxupquote{AbstractInterestingnessMeasure}}}~\index{ensure\_statistics() (AbstractInterestingnessMeasure method)@\spxentry{ensure\_statistics()}\spxextra{AbstractInterestingnessMeasure method}}

\begin{fulllineitems}
\phantomsection\label{\detokenize{source/pysubgroup:pysubgroup.measures.AbstractInterestingnessMeasure.ensure_statistics}}\pysiglinewithargsret{\sphinxbfcode{\sphinxupquote{ensure\_statistics}}}{\emph{subgroup}, \emph{statistics\_or\_data}}{}
\end{fulllineitems}

\index{is\_applicable() (AbstractInterestingnessMeasure method)@\spxentry{is\_applicable()}\spxextra{AbstractInterestingnessMeasure method}}

\begin{fulllineitems}
\phantomsection\label{\detokenize{source/pysubgroup:pysubgroup.measures.AbstractInterestingnessMeasure.is_applicable}}\pysiglinewithargsret{\sphinxbfcode{\sphinxupquote{abstract }}\sphinxbfcode{\sphinxupquote{is\_applicable}}}{\emph{subgroup}}{}
\end{fulllineitems}

\index{supports\_weights() (AbstractInterestingnessMeasure method)@\spxentry{supports\_weights()}\spxextra{AbstractInterestingnessMeasure method}}

\begin{fulllineitems}
\phantomsection\label{\detokenize{source/pysubgroup:pysubgroup.measures.AbstractInterestingnessMeasure.supports_weights}}\pysiglinewithargsret{\sphinxbfcode{\sphinxupquote{abstract }}\sphinxbfcode{\sphinxupquote{supports\_weights}}}{}{}
\end{fulllineitems}


\end{fulllineitems}

\index{BoundedInterestingnessMeasure (class in pysubgroup.measures)@\spxentry{BoundedInterestingnessMeasure}\spxextra{class in pysubgroup.measures}}

\begin{fulllineitems}
\phantomsection\label{\detokenize{source/pysubgroup:pysubgroup.measures.BoundedInterestingnessMeasure}}\pysigline{\sphinxbfcode{\sphinxupquote{class }}\sphinxbfcode{\sphinxupquote{BoundedInterestingnessMeasure}}}
\end{fulllineitems}

\index{CombinedInterestingnessMeasure (class in pysubgroup.measures)@\spxentry{CombinedInterestingnessMeasure}\spxextra{class in pysubgroup.measures}}

\begin{fulllineitems}
\phantomsection\label{\detokenize{source/pysubgroup:pysubgroup.measures.CombinedInterestingnessMeasure}}\pysiglinewithargsret{\sphinxbfcode{\sphinxupquote{class }}\sphinxbfcode{\sphinxupquote{CombinedInterestingnessMeasure}}}{\emph{measures}, \emph{weights=None}}{}~\index{evaluate\_from\_dataset() (CombinedInterestingnessMeasure method)@\spxentry{evaluate\_from\_dataset()}\spxextra{CombinedInterestingnessMeasure method}}

\begin{fulllineitems}
\phantomsection\label{\detokenize{source/pysubgroup:pysubgroup.measures.CombinedInterestingnessMeasure.evaluate_from_dataset}}\pysiglinewithargsret{\sphinxbfcode{\sphinxupquote{evaluate\_from\_dataset}}}{\emph{data}, \emph{subgroup}, \emph{weighting\_attribute=None}}{}
\end{fulllineitems}

\index{evaluate\_from\_statistics() (CombinedInterestingnessMeasure method)@\spxentry{evaluate\_from\_statistics()}\spxextra{CombinedInterestingnessMeasure method}}

\begin{fulllineitems}
\phantomsection\label{\detokenize{source/pysubgroup:pysubgroup.measures.CombinedInterestingnessMeasure.evaluate_from_statistics}}\pysiglinewithargsret{\sphinxbfcode{\sphinxupquote{evaluate\_from\_statistics}}}{\emph{instances\_dataset}, \emph{positives\_dataset}, \emph{instances\_subgroup}, \emph{positives\_subgroup}}{}
\end{fulllineitems}

\index{is\_applicable() (CombinedInterestingnessMeasure method)@\spxentry{is\_applicable()}\spxextra{CombinedInterestingnessMeasure method}}

\begin{fulllineitems}
\phantomsection\label{\detokenize{source/pysubgroup:pysubgroup.measures.CombinedInterestingnessMeasure.is_applicable}}\pysiglinewithargsret{\sphinxbfcode{\sphinxupquote{is\_applicable}}}{\emph{subgroup}}{}
\end{fulllineitems}

\index{optimistic\_estimate\_from\_dataset() (CombinedInterestingnessMeasure method)@\spxentry{optimistic\_estimate\_from\_dataset()}\spxextra{CombinedInterestingnessMeasure method}}

\begin{fulllineitems}
\phantomsection\label{\detokenize{source/pysubgroup:pysubgroup.measures.CombinedInterestingnessMeasure.optimistic_estimate_from_dataset}}\pysiglinewithargsret{\sphinxbfcode{\sphinxupquote{optimistic\_estimate\_from\_dataset}}}{\emph{data}, \emph{subgroup}, \emph{weighting\_attribute=None}}{}
\end{fulllineitems}

\index{optimistic\_estimate\_from\_statistics() (CombinedInterestingnessMeasure method)@\spxentry{optimistic\_estimate\_from\_statistics()}\spxextra{CombinedInterestingnessMeasure method}}

\begin{fulllineitems}
\phantomsection\label{\detokenize{source/pysubgroup:pysubgroup.measures.CombinedInterestingnessMeasure.optimistic_estimate_from_statistics}}\pysiglinewithargsret{\sphinxbfcode{\sphinxupquote{optimistic\_estimate\_from\_statistics}}}{\emph{instances\_dataset}, \emph{positives\_dataset}, \emph{instances\_subgroup}, \emph{positives\_subgroup}}{}
\end{fulllineitems}

\index{supports\_weights() (CombinedInterestingnessMeasure method)@\spxentry{supports\_weights()}\spxextra{CombinedInterestingnessMeasure method}}

\begin{fulllineitems}
\phantomsection\label{\detokenize{source/pysubgroup:pysubgroup.measures.CombinedInterestingnessMeasure.supports_weights}}\pysiglinewithargsret{\sphinxbfcode{\sphinxupquote{supports\_weights}}}{}{}
\end{fulllineitems}


\end{fulllineitems}

\index{CountCallsInterestingMeasure (class in pysubgroup.measures)@\spxentry{CountCallsInterestingMeasure}\spxextra{class in pysubgroup.measures}}

\begin{fulllineitems}
\phantomsection\label{\detokenize{source/pysubgroup:pysubgroup.measures.CountCallsInterestingMeasure}}\pysiglinewithargsret{\sphinxbfcode{\sphinxupquote{class }}\sphinxbfcode{\sphinxupquote{CountCallsInterestingMeasure}}}{\emph{qf}}{}~\index{calculate\_statistics() (CountCallsInterestingMeasure method)@\spxentry{calculate\_statistics()}\spxextra{CountCallsInterestingMeasure method}}

\begin{fulllineitems}
\phantomsection\label{\detokenize{source/pysubgroup:pysubgroup.measures.CountCallsInterestingMeasure.calculate_statistics}}\pysiglinewithargsret{\sphinxbfcode{\sphinxupquote{calculate\_statistics}}}{\emph{sg}, \emph{data=None}}{}
\end{fulllineitems}

\index{is\_applicable() (CountCallsInterestingMeasure method)@\spxentry{is\_applicable()}\spxextra{CountCallsInterestingMeasure method}}

\begin{fulllineitems}
\phantomsection\label{\detokenize{source/pysubgroup:pysubgroup.measures.CountCallsInterestingMeasure.is_applicable}}\pysiglinewithargsret{\sphinxbfcode{\sphinxupquote{is\_applicable}}}{\emph{subgroup}}{}
\end{fulllineitems}

\index{supports\_weights() (CountCallsInterestingMeasure method)@\spxentry{supports\_weights()}\spxextra{CountCallsInterestingMeasure method}}

\begin{fulllineitems}
\phantomsection\label{\detokenize{source/pysubgroup:pysubgroup.measures.CountCallsInterestingMeasure.supports_weights}}\pysiglinewithargsret{\sphinxbfcode{\sphinxupquote{supports\_weights}}}{}{}
\end{fulllineitems}


\end{fulllineitems}

\index{GeneralizationAwareQF (class in pysubgroup.measures)@\spxentry{GeneralizationAwareQF}\spxextra{class in pysubgroup.measures}}

\begin{fulllineitems}
\phantomsection\label{\detokenize{source/pysubgroup:pysubgroup.measures.GeneralizationAwareQF}}\pysiglinewithargsret{\sphinxbfcode{\sphinxupquote{class }}\sphinxbfcode{\sphinxupquote{GeneralizationAwareQF}}}{\emph{qf}}{}~\index{calculate\_constant\_statistics() (GeneralizationAwareQF method)@\spxentry{calculate\_constant\_statistics()}\spxextra{GeneralizationAwareQF method}}

\begin{fulllineitems}
\phantomsection\label{\detokenize{source/pysubgroup:pysubgroup.measures.GeneralizationAwareQF.calculate_constant_statistics}}\pysiglinewithargsret{\sphinxbfcode{\sphinxupquote{calculate\_constant\_statistics}}}{\emph{task}}{}
\end{fulllineitems}

\index{calculate\_statistics() (GeneralizationAwareQF method)@\spxentry{calculate\_statistics()}\spxextra{GeneralizationAwareQF method}}

\begin{fulllineitems}
\phantomsection\label{\detokenize{source/pysubgroup:pysubgroup.measures.GeneralizationAwareQF.calculate_statistics}}\pysiglinewithargsret{\sphinxbfcode{\sphinxupquote{calculate\_statistics}}}{\emph{subgroup}, \emph{data=None}}{}
\end{fulllineitems}

\index{evaluate() (GeneralizationAwareQF method)@\spxentry{evaluate()}\spxextra{GeneralizationAwareQF method}}

\begin{fulllineitems}
\phantomsection\label{\detokenize{source/pysubgroup:pysubgroup.measures.GeneralizationAwareQF.evaluate}}\pysiglinewithargsret{\sphinxbfcode{\sphinxupquote{evaluate}}}{\emph{subgroup}, \emph{statistics\_or\_data=None}}{}
\end{fulllineitems}

\index{GeneralizationAwareQF.ga\_tuple (class in pysubgroup.measures)@\spxentry{GeneralizationAwareQF.ga\_tuple}\spxextra{class in pysubgroup.measures}}

\begin{fulllineitems}
\phantomsection\label{\detokenize{source/pysubgroup:pysubgroup.measures.GeneralizationAwareQF.ga_tuple}}\pysiglinewithargsret{\sphinxbfcode{\sphinxupquote{class }}\sphinxbfcode{\sphinxupquote{ga\_tuple}}}{\emph{subgroup\_quality}, \emph{generalisation\_quality}}{}
Create new instance of ga\_tuple(subgroup\_quality, generalisation\_quality)
\index{generalisation\_quality() (GeneralizationAwareQF.ga\_tuple property)@\spxentry{generalisation\_quality()}\spxextra{GeneralizationAwareQF.ga\_tuple property}}

\begin{fulllineitems}
\phantomsection\label{\detokenize{source/pysubgroup:pysubgroup.measures.GeneralizationAwareQF.ga_tuple.generalisation_quality}}\pysigline{\sphinxbfcode{\sphinxupquote{property }}\sphinxbfcode{\sphinxupquote{generalisation\_quality}}}
Alias for field number 1

\end{fulllineitems}

\index{subgroup\_quality() (GeneralizationAwareQF.ga\_tuple property)@\spxentry{subgroup\_quality()}\spxextra{GeneralizationAwareQF.ga\_tuple property}}

\begin{fulllineitems}
\phantomsection\label{\detokenize{source/pysubgroup:pysubgroup.measures.GeneralizationAwareQF.ga_tuple.subgroup_quality}}\pysigline{\sphinxbfcode{\sphinxupquote{property }}\sphinxbfcode{\sphinxupquote{subgroup\_quality}}}
Alias for field number 0

\end{fulllineitems}


\end{fulllineitems}

\index{get\_qual\_and\_previous\_qual() (GeneralizationAwareQF method)@\spxentry{get\_qual\_and\_previous\_qual()}\spxextra{GeneralizationAwareQF method}}

\begin{fulllineitems}
\phantomsection\label{\detokenize{source/pysubgroup:pysubgroup.measures.GeneralizationAwareQF.get_qual_and_previous_qual}}\pysiglinewithargsret{\sphinxbfcode{\sphinxupquote{get\_qual\_and\_previous\_qual}}}{\emph{subgroup}, \emph{data}}{}
\end{fulllineitems}

\index{is\_applicable() (GeneralizationAwareQF method)@\spxentry{is\_applicable()}\spxextra{GeneralizationAwareQF method}}

\begin{fulllineitems}
\phantomsection\label{\detokenize{source/pysubgroup:pysubgroup.measures.GeneralizationAwareQF.is_applicable}}\pysiglinewithargsret{\sphinxbfcode{\sphinxupquote{is\_applicable}}}{\emph{subgroup}}{}
\end{fulllineitems}

\index{supports\_weights() (GeneralizationAwareQF method)@\spxentry{supports\_weights()}\spxextra{GeneralizationAwareQF method}}

\begin{fulllineitems}
\phantomsection\label{\detokenize{source/pysubgroup:pysubgroup.measures.GeneralizationAwareQF.supports_weights}}\pysiglinewithargsret{\sphinxbfcode{\sphinxupquote{supports\_weights}}}{}{}
\end{fulllineitems}


\end{fulllineitems}

\index{GeneralizationAwareQF\_stats (class in pysubgroup.measures)@\spxentry{GeneralizationAwareQF\_stats}\spxextra{class in pysubgroup.measures}}

\begin{fulllineitems}
\phantomsection\label{\detokenize{source/pysubgroup:pysubgroup.measures.GeneralizationAwareQF_stats}}\pysiglinewithargsret{\sphinxbfcode{\sphinxupquote{class }}\sphinxbfcode{\sphinxupquote{GeneralizationAwareQF\_stats}}}{\emph{qf}}{}~\index{calculate\_constant\_statistics() (GeneralizationAwareQF\_stats method)@\spxentry{calculate\_constant\_statistics()}\spxextra{GeneralizationAwareQF\_stats method}}

\begin{fulllineitems}
\phantomsection\label{\detokenize{source/pysubgroup:pysubgroup.measures.GeneralizationAwareQF_stats.calculate_constant_statistics}}\pysiglinewithargsret{\sphinxbfcode{\sphinxupquote{calculate\_constant\_statistics}}}{\emph{task}}{}
\end{fulllineitems}

\index{calculate\_statistics() (GeneralizationAwareQF\_stats method)@\spxentry{calculate\_statistics()}\spxextra{GeneralizationAwareQF\_stats method}}

\begin{fulllineitems}
\phantomsection\label{\detokenize{source/pysubgroup:pysubgroup.measures.GeneralizationAwareQF_stats.calculate_statistics}}\pysiglinewithargsret{\sphinxbfcode{\sphinxupquote{calculate\_statistics}}}{\emph{subgroup}, \emph{data=None}}{}
\end{fulllineitems}

\index{evaluate() (GeneralizationAwareQF\_stats method)@\spxentry{evaluate()}\spxextra{GeneralizationAwareQF\_stats method}}

\begin{fulllineitems}
\phantomsection\label{\detokenize{source/pysubgroup:pysubgroup.measures.GeneralizationAwareQF_stats.evaluate}}\pysiglinewithargsret{\sphinxbfcode{\sphinxupquote{evaluate}}}{\emph{subgroup}, \emph{statistics\_or\_data=None}}{}
\end{fulllineitems}

\index{ga\_tuple (GeneralizationAwareQF\_stats attribute)@\spxentry{ga\_tuple}\spxextra{GeneralizationAwareQF\_stats attribute}}

\begin{fulllineitems}
\phantomsection\label{\detokenize{source/pysubgroup:pysubgroup.measures.GeneralizationAwareQF_stats.ga_tuple}}\pysigline{\sphinxbfcode{\sphinxupquote{ga\_tuple}}}
alias of \sphinxcode{\sphinxupquote{ga\_stats\_tuple}}

\end{fulllineitems}

\index{get\_max() (GeneralizationAwareQF\_stats method)@\spxentry{get\_max()}\spxextra{GeneralizationAwareQF\_stats method}}

\begin{fulllineitems}
\phantomsection\label{\detokenize{source/pysubgroup:pysubgroup.measures.GeneralizationAwareQF_stats.get_max}}\pysiglinewithargsret{\sphinxbfcode{\sphinxupquote{get\_max}}}{\emph{*args}}{}
\end{fulllineitems}

\index{get\_stats\_and\_previous\_stats() (GeneralizationAwareQF\_stats method)@\spxentry{get\_stats\_and\_previous\_stats()}\spxextra{GeneralizationAwareQF\_stats method}}

\begin{fulllineitems}
\phantomsection\label{\detokenize{source/pysubgroup:pysubgroup.measures.GeneralizationAwareQF_stats.get_stats_and_previous_stats}}\pysiglinewithargsret{\sphinxbfcode{\sphinxupquote{get\_stats\_and\_previous\_stats}}}{\emph{subgroup}, \emph{data}}{}
\end{fulllineitems}

\index{is\_applicable() (GeneralizationAwareQF\_stats method)@\spxentry{is\_applicable()}\spxextra{GeneralizationAwareQF\_stats method}}

\begin{fulllineitems}
\phantomsection\label{\detokenize{source/pysubgroup:pysubgroup.measures.GeneralizationAwareQF_stats.is_applicable}}\pysiglinewithargsret{\sphinxbfcode{\sphinxupquote{is\_applicable}}}{\emph{subgroup}}{}
\end{fulllineitems}

\index{supports\_weights() (GeneralizationAwareQF\_stats method)@\spxentry{supports\_weights()}\spxextra{GeneralizationAwareQF\_stats method}}

\begin{fulllineitems}
\phantomsection\label{\detokenize{source/pysubgroup:pysubgroup.measures.GeneralizationAwareQF_stats.supports_weights}}\pysiglinewithargsret{\sphinxbfcode{\sphinxupquote{supports\_weights}}}{}{}
\end{fulllineitems}


\end{fulllineitems}

\index{maximum\_statistic\_filter() (in module pysubgroup.measures)@\spxentry{maximum\_statistic\_filter()}\spxextra{in module pysubgroup.measures}}

\begin{fulllineitems}
\phantomsection\label{\detokenize{source/pysubgroup:pysubgroup.measures.maximum_statistic_filter}}\pysiglinewithargsret{\sphinxbfcode{\sphinxupquote{maximum\_statistic\_filter}}}{\emph{result\_set}, \emph{statistic}, \emph{maximum}}{}
\end{fulllineitems}

\index{minimum\_quality\_filter() (in module pysubgroup.measures)@\spxentry{minimum\_quality\_filter()}\spxextra{in module pysubgroup.measures}}

\begin{fulllineitems}
\phantomsection\label{\detokenize{source/pysubgroup:pysubgroup.measures.minimum_quality_filter}}\pysiglinewithargsret{\sphinxbfcode{\sphinxupquote{minimum\_quality\_filter}}}{\emph{result\_set}, \emph{minimum}}{}
\end{fulllineitems}

\index{minimum\_statistic\_filter() (in module pysubgroup.measures)@\spxentry{minimum\_statistic\_filter()}\spxextra{in module pysubgroup.measures}}

\begin{fulllineitems}
\phantomsection\label{\detokenize{source/pysubgroup:pysubgroup.measures.minimum_statistic_filter}}\pysiglinewithargsret{\sphinxbfcode{\sphinxupquote{minimum\_statistic\_filter}}}{\emph{result\_set}, \emph{statistic}, \emph{minimum}, \emph{data}}{}
\end{fulllineitems}

\index{overlap\_filter() (in module pysubgroup.measures)@\spxentry{overlap\_filter()}\spxextra{in module pysubgroup.measures}}

\begin{fulllineitems}
\phantomsection\label{\detokenize{source/pysubgroup:pysubgroup.measures.overlap_filter}}\pysiglinewithargsret{\sphinxbfcode{\sphinxupquote{overlap\_filter}}}{\emph{result\_set}, \emph{data}, \emph{similarity\_level=0.9}}{}
\end{fulllineitems}

\index{overlaps\_list() (in module pysubgroup.measures)@\spxentry{overlaps\_list()}\spxextra{in module pysubgroup.measures}}

\begin{fulllineitems}
\phantomsection\label{\detokenize{source/pysubgroup:pysubgroup.measures.overlaps_list}}\pysiglinewithargsret{\sphinxbfcode{\sphinxupquote{overlaps\_list}}}{\emph{sg}, \emph{list\_of\_sgs}, \emph{data}, \emph{similarity\_level=0.9}}{}
\end{fulllineitems}

\index{unique\_attributes() (in module pysubgroup.measures)@\spxentry{unique\_attributes()}\spxextra{in module pysubgroup.measures}}

\begin{fulllineitems}
\phantomsection\label{\detokenize{source/pysubgroup:pysubgroup.measures.unique_attributes}}\pysiglinewithargsret{\sphinxbfcode{\sphinxupquote{unique\_attributes}}}{\emph{result\_set}, \emph{data}}{}
\end{fulllineitems}



\subsection{pysubgroup.model\_target module}
\label{\detokenize{source/pysubgroup:module-pysubgroup.model_target}}\label{\detokenize{source/pysubgroup:pysubgroup-model-target-module}}\index{pysubgroup.model\_target (module)@\spxentry{pysubgroup.model\_target}\spxextra{module}}\index{EMM\_Likelihood (class in pysubgroup.model\_target)@\spxentry{EMM\_Likelihood}\spxextra{class in pysubgroup.model\_target}}

\begin{fulllineitems}
\phantomsection\label{\detokenize{source/pysubgroup:pysubgroup.model_target.EMM_Likelihood}}\pysiglinewithargsret{\sphinxbfcode{\sphinxupquote{class }}\sphinxbfcode{\sphinxupquote{EMM\_Likelihood}}}{\emph{model}}{}~\index{calculate\_constant\_statistics() (EMM\_Likelihood method)@\spxentry{calculate\_constant\_statistics()}\spxextra{EMM\_Likelihood method}}

\begin{fulllineitems}
\phantomsection\label{\detokenize{source/pysubgroup:pysubgroup.model_target.EMM_Likelihood.calculate_constant_statistics}}\pysiglinewithargsret{\sphinxbfcode{\sphinxupquote{calculate\_constant\_statistics}}}{\emph{task}}{}
\end{fulllineitems}

\index{calculate\_statistics() (EMM\_Likelihood method)@\spxentry{calculate\_statistics()}\spxextra{EMM\_Likelihood method}}

\begin{fulllineitems}
\phantomsection\label{\detokenize{source/pysubgroup:pysubgroup.model_target.EMM_Likelihood.calculate_statistics}}\pysiglinewithargsret{\sphinxbfcode{\sphinxupquote{calculate\_statistics}}}{\emph{subgroup}, \emph{data=None}}{}
\end{fulllineitems}

\index{evaluate() (EMM\_Likelihood method)@\spxentry{evaluate()}\spxextra{EMM\_Likelihood method}}

\begin{fulllineitems}
\phantomsection\label{\detokenize{source/pysubgroup:pysubgroup.model_target.EMM_Likelihood.evaluate}}\pysiglinewithargsret{\sphinxbfcode{\sphinxupquote{evaluate}}}{\emph{subgroup}, \emph{statistics=None}}{}
\end{fulllineitems}

\index{get\_tuple() (EMM\_Likelihood method)@\spxentry{get\_tuple()}\spxextra{EMM\_Likelihood method}}

\begin{fulllineitems}
\phantomsection\label{\detokenize{source/pysubgroup:pysubgroup.model_target.EMM_Likelihood.get_tuple}}\pysiglinewithargsret{\sphinxbfcode{\sphinxupquote{get\_tuple}}}{\emph{sg\_size}, \emph{params}, \emph{cover\_arr}}{}
\end{fulllineitems}

\index{gp\_get\_params() (EMM\_Likelihood method)@\spxentry{gp\_get\_params()}\spxextra{EMM\_Likelihood method}}

\begin{fulllineitems}
\phantomsection\label{\detokenize{source/pysubgroup:pysubgroup.model_target.EMM_Likelihood.gp_get_params}}\pysiglinewithargsret{\sphinxbfcode{\sphinxupquote{gp\_get\_params}}}{\emph{cover\_arr}, \emph{v}}{}
\end{fulllineitems}

\index{is\_applicable() (EMM\_Likelihood method)@\spxentry{is\_applicable()}\spxextra{EMM\_Likelihood method}}

\begin{fulllineitems}
\phantomsection\label{\detokenize{source/pysubgroup:pysubgroup.model_target.EMM_Likelihood.is_applicable}}\pysiglinewithargsret{\sphinxbfcode{\sphinxupquote{is\_applicable}}}{\emph{\_}}{}
\end{fulllineitems}

\index{supports\_weights() (EMM\_Likelihood method)@\spxentry{supports\_weights()}\spxextra{EMM\_Likelihood method}}

\begin{fulllineitems}
\phantomsection\label{\detokenize{source/pysubgroup:pysubgroup.model_target.EMM_Likelihood.supports_weights}}\pysiglinewithargsret{\sphinxbfcode{\sphinxupquote{supports\_weights}}}{}{}
\end{fulllineitems}

\index{tpl (EMM\_Likelihood attribute)@\spxentry{tpl}\spxextra{EMM\_Likelihood attribute}}

\begin{fulllineitems}
\phantomsection\label{\detokenize{source/pysubgroup:pysubgroup.model_target.EMM_Likelihood.tpl}}\pysigline{\sphinxbfcode{\sphinxupquote{tpl}}}
alias of {\hyperref[\detokenize{source/pysubgroup:pysubgroup.model_target.EMM_Likelihood}]{\sphinxcrossref{\sphinxcode{\sphinxupquote{EMM\_Likelihood}}}}}

\end{fulllineitems}


\end{fulllineitems}

\index{PolyRegression\_ModelClass (class in pysubgroup.model\_target)@\spxentry{PolyRegression\_ModelClass}\spxextra{class in pysubgroup.model\_target}}

\begin{fulllineitems}
\phantomsection\label{\detokenize{source/pysubgroup:pysubgroup.model_target.PolyRegression_ModelClass}}\pysiglinewithargsret{\sphinxbfcode{\sphinxupquote{class }}\sphinxbfcode{\sphinxupquote{PolyRegression\_ModelClass}}}{\emph{x\_name=\textquotesingle{}x\textquotesingle{}}, \emph{y\_name=\textquotesingle{}y\textquotesingle{}}, \emph{degree=1}}{}~\index{calculate\_constant\_statistics() (PolyRegression\_ModelClass method)@\spxentry{calculate\_constant\_statistics()}\spxextra{PolyRegression\_ModelClass method}}

\begin{fulllineitems}
\phantomsection\label{\detokenize{source/pysubgroup:pysubgroup.model_target.PolyRegression_ModelClass.calculate_constant_statistics}}\pysiglinewithargsret{\sphinxbfcode{\sphinxupquote{calculate\_constant\_statistics}}}{\emph{task}}{}
\end{fulllineitems}

\index{fit() (PolyRegression\_ModelClass method)@\spxentry{fit()}\spxextra{PolyRegression\_ModelClass method}}

\begin{fulllineitems}
\phantomsection\label{\detokenize{source/pysubgroup:pysubgroup.model_target.PolyRegression_ModelClass.fit}}\pysiglinewithargsret{\sphinxbfcode{\sphinxupquote{fit}}}{\emph{subgroup}, \emph{data=None}}{}
\end{fulllineitems}

\index{gp\_get\_null\_vector() (PolyRegression\_ModelClass method)@\spxentry{gp\_get\_null\_vector()}\spxextra{PolyRegression\_ModelClass method}}

\begin{fulllineitems}
\phantomsection\label{\detokenize{source/pysubgroup:pysubgroup.model_target.PolyRegression_ModelClass.gp_get_null_vector}}\pysiglinewithargsret{\sphinxbfcode{\sphinxupquote{gp\_get\_null\_vector}}}{}{}
\end{fulllineitems}

\index{gp\_get\_params() (PolyRegression\_ModelClass method)@\spxentry{gp\_get\_params()}\spxextra{PolyRegression\_ModelClass method}}

\begin{fulllineitems}
\phantomsection\label{\detokenize{source/pysubgroup:pysubgroup.model_target.PolyRegression_ModelClass.gp_get_params}}\pysiglinewithargsret{\sphinxbfcode{\sphinxupquote{gp\_get\_params}}}{\emph{v}}{}
\end{fulllineitems}

\index{gp\_get\_stats() (PolyRegression\_ModelClass method)@\spxentry{gp\_get\_stats()}\spxextra{PolyRegression\_ModelClass method}}

\begin{fulllineitems}
\phantomsection\label{\detokenize{source/pysubgroup:pysubgroup.model_target.PolyRegression_ModelClass.gp_get_stats}}\pysiglinewithargsret{\sphinxbfcode{\sphinxupquote{gp\_get\_stats}}}{\emph{row\_index}}{}
\end{fulllineitems}

\index{gp\_merge() (PolyRegression\_ModelClass static method)@\spxentry{gp\_merge()}\spxextra{PolyRegression\_ModelClass static method}}

\begin{fulllineitems}
\phantomsection\label{\detokenize{source/pysubgroup:pysubgroup.model_target.PolyRegression_ModelClass.gp_merge}}\pysiglinewithargsret{\sphinxbfcode{\sphinxupquote{static }}\sphinxbfcode{\sphinxupquote{gp\_merge}}}{\emph{u}, \emph{v}}{}
\end{fulllineitems}

\index{likelihood() (PolyRegression\_ModelClass method)@\spxentry{likelihood()}\spxextra{PolyRegression\_ModelClass method}}

\begin{fulllineitems}
\phantomsection\label{\detokenize{source/pysubgroup:pysubgroup.model_target.PolyRegression_ModelClass.likelihood}}\pysiglinewithargsret{\sphinxbfcode{\sphinxupquote{likelihood}}}{\emph{stats}, \emph{sg}}{}
\end{fulllineitems}

\index{loglikelihood() (PolyRegression\_ModelClass method)@\spxentry{loglikelihood()}\spxextra{PolyRegression\_ModelClass method}}

\begin{fulllineitems}
\phantomsection\label{\detokenize{source/pysubgroup:pysubgroup.model_target.PolyRegression_ModelClass.loglikelihood}}\pysiglinewithargsret{\sphinxbfcode{\sphinxupquote{loglikelihood}}}{\emph{stats}, \emph{sg}}{}
\end{fulllineitems}


\end{fulllineitems}

\index{beta\_tuple (class in pysubgroup.model\_target)@\spxentry{beta\_tuple}\spxextra{class in pysubgroup.model\_target}}

\begin{fulllineitems}
\phantomsection\label{\detokenize{source/pysubgroup:pysubgroup.model_target.beta_tuple}}\pysiglinewithargsret{\sphinxbfcode{\sphinxupquote{class }}\sphinxbfcode{\sphinxupquote{beta\_tuple}}}{\emph{beta}, \emph{size}}{}
Create new instance of beta\_tuple(beta, size)
\index{beta() (beta\_tuple property)@\spxentry{beta()}\spxextra{beta\_tuple property}}

\begin{fulllineitems}
\phantomsection\label{\detokenize{source/pysubgroup:pysubgroup.model_target.beta_tuple.beta}}\pysigline{\sphinxbfcode{\sphinxupquote{property }}\sphinxbfcode{\sphinxupquote{beta}}}
Alias for field number 0

\end{fulllineitems}

\index{size() (beta\_tuple property)@\spxentry{size()}\spxextra{beta\_tuple property}}

\begin{fulllineitems}
\phantomsection\label{\detokenize{source/pysubgroup:pysubgroup.model_target.beta_tuple.size}}\pysigline{\sphinxbfcode{\sphinxupquote{property }}\sphinxbfcode{\sphinxupquote{size}}}
Alias for field number 1

\end{fulllineitems}


\end{fulllineitems}



\subsection{pysubgroup.nominal\_target module}
\label{\detokenize{source/pysubgroup:pysubgroup-nominal-target-module}}

\subsection{pysubgroup.numeric\_target module}
\label{\detokenize{source/pysubgroup:module-pysubgroup.numeric_target}}\label{\detokenize{source/pysubgroup:pysubgroup-numeric-target-module}}\index{pysubgroup.numeric\_target (module)@\spxentry{pysubgroup.numeric\_target}\spxextra{module}}
Created on 29.09.2017

@author: lemmerfn
\index{GAStandardQFNumeric (class in pysubgroup.numeric\_target)@\spxentry{GAStandardQFNumeric}\spxextra{class in pysubgroup.numeric\_target}}

\begin{fulllineitems}
\phantomsection\label{\detokenize{source/pysubgroup:pysubgroup.numeric_target.GAStandardQFNumeric}}\pysiglinewithargsret{\sphinxbfcode{\sphinxupquote{class }}\sphinxbfcode{\sphinxupquote{GAStandardQFNumeric}}}{\emph{a}, \emph{invert=False}}{}~\index{evaluate\_from\_dataset() (GAStandardQFNumeric method)@\spxentry{evaluate\_from\_dataset()}\spxextra{GAStandardQFNumeric method}}

\begin{fulllineitems}
\phantomsection\label{\detokenize{source/pysubgroup:pysubgroup.numeric_target.GAStandardQFNumeric.evaluate_from_dataset}}\pysiglinewithargsret{\sphinxbfcode{\sphinxupquote{evaluate\_from\_dataset}}}{\emph{data}, \emph{subgroup}, \emph{weighting\_attribute=None}}{}
\end{fulllineitems}

\index{is\_applicable() (GAStandardQFNumeric method)@\spxentry{is\_applicable()}\spxextra{GAStandardQFNumeric method}}

\begin{fulllineitems}
\phantomsection\label{\detokenize{source/pysubgroup:pysubgroup.numeric_target.GAStandardQFNumeric.is_applicable}}\pysiglinewithargsret{\sphinxbfcode{\sphinxupquote{is\_applicable}}}{\emph{subgroup}}{}
\end{fulllineitems}

\index{supports\_weights() (GAStandardQFNumeric method)@\spxentry{supports\_weights()}\spxextra{GAStandardQFNumeric method}}

\begin{fulllineitems}
\phantomsection\label{\detokenize{source/pysubgroup:pysubgroup.numeric_target.GAStandardQFNumeric.supports_weights}}\pysiglinewithargsret{\sphinxbfcode{\sphinxupquote{supports\_weights}}}{}{}
\end{fulllineitems}


\end{fulllineitems}

\index{NumericTarget (class in pysubgroup.numeric\_target)@\spxentry{NumericTarget}\spxextra{class in pysubgroup.numeric\_target}}

\begin{fulllineitems}
\phantomsection\label{\detokenize{source/pysubgroup:pysubgroup.numeric_target.NumericTarget}}\pysiglinewithargsret{\sphinxbfcode{\sphinxupquote{class }}\sphinxbfcode{\sphinxupquote{NumericTarget}}}{\emph{target\_variable}}{}~\index{calculate\_statistics() (NumericTarget method)@\spxentry{calculate\_statistics()}\spxextra{NumericTarget method}}

\begin{fulllineitems}
\phantomsection\label{\detokenize{source/pysubgroup:pysubgroup.numeric_target.NumericTarget.calculate_statistics}}\pysiglinewithargsret{\sphinxbfcode{\sphinxupquote{calculate\_statistics}}}{\emph{subgroup}, \emph{data}}{}
\end{fulllineitems}

\index{get\_attributes() (NumericTarget method)@\spxentry{get\_attributes()}\spxextra{NumericTarget method}}

\begin{fulllineitems}
\phantomsection\label{\detokenize{source/pysubgroup:pysubgroup.numeric_target.NumericTarget.get_attributes}}\pysiglinewithargsret{\sphinxbfcode{\sphinxupquote{get\_attributes}}}{}{}
\end{fulllineitems}

\index{get\_base\_statistics() (NumericTarget method)@\spxentry{get\_base\_statistics()}\spxextra{NumericTarget method}}

\begin{fulllineitems}
\phantomsection\label{\detokenize{source/pysubgroup:pysubgroup.numeric_target.NumericTarget.get_base_statistics}}\pysiglinewithargsret{\sphinxbfcode{\sphinxupquote{get\_base\_statistics}}}{\emph{data}, \emph{subgroup}, \emph{weighting\_attribute=None}}{}
\end{fulllineitems}


\end{fulllineitems}

\index{StandardQFNumeric (class in pysubgroup.numeric\_target)@\spxentry{StandardQFNumeric}\spxextra{class in pysubgroup.numeric\_target}}

\begin{fulllineitems}
\phantomsection\label{\detokenize{source/pysubgroup:pysubgroup.numeric_target.StandardQFNumeric}}\pysiglinewithargsret{\sphinxbfcode{\sphinxupquote{class }}\sphinxbfcode{\sphinxupquote{StandardQFNumeric}}}{\emph{a}, \emph{invert=False}, \emph{estimator=\textquotesingle{}sum\textquotesingle{}}}{}~\index{StandardQFNumeric.Average\_Estimator (class in pysubgroup.numeric\_target)@\spxentry{StandardQFNumeric.Average\_Estimator}\spxextra{class in pysubgroup.numeric\_target}}

\begin{fulllineitems}
\phantomsection\label{\detokenize{source/pysubgroup:pysubgroup.numeric_target.StandardQFNumeric.Average_Estimator}}\pysiglinewithargsret{\sphinxbfcode{\sphinxupquote{class }}\sphinxbfcode{\sphinxupquote{Average\_Estimator}}}{\emph{qf}}{}~\index{calculate\_constant\_statistics() (StandardQFNumeric.Average\_Estimator method)@\spxentry{calculate\_constant\_statistics()}\spxextra{StandardQFNumeric.Average\_Estimator method}}

\begin{fulllineitems}
\phantomsection\label{\detokenize{source/pysubgroup:pysubgroup.numeric_target.StandardQFNumeric.Average_Estimator.calculate_constant_statistics}}\pysiglinewithargsret{\sphinxbfcode{\sphinxupquote{calculate\_constant\_statistics}}}{\emph{task}}{}
\end{fulllineitems}

\index{get\_data() (StandardQFNumeric.Average\_Estimator method)@\spxentry{get\_data()}\spxextra{StandardQFNumeric.Average\_Estimator method}}

\begin{fulllineitems}
\phantomsection\label{\detokenize{source/pysubgroup:pysubgroup.numeric_target.StandardQFNumeric.Average_Estimator.get_data}}\pysiglinewithargsret{\sphinxbfcode{\sphinxupquote{get\_data}}}{\emph{task}}{}
\end{fulllineitems}

\index{get\_estimate() (StandardQFNumeric.Average\_Estimator method)@\spxentry{get\_estimate()}\spxextra{StandardQFNumeric.Average\_Estimator method}}

\begin{fulllineitems}
\phantomsection\label{\detokenize{source/pysubgroup:pysubgroup.numeric_target.StandardQFNumeric.Average_Estimator.get_estimate}}\pysiglinewithargsret{\sphinxbfcode{\sphinxupquote{get\_estimate}}}{\emph{subgroup}, \emph{sg\_size}, \emph{sg\_mean}, \emph{cover\_arr}, \emph{\_}}{}
\end{fulllineitems}


\end{fulllineitems}

\index{StandardQFNumeric.Ordering\_Estimator (class in pysubgroup.numeric\_target)@\spxentry{StandardQFNumeric.Ordering\_Estimator}\spxextra{class in pysubgroup.numeric\_target}}

\begin{fulllineitems}
\phantomsection\label{\detokenize{source/pysubgroup:pysubgroup.numeric_target.StandardQFNumeric.Ordering_Estimator}}\pysiglinewithargsret{\sphinxbfcode{\sphinxupquote{class }}\sphinxbfcode{\sphinxupquote{Ordering\_Estimator}}}{\emph{qf}}{}~\index{calculate\_constant\_statistics() (StandardQFNumeric.Ordering\_Estimator method)@\spxentry{calculate\_constant\_statistics()}\spxextra{StandardQFNumeric.Ordering\_Estimator method}}

\begin{fulllineitems}
\phantomsection\label{\detokenize{source/pysubgroup:pysubgroup.numeric_target.StandardQFNumeric.Ordering_Estimator.calculate_constant_statistics}}\pysiglinewithargsret{\sphinxbfcode{\sphinxupquote{calculate\_constant\_statistics}}}{\emph{task}}{}
\end{fulllineitems}

\index{get\_data() (StandardQFNumeric.Ordering\_Estimator method)@\spxentry{get\_data()}\spxextra{StandardQFNumeric.Ordering\_Estimator method}}

\begin{fulllineitems}
\phantomsection\label{\detokenize{source/pysubgroup:pysubgroup.numeric_target.StandardQFNumeric.Ordering_Estimator.get_data}}\pysiglinewithargsret{\sphinxbfcode{\sphinxupquote{get\_data}}}{\emph{task}}{}
\end{fulllineitems}

\index{get\_estimate() (StandardQFNumeric.Ordering\_Estimator method)@\spxentry{get\_estimate()}\spxextra{StandardQFNumeric.Ordering\_Estimator method}}

\begin{fulllineitems}
\phantomsection\label{\detokenize{source/pysubgroup:pysubgroup.numeric_target.StandardQFNumeric.Ordering_Estimator.get_estimate}}\pysiglinewithargsret{\sphinxbfcode{\sphinxupquote{get\_estimate}}}{\emph{subgroup}, \emph{sg\_size}, \emph{sg\_mean}, \emph{cover\_arr}, \emph{target\_values\_sg}}{}
\end{fulllineitems}

\index{get\_estimate\_numpy() (StandardQFNumeric.Ordering\_Estimator method)@\spxentry{get\_estimate\_numpy()}\spxextra{StandardQFNumeric.Ordering\_Estimator method}}

\begin{fulllineitems}
\phantomsection\label{\detokenize{source/pysubgroup:pysubgroup.numeric_target.StandardQFNumeric.Ordering_Estimator.get_estimate_numpy}}\pysiglinewithargsret{\sphinxbfcode{\sphinxupquote{get\_estimate\_numpy}}}{\emph{values\_sg}, \emph{a}, \emph{mean\_dataset}}{}
\end{fulllineitems}


\end{fulllineitems}

\index{StandardQFNumeric.Summation\_Estimator (class in pysubgroup.numeric\_target)@\spxentry{StandardQFNumeric.Summation\_Estimator}\spxextra{class in pysubgroup.numeric\_target}}

\begin{fulllineitems}
\phantomsection\label{\detokenize{source/pysubgroup:pysubgroup.numeric_target.StandardQFNumeric.Summation_Estimator}}\pysiglinewithargsret{\sphinxbfcode{\sphinxupquote{class }}\sphinxbfcode{\sphinxupquote{Summation\_Estimator}}}{\emph{qf}}{}~\index{calculate\_constant\_statistics() (StandardQFNumeric.Summation\_Estimator method)@\spxentry{calculate\_constant\_statistics()}\spxextra{StandardQFNumeric.Summation\_Estimator method}}

\begin{fulllineitems}
\phantomsection\label{\detokenize{source/pysubgroup:pysubgroup.numeric_target.StandardQFNumeric.Summation_Estimator.calculate_constant_statistics}}\pysiglinewithargsret{\sphinxbfcode{\sphinxupquote{calculate\_constant\_statistics}}}{\emph{task}}{}
\end{fulllineitems}

\index{get\_data() (StandardQFNumeric.Summation\_Estimator method)@\spxentry{get\_data()}\spxextra{StandardQFNumeric.Summation\_Estimator method}}

\begin{fulllineitems}
\phantomsection\label{\detokenize{source/pysubgroup:pysubgroup.numeric_target.StandardQFNumeric.Summation_Estimator.get_data}}\pysiglinewithargsret{\sphinxbfcode{\sphinxupquote{get\_data}}}{\emph{task}}{}
\end{fulllineitems}

\index{get\_estimate() (StandardQFNumeric.Summation\_Estimator method)@\spxentry{get\_estimate()}\spxextra{StandardQFNumeric.Summation\_Estimator method}}

\begin{fulllineitems}
\phantomsection\label{\detokenize{source/pysubgroup:pysubgroup.numeric_target.StandardQFNumeric.Summation_Estimator.get_estimate}}\pysiglinewithargsret{\sphinxbfcode{\sphinxupquote{get\_estimate}}}{\emph{subgroup}, \emph{sg\_size}, \emph{sg\_mean}, \emph{cover\_arr}, \emph{\_}}{}
\end{fulllineitems}


\end{fulllineitems}

\index{calculate\_constant\_statistics() (StandardQFNumeric method)@\spxentry{calculate\_constant\_statistics()}\spxextra{StandardQFNumeric method}}

\begin{fulllineitems}
\phantomsection\label{\detokenize{source/pysubgroup:pysubgroup.numeric_target.StandardQFNumeric.calculate_constant_statistics}}\pysiglinewithargsret{\sphinxbfcode{\sphinxupquote{calculate\_constant\_statistics}}}{\emph{task}}{}
\end{fulllineitems}

\index{calculate\_statistics() (StandardQFNumeric method)@\spxentry{calculate\_statistics()}\spxextra{StandardQFNumeric method}}

\begin{fulllineitems}
\phantomsection\label{\detokenize{source/pysubgroup:pysubgroup.numeric_target.StandardQFNumeric.calculate_statistics}}\pysiglinewithargsret{\sphinxbfcode{\sphinxupquote{calculate\_statistics}}}{\emph{subgroup}, \emph{data=None}}{}
\end{fulllineitems}

\index{evaluate() (StandardQFNumeric method)@\spxentry{evaluate()}\spxextra{StandardQFNumeric method}}

\begin{fulllineitems}
\phantomsection\label{\detokenize{source/pysubgroup:pysubgroup.numeric_target.StandardQFNumeric.evaluate}}\pysiglinewithargsret{\sphinxbfcode{\sphinxupquote{evaluate}}}{\emph{subgroup}, \emph{statistics=None}}{}
\end{fulllineitems}

\index{is\_applicable() (StandardQFNumeric method)@\spxentry{is\_applicable()}\spxextra{StandardQFNumeric method}}

\begin{fulllineitems}
\phantomsection\label{\detokenize{source/pysubgroup:pysubgroup.numeric_target.StandardQFNumeric.is_applicable}}\pysiglinewithargsret{\sphinxbfcode{\sphinxupquote{is\_applicable}}}{\emph{subgroup}}{}
\end{fulllineitems}

\index{optimistic\_estimate() (StandardQFNumeric method)@\spxentry{optimistic\_estimate()}\spxextra{StandardQFNumeric method}}

\begin{fulllineitems}
\phantomsection\label{\detokenize{source/pysubgroup:pysubgroup.numeric_target.StandardQFNumeric.optimistic_estimate}}\pysiglinewithargsret{\sphinxbfcode{\sphinxupquote{optimistic\_estimate}}}{\emph{subgroup}, \emph{statistics=None}}{}
\end{fulllineitems}

\index{standard\_qf\_numeric() (StandardQFNumeric static method)@\spxentry{standard\_qf\_numeric()}\spxextra{StandardQFNumeric static method}}

\begin{fulllineitems}
\phantomsection\label{\detokenize{source/pysubgroup:pysubgroup.numeric_target.StandardQFNumeric.standard_qf_numeric}}\pysiglinewithargsret{\sphinxbfcode{\sphinxupquote{static }}\sphinxbfcode{\sphinxupquote{standard\_qf\_numeric}}}{\emph{a}, \emph{\_}, \emph{mean\_dataset}, \emph{instances\_subgroup}, \emph{mean\_sg}}{}
\end{fulllineitems}

\index{supports\_weights() (StandardQFNumeric method)@\spxentry{supports\_weights()}\spxextra{StandardQFNumeric method}}

\begin{fulllineitems}
\phantomsection\label{\detokenize{source/pysubgroup:pysubgroup.numeric_target.StandardQFNumeric.supports_weights}}\pysiglinewithargsret{\sphinxbfcode{\sphinxupquote{supports\_weights}}}{}{}
\end{fulllineitems}

\index{tpl (StandardQFNumeric attribute)@\spxentry{tpl}\spxextra{StandardQFNumeric attribute}}

\begin{fulllineitems}
\phantomsection\label{\detokenize{source/pysubgroup:pysubgroup.numeric_target.StandardQFNumeric.tpl}}\pysigline{\sphinxbfcode{\sphinxupquote{tpl}}}
alias of \sphinxcode{\sphinxupquote{StandardQFNumeric\_parameters}}

\end{fulllineitems}


\end{fulllineitems}

\index{get\_max\_generalization\_mean() (in module pysubgroup.numeric\_target)@\spxentry{get\_max\_generalization\_mean()}\spxextra{in module pysubgroup.numeric\_target}}

\begin{fulllineitems}
\phantomsection\label{\detokenize{source/pysubgroup:pysubgroup.numeric_target.get_max_generalization_mean}}\pysiglinewithargsret{\sphinxbfcode{\sphinxupquote{get\_max\_generalization\_mean}}}{\emph{data}, \emph{subgroup}, \emph{weighting\_attribute=None}}{}
\end{fulllineitems}



\subsection{pysubgroup.refinement\_operator module}
\label{\detokenize{source/pysubgroup:module-pysubgroup.refinement_operator}}\label{\detokenize{source/pysubgroup:pysubgroup-refinement-operator-module}}\index{pysubgroup.refinement\_operator (module)@\spxentry{pysubgroup.refinement\_operator}\spxextra{module}}\index{RefinementOperator (class in pysubgroup.refinement\_operator)@\spxentry{RefinementOperator}\spxextra{class in pysubgroup.refinement\_operator}}

\begin{fulllineitems}
\phantomsection\label{\detokenize{source/pysubgroup:pysubgroup.refinement_operator.RefinementOperator}}\pysigline{\sphinxbfcode{\sphinxupquote{class }}\sphinxbfcode{\sphinxupquote{RefinementOperator}}}
\end{fulllineitems}

\index{StaticGeneralizationOperator (class in pysubgroup.refinement\_operator)@\spxentry{StaticGeneralizationOperator}\spxextra{class in pysubgroup.refinement\_operator}}

\begin{fulllineitems}
\phantomsection\label{\detokenize{source/pysubgroup:pysubgroup.refinement_operator.StaticGeneralizationOperator}}\pysiglinewithargsret{\sphinxbfcode{\sphinxupquote{class }}\sphinxbfcode{\sphinxupquote{StaticGeneralizationOperator}}}{\emph{selectors}}{}~\index{refinements() (StaticGeneralizationOperator method)@\spxentry{refinements()}\spxextra{StaticGeneralizationOperator method}}

\begin{fulllineitems}
\phantomsection\label{\detokenize{source/pysubgroup:pysubgroup.refinement_operator.StaticGeneralizationOperator.refinements}}\pysiglinewithargsret{\sphinxbfcode{\sphinxupquote{refinements}}}{\emph{sG}}{}
\end{fulllineitems}


\end{fulllineitems}

\index{StaticSpecializationOperator (class in pysubgroup.refinement\_operator)@\spxentry{StaticSpecializationOperator}\spxextra{class in pysubgroup.refinement\_operator}}

\begin{fulllineitems}
\phantomsection\label{\detokenize{source/pysubgroup:pysubgroup.refinement_operator.StaticSpecializationOperator}}\pysiglinewithargsret{\sphinxbfcode{\sphinxupquote{class }}\sphinxbfcode{\sphinxupquote{StaticSpecializationOperator}}}{\emph{selectors}}{}~\index{refinements() (StaticSpecializationOperator method)@\spxentry{refinements()}\spxextra{StaticSpecializationOperator method}}

\begin{fulllineitems}
\phantomsection\label{\detokenize{source/pysubgroup:pysubgroup.refinement_operator.StaticSpecializationOperator.refinements}}\pysiglinewithargsret{\sphinxbfcode{\sphinxupquote{refinements}}}{\emph{subgroup}}{}
\end{fulllineitems}


\end{fulllineitems}



\subsection{pysubgroup.representations module}
\label{\detokenize{source/pysubgroup:module-pysubgroup.representations}}\label{\detokenize{source/pysubgroup:pysubgroup-representations-module}}\index{pysubgroup.representations (module)@\spxentry{pysubgroup.representations}\spxextra{module}}\index{BitSetRepresentation (class in pysubgroup.representations)@\spxentry{BitSetRepresentation}\spxextra{class in pysubgroup.representations}}

\begin{fulllineitems}
\phantomsection\label{\detokenize{source/pysubgroup:pysubgroup.representations.BitSetRepresentation}}\pysiglinewithargsret{\sphinxbfcode{\sphinxupquote{class }}\sphinxbfcode{\sphinxupquote{BitSetRepresentation}}}{\emph{df}, \emph{selectors\_to\_patch}}{}~\index{Conjunction (BitSetRepresentation attribute)@\spxentry{Conjunction}\spxextra{BitSetRepresentation attribute}}

\begin{fulllineitems}
\phantomsection\label{\detokenize{source/pysubgroup:pysubgroup.representations.BitSetRepresentation.Conjunction}}\pysigline{\sphinxbfcode{\sphinxupquote{Conjunction}}}
alias of {\hyperref[\detokenize{source/pysubgroup:pysubgroup.representations.BitSet_Conjunction}]{\sphinxcrossref{\sphinxcode{\sphinxupquote{BitSet\_Conjunction}}}}}

\end{fulllineitems}

\index{Disjunction (BitSetRepresentation attribute)@\spxentry{Disjunction}\spxextra{BitSetRepresentation attribute}}

\begin{fulllineitems}
\phantomsection\label{\detokenize{source/pysubgroup:pysubgroup.representations.BitSetRepresentation.Disjunction}}\pysigline{\sphinxbfcode{\sphinxupquote{Disjunction}}}
alias of {\hyperref[\detokenize{source/pysubgroup:pysubgroup.representations.BitSet_Disjunction}]{\sphinxcrossref{\sphinxcode{\sphinxupquote{BitSet\_Disjunction}}}}}

\end{fulllineitems}

\index{patch\_classes() (BitSetRepresentation method)@\spxentry{patch\_classes()}\spxextra{BitSetRepresentation method}}

\begin{fulllineitems}
\phantomsection\label{\detokenize{source/pysubgroup:pysubgroup.representations.BitSetRepresentation.patch_classes}}\pysiglinewithargsret{\sphinxbfcode{\sphinxupquote{patch\_classes}}}{}{}
\end{fulllineitems}

\index{patch\_selector() (BitSetRepresentation method)@\spxentry{patch\_selector()}\spxextra{BitSetRepresentation method}}

\begin{fulllineitems}
\phantomsection\label{\detokenize{source/pysubgroup:pysubgroup.representations.BitSetRepresentation.patch_selector}}\pysiglinewithargsret{\sphinxbfcode{\sphinxupquote{patch\_selector}}}{\emph{sel}}{}
\end{fulllineitems}


\end{fulllineitems}

\index{BitSet\_Conjunction (class in pysubgroup.representations)@\spxentry{BitSet\_Conjunction}\spxextra{class in pysubgroup.representations}}

\begin{fulllineitems}
\phantomsection\label{\detokenize{source/pysubgroup:pysubgroup.representations.BitSet_Conjunction}}\pysiglinewithargsret{\sphinxbfcode{\sphinxupquote{class }}\sphinxbfcode{\sphinxupquote{BitSet\_Conjunction}}}{\emph{*args}, \emph{**kwargs}}{}~\index{append\_and() (BitSet\_Conjunction method)@\spxentry{append\_and()}\spxextra{BitSet\_Conjunction method}}

\begin{fulllineitems}
\phantomsection\label{\detokenize{source/pysubgroup:pysubgroup.representations.BitSet_Conjunction.append_and}}\pysiglinewithargsret{\sphinxbfcode{\sphinxupquote{append\_and}}}{\emph{to\_append}}{}
\end{fulllineitems}

\index{compute\_representation() (BitSet\_Conjunction method)@\spxentry{compute\_representation()}\spxextra{BitSet\_Conjunction method}}

\begin{fulllineitems}
\phantomsection\label{\detokenize{source/pysubgroup:pysubgroup.representations.BitSet_Conjunction.compute_representation}}\pysiglinewithargsret{\sphinxbfcode{\sphinxupquote{compute\_representation}}}{}{}
\end{fulllineitems}

\index{n\_instances (BitSet\_Conjunction attribute)@\spxentry{n\_instances}\spxextra{BitSet\_Conjunction attribute}}

\begin{fulllineitems}
\phantomsection\label{\detokenize{source/pysubgroup:pysubgroup.representations.BitSet_Conjunction.n_instances}}\pysigline{\sphinxbfcode{\sphinxupquote{n\_instances}}\sphinxbfcode{\sphinxupquote{ = 0}}}
\end{fulllineitems}

\index{size() (BitSet\_Conjunction property)@\spxentry{size()}\spxextra{BitSet\_Conjunction property}}

\begin{fulllineitems}
\phantomsection\label{\detokenize{source/pysubgroup:pysubgroup.representations.BitSet_Conjunction.size}}\pysigline{\sphinxbfcode{\sphinxupquote{property }}\sphinxbfcode{\sphinxupquote{size}}}
\end{fulllineitems}


\end{fulllineitems}

\index{BitSet\_Disjunction (class in pysubgroup.representations)@\spxentry{BitSet\_Disjunction}\spxextra{class in pysubgroup.representations}}

\begin{fulllineitems}
\phantomsection\label{\detokenize{source/pysubgroup:pysubgroup.representations.BitSet_Disjunction}}\pysiglinewithargsret{\sphinxbfcode{\sphinxupquote{class }}\sphinxbfcode{\sphinxupquote{BitSet\_Disjunction}}}{\emph{*args}, \emph{**kwargs}}{}~\index{append\_or() (BitSet\_Disjunction method)@\spxentry{append\_or()}\spxextra{BitSet\_Disjunction method}}

\begin{fulllineitems}
\phantomsection\label{\detokenize{source/pysubgroup:pysubgroup.representations.BitSet_Disjunction.append_or}}\pysiglinewithargsret{\sphinxbfcode{\sphinxupquote{append\_or}}}{\emph{to\_append}}{}
\end{fulllineitems}

\index{compute\_representation() (BitSet\_Disjunction method)@\spxentry{compute\_representation()}\spxextra{BitSet\_Disjunction method}}

\begin{fulllineitems}
\phantomsection\label{\detokenize{source/pysubgroup:pysubgroup.representations.BitSet_Disjunction.compute_representation}}\pysiglinewithargsret{\sphinxbfcode{\sphinxupquote{compute\_representation}}}{}{}
\end{fulllineitems}

\index{size() (BitSet\_Disjunction property)@\spxentry{size()}\spxextra{BitSet\_Disjunction property}}

\begin{fulllineitems}
\phantomsection\label{\detokenize{source/pysubgroup:pysubgroup.representations.BitSet_Disjunction.size}}\pysigline{\sphinxbfcode{\sphinxupquote{property }}\sphinxbfcode{\sphinxupquote{size}}}
\end{fulllineitems}


\end{fulllineitems}

\index{NumpySetRepresentation (class in pysubgroup.representations)@\spxentry{NumpySetRepresentation}\spxextra{class in pysubgroup.representations}}

\begin{fulllineitems}
\phantomsection\label{\detokenize{source/pysubgroup:pysubgroup.representations.NumpySetRepresentation}}\pysiglinewithargsret{\sphinxbfcode{\sphinxupquote{class }}\sphinxbfcode{\sphinxupquote{NumpySetRepresentation}}}{\emph{df}, \emph{selectors\_to\_patch}}{}~\index{Conjunction (NumpySetRepresentation attribute)@\spxentry{Conjunction}\spxextra{NumpySetRepresentation attribute}}

\begin{fulllineitems}
\phantomsection\label{\detokenize{source/pysubgroup:pysubgroup.representations.NumpySetRepresentation.Conjunction}}\pysigline{\sphinxbfcode{\sphinxupquote{Conjunction}}}
alias of {\hyperref[\detokenize{source/pysubgroup:pysubgroup.representations.NumpySet_Conjunction}]{\sphinxcrossref{\sphinxcode{\sphinxupquote{NumpySet\_Conjunction}}}}}

\end{fulllineitems}

\index{patch\_classes() (NumpySetRepresentation method)@\spxentry{patch\_classes()}\spxextra{NumpySetRepresentation method}}

\begin{fulllineitems}
\phantomsection\label{\detokenize{source/pysubgroup:pysubgroup.representations.NumpySetRepresentation.patch_classes}}\pysiglinewithargsret{\sphinxbfcode{\sphinxupquote{patch\_classes}}}{}{}
\end{fulllineitems}

\index{patch\_selector() (NumpySetRepresentation method)@\spxentry{patch\_selector()}\spxextra{NumpySetRepresentation method}}

\begin{fulllineitems}
\phantomsection\label{\detokenize{source/pysubgroup:pysubgroup.representations.NumpySetRepresentation.patch_selector}}\pysiglinewithargsret{\sphinxbfcode{\sphinxupquote{patch\_selector}}}{\emph{sel}}{}
\end{fulllineitems}


\end{fulllineitems}

\index{NumpySet\_Conjunction (class in pysubgroup.representations)@\spxentry{NumpySet\_Conjunction}\spxextra{class in pysubgroup.representations}}

\begin{fulllineitems}
\phantomsection\label{\detokenize{source/pysubgroup:pysubgroup.representations.NumpySet_Conjunction}}\pysiglinewithargsret{\sphinxbfcode{\sphinxupquote{class }}\sphinxbfcode{\sphinxupquote{NumpySet\_Conjunction}}}{\emph{*args}, \emph{**kwargs}}{}~\index{all\_set (NumpySet\_Conjunction attribute)@\spxentry{all\_set}\spxextra{NumpySet\_Conjunction attribute}}

\begin{fulllineitems}
\phantomsection\label{\detokenize{source/pysubgroup:pysubgroup.representations.NumpySet_Conjunction.all_set}}\pysigline{\sphinxbfcode{\sphinxupquote{all\_set}}\sphinxbfcode{\sphinxupquote{ = None}}}
\end{fulllineitems}

\index{append\_and() (NumpySet\_Conjunction method)@\spxentry{append\_and()}\spxextra{NumpySet\_Conjunction method}}

\begin{fulllineitems}
\phantomsection\label{\detokenize{source/pysubgroup:pysubgroup.representations.NumpySet_Conjunction.append_and}}\pysiglinewithargsret{\sphinxbfcode{\sphinxupquote{append\_and}}}{\emph{to\_append}}{}
\end{fulllineitems}

\index{compute\_representation() (NumpySet\_Conjunction method)@\spxentry{compute\_representation()}\spxextra{NumpySet\_Conjunction method}}

\begin{fulllineitems}
\phantomsection\label{\detokenize{source/pysubgroup:pysubgroup.representations.NumpySet_Conjunction.compute_representation}}\pysiglinewithargsret{\sphinxbfcode{\sphinxupquote{compute\_representation}}}{}{}
\end{fulllineitems}

\index{size() (NumpySet\_Conjunction property)@\spxentry{size()}\spxextra{NumpySet\_Conjunction property}}

\begin{fulllineitems}
\phantomsection\label{\detokenize{source/pysubgroup:pysubgroup.representations.NumpySet_Conjunction.size}}\pysigline{\sphinxbfcode{\sphinxupquote{property }}\sphinxbfcode{\sphinxupquote{size}}}
\end{fulllineitems}


\end{fulllineitems}

\index{RepresentationBase (class in pysubgroup.representations)@\spxentry{RepresentationBase}\spxextra{class in pysubgroup.representations}}

\begin{fulllineitems}
\phantomsection\label{\detokenize{source/pysubgroup:pysubgroup.representations.RepresentationBase}}\pysiglinewithargsret{\sphinxbfcode{\sphinxupquote{class }}\sphinxbfcode{\sphinxupquote{RepresentationBase}}}{\emph{new\_conjunction}, \emph{selectors\_to\_patch}}{}~\index{patch\_all\_selectors() (RepresentationBase method)@\spxentry{patch\_all\_selectors()}\spxextra{RepresentationBase method}}

\begin{fulllineitems}
\phantomsection\label{\detokenize{source/pysubgroup:pysubgroup.representations.RepresentationBase.patch_all_selectors}}\pysiglinewithargsret{\sphinxbfcode{\sphinxupquote{patch\_all\_selectors}}}{}{}
\end{fulllineitems}

\index{patch\_classes() (RepresentationBase method)@\spxentry{patch\_classes()}\spxextra{RepresentationBase method}}

\begin{fulllineitems}
\phantomsection\label{\detokenize{source/pysubgroup:pysubgroup.representations.RepresentationBase.patch_classes}}\pysiglinewithargsret{\sphinxbfcode{\sphinxupquote{patch\_classes}}}{}{}
\end{fulllineitems}

\index{patch\_selector() (RepresentationBase method)@\spxentry{patch\_selector()}\spxextra{RepresentationBase method}}

\begin{fulllineitems}
\phantomsection\label{\detokenize{source/pysubgroup:pysubgroup.representations.RepresentationBase.patch_selector}}\pysiglinewithargsret{\sphinxbfcode{\sphinxupquote{patch\_selector}}}{\emph{sel}}{}
\end{fulllineitems}

\index{undo\_patch\_classes() (RepresentationBase method)@\spxentry{undo\_patch\_classes()}\spxextra{RepresentationBase method}}

\begin{fulllineitems}
\phantomsection\label{\detokenize{source/pysubgroup:pysubgroup.representations.RepresentationBase.undo_patch_classes}}\pysiglinewithargsret{\sphinxbfcode{\sphinxupquote{undo\_patch\_classes}}}{}{}
\end{fulllineitems}


\end{fulllineitems}

\index{SetRepresentation (class in pysubgroup.representations)@\spxentry{SetRepresentation}\spxextra{class in pysubgroup.representations}}

\begin{fulllineitems}
\phantomsection\label{\detokenize{source/pysubgroup:pysubgroup.representations.SetRepresentation}}\pysiglinewithargsret{\sphinxbfcode{\sphinxupquote{class }}\sphinxbfcode{\sphinxupquote{SetRepresentation}}}{\emph{df}, \emph{selectors\_to\_patch}}{}~\index{Conjunction (SetRepresentation attribute)@\spxentry{Conjunction}\spxextra{SetRepresentation attribute}}

\begin{fulllineitems}
\phantomsection\label{\detokenize{source/pysubgroup:pysubgroup.representations.SetRepresentation.Conjunction}}\pysigline{\sphinxbfcode{\sphinxupquote{Conjunction}}}
alias of {\hyperref[\detokenize{source/pysubgroup:pysubgroup.representations.Set_Conjunction}]{\sphinxcrossref{\sphinxcode{\sphinxupquote{Set\_Conjunction}}}}}

\end{fulllineitems}

\index{patch\_classes() (SetRepresentation method)@\spxentry{patch\_classes()}\spxextra{SetRepresentation method}}

\begin{fulllineitems}
\phantomsection\label{\detokenize{source/pysubgroup:pysubgroup.representations.SetRepresentation.patch_classes}}\pysiglinewithargsret{\sphinxbfcode{\sphinxupquote{patch\_classes}}}{}{}
\end{fulllineitems}

\index{patch\_selector() (SetRepresentation method)@\spxentry{patch\_selector()}\spxextra{SetRepresentation method}}

\begin{fulllineitems}
\phantomsection\label{\detokenize{source/pysubgroup:pysubgroup.representations.SetRepresentation.patch_selector}}\pysiglinewithargsret{\sphinxbfcode{\sphinxupquote{patch\_selector}}}{\emph{sel}}{}
\end{fulllineitems}


\end{fulllineitems}

\index{Set\_Conjunction (class in pysubgroup.representations)@\spxentry{Set\_Conjunction}\spxextra{class in pysubgroup.representations}}

\begin{fulllineitems}
\phantomsection\label{\detokenize{source/pysubgroup:pysubgroup.representations.Set_Conjunction}}\pysiglinewithargsret{\sphinxbfcode{\sphinxupquote{class }}\sphinxbfcode{\sphinxupquote{Set\_Conjunction}}}{\emph{*args}, \emph{**kwargs}}{}~\index{all\_set (Set\_Conjunction attribute)@\spxentry{all\_set}\spxextra{Set\_Conjunction attribute}}

\begin{fulllineitems}
\phantomsection\label{\detokenize{source/pysubgroup:pysubgroup.representations.Set_Conjunction.all_set}}\pysigline{\sphinxbfcode{\sphinxupquote{all\_set}}\sphinxbfcode{\sphinxupquote{ = \{\}}}}
\end{fulllineitems}

\index{append\_and() (Set\_Conjunction method)@\spxentry{append\_and()}\spxextra{Set\_Conjunction method}}

\begin{fulllineitems}
\phantomsection\label{\detokenize{source/pysubgroup:pysubgroup.representations.Set_Conjunction.append_and}}\pysiglinewithargsret{\sphinxbfcode{\sphinxupquote{append\_and}}}{\emph{to\_append}}{}
\end{fulllineitems}

\index{compute\_representation() (Set\_Conjunction method)@\spxentry{compute\_representation()}\spxextra{Set\_Conjunction method}}

\begin{fulllineitems}
\phantomsection\label{\detokenize{source/pysubgroup:pysubgroup.representations.Set_Conjunction.compute_representation}}\pysiglinewithargsret{\sphinxbfcode{\sphinxupquote{compute\_representation}}}{}{}
\end{fulllineitems}

\index{size() (Set\_Conjunction property)@\spxentry{size()}\spxextra{Set\_Conjunction property}}

\begin{fulllineitems}
\phantomsection\label{\detokenize{source/pysubgroup:pysubgroup.representations.Set_Conjunction.size}}\pysigline{\sphinxbfcode{\sphinxupquote{property }}\sphinxbfcode{\sphinxupquote{size}}}
\end{fulllineitems}


\end{fulllineitems}



\subsection{pysubgroup.subgroup module}
\label{\detokenize{source/pysubgroup:module-pysubgroup.subgroup}}\label{\detokenize{source/pysubgroup:pysubgroup-subgroup-module}}\index{pysubgroup.subgroup (module)@\spxentry{pysubgroup.subgroup}\spxextra{module}}
Created on 28.04.2016

@author: lemmerfn
\index{EqualitySelector (class in pysubgroup.subgroup)@\spxentry{EqualitySelector}\spxextra{class in pysubgroup.subgroup}}

\begin{fulllineitems}
\phantomsection\label{\detokenize{source/pysubgroup:pysubgroup.subgroup.EqualitySelector}}\pysiglinewithargsret{\sphinxbfcode{\sphinxupquote{class }}\sphinxbfcode{\sphinxupquote{EqualitySelector}}}{\emph{attribute\_name}, \emph{attribute\_value}, \emph{selector\_name=None}}{}~\index{attribute\_name() (EqualitySelector property)@\spxentry{attribute\_name()}\spxextra{EqualitySelector property}}

\begin{fulllineitems}
\phantomsection\label{\detokenize{source/pysubgroup:pysubgroup.subgroup.EqualitySelector.attribute_name}}\pysigline{\sphinxbfcode{\sphinxupquote{property }}\sphinxbfcode{\sphinxupquote{attribute\_name}}}
\end{fulllineitems}

\index{attribute\_value() (EqualitySelector property)@\spxentry{attribute\_value()}\spxextra{EqualitySelector property}}

\begin{fulllineitems}
\phantomsection\label{\detokenize{source/pysubgroup:pysubgroup.subgroup.EqualitySelector.attribute_value}}\pysigline{\sphinxbfcode{\sphinxupquote{property }}\sphinxbfcode{\sphinxupquote{attribute\_value}}}
\end{fulllineitems}

\index{compute\_descriptions() (EqualitySelector class method)@\spxentry{compute\_descriptions()}\spxextra{EqualitySelector class method}}

\begin{fulllineitems}
\phantomsection\label{\detokenize{source/pysubgroup:pysubgroup.subgroup.EqualitySelector.compute_descriptions}}\pysiglinewithargsret{\sphinxbfcode{\sphinxupquote{classmethod }}\sphinxbfcode{\sphinxupquote{compute\_descriptions}}}{\emph{attribute\_name}, \emph{attribute\_value}, \emph{selector\_name}}{}
\end{fulllineitems}

\index{covers() (EqualitySelector method)@\spxentry{covers()}\spxextra{EqualitySelector method}}

\begin{fulllineitems}
\phantomsection\label{\detokenize{source/pysubgroup:pysubgroup.subgroup.EqualitySelector.covers}}\pysiglinewithargsret{\sphinxbfcode{\sphinxupquote{covers}}}{\emph{data}}{}
\end{fulllineitems}

\index{set\_descriptions() (EqualitySelector method)@\spxentry{set\_descriptions()}\spxextra{EqualitySelector method}}

\begin{fulllineitems}
\phantomsection\label{\detokenize{source/pysubgroup:pysubgroup.subgroup.EqualitySelector.set_descriptions}}\pysiglinewithargsret{\sphinxbfcode{\sphinxupquote{set\_descriptions}}}{\emph{attribute\_name}, \emph{attribute\_value}, \emph{selector\_name=None}}{}
\end{fulllineitems}


\end{fulllineitems}

\index{IntervalSelector (class in pysubgroup.subgroup)@\spxentry{IntervalSelector}\spxextra{class in pysubgroup.subgroup}}

\begin{fulllineitems}
\phantomsection\label{\detokenize{source/pysubgroup:pysubgroup.subgroup.IntervalSelector}}\pysiglinewithargsret{\sphinxbfcode{\sphinxupquote{class }}\sphinxbfcode{\sphinxupquote{IntervalSelector}}}{\emph{attribute\_name}, \emph{lower\_bound}, \emph{upper\_bound}, \emph{selector\_name=None}}{}~\index{attribute\_name() (IntervalSelector property)@\spxentry{attribute\_name()}\spxextra{IntervalSelector property}}

\begin{fulllineitems}
\phantomsection\label{\detokenize{source/pysubgroup:pysubgroup.subgroup.IntervalSelector.attribute_name}}\pysigline{\sphinxbfcode{\sphinxupquote{property }}\sphinxbfcode{\sphinxupquote{attribute\_name}}}
\end{fulllineitems}

\index{compute\_descriptions() (IntervalSelector class method)@\spxentry{compute\_descriptions()}\spxextra{IntervalSelector class method}}

\begin{fulllineitems}
\phantomsection\label{\detokenize{source/pysubgroup:pysubgroup.subgroup.IntervalSelector.compute_descriptions}}\pysiglinewithargsret{\sphinxbfcode{\sphinxupquote{classmethod }}\sphinxbfcode{\sphinxupquote{compute\_descriptions}}}{\emph{attribute\_name}, \emph{lower\_bound}, \emph{upper\_bound}, \emph{selector\_name=None}}{}
\end{fulllineitems}

\index{compute\_string() (IntervalSelector class method)@\spxentry{compute\_string()}\spxextra{IntervalSelector class method}}

\begin{fulllineitems}
\phantomsection\label{\detokenize{source/pysubgroup:pysubgroup.subgroup.IntervalSelector.compute_string}}\pysiglinewithargsret{\sphinxbfcode{\sphinxupquote{classmethod }}\sphinxbfcode{\sphinxupquote{compute\_string}}}{\emph{attribute\_name}, \emph{lower\_bound}, \emph{upper\_bound}, \emph{rounding\_digits}}{}
\end{fulllineitems}

\index{covers() (IntervalSelector method)@\spxentry{covers()}\spxextra{IntervalSelector method}}

\begin{fulllineitems}
\phantomsection\label{\detokenize{source/pysubgroup:pysubgroup.subgroup.IntervalSelector.covers}}\pysiglinewithargsret{\sphinxbfcode{\sphinxupquote{covers}}}{\emph{data\_instance}}{}
\end{fulllineitems}

\index{lower\_bound() (IntervalSelector property)@\spxentry{lower\_bound()}\spxextra{IntervalSelector property}}

\begin{fulllineitems}
\phantomsection\label{\detokenize{source/pysubgroup:pysubgroup.subgroup.IntervalSelector.lower_bound}}\pysigline{\sphinxbfcode{\sphinxupquote{property }}\sphinxbfcode{\sphinxupquote{lower\_bound}}}
\end{fulllineitems}

\index{set\_descriptions() (IntervalSelector method)@\spxentry{set\_descriptions()}\spxextra{IntervalSelector method}}

\begin{fulllineitems}
\phantomsection\label{\detokenize{source/pysubgroup:pysubgroup.subgroup.IntervalSelector.set_descriptions}}\pysiglinewithargsret{\sphinxbfcode{\sphinxupquote{set\_descriptions}}}{\emph{attribute\_name}, \emph{lower\_bound}, \emph{upper\_bound}, \emph{selector\_name=None}}{}
\end{fulllineitems}

\index{upper\_bound() (IntervalSelector property)@\spxentry{upper\_bound()}\spxextra{IntervalSelector property}}

\begin{fulllineitems}
\phantomsection\label{\detokenize{source/pysubgroup:pysubgroup.subgroup.IntervalSelector.upper_bound}}\pysigline{\sphinxbfcode{\sphinxupquote{property }}\sphinxbfcode{\sphinxupquote{upper\_bound}}}
\end{fulllineitems}


\end{fulllineitems}

\index{NegatedSelector (class in pysubgroup.subgroup)@\spxentry{NegatedSelector}\spxextra{class in pysubgroup.subgroup}}

\begin{fulllineitems}
\phantomsection\label{\detokenize{source/pysubgroup:pysubgroup.subgroup.NegatedSelector}}\pysiglinewithargsret{\sphinxbfcode{\sphinxupquote{class }}\sphinxbfcode{\sphinxupquote{NegatedSelector}}}{\emph{selector}}{}~\index{attribute\_name() (NegatedSelector property)@\spxentry{attribute\_name()}\spxextra{NegatedSelector property}}

\begin{fulllineitems}
\phantomsection\label{\detokenize{source/pysubgroup:pysubgroup.subgroup.NegatedSelector.attribute_name}}\pysigline{\sphinxbfcode{\sphinxupquote{property }}\sphinxbfcode{\sphinxupquote{attribute\_name}}}
\end{fulllineitems}

\index{covers() (NegatedSelector method)@\spxentry{covers()}\spxextra{NegatedSelector method}}

\begin{fulllineitems}
\phantomsection\label{\detokenize{source/pysubgroup:pysubgroup.subgroup.NegatedSelector.covers}}\pysiglinewithargsret{\sphinxbfcode{\sphinxupquote{covers}}}{\emph{data\_instance}}{}
\end{fulllineitems}

\index{set\_descriptions() (NegatedSelector method)@\spxentry{set\_descriptions()}\spxextra{NegatedSelector method}}

\begin{fulllineitems}
\phantomsection\label{\detokenize{source/pysubgroup:pysubgroup.subgroup.NegatedSelector.set_descriptions}}\pysiglinewithargsret{\sphinxbfcode{\sphinxupquote{set\_descriptions}}}{\emph{selector}}{}
\end{fulllineitems}


\end{fulllineitems}

\index{SelectorBase (class in pysubgroup.subgroup)@\spxentry{SelectorBase}\spxextra{class in pysubgroup.subgroup}}

\begin{fulllineitems}
\phantomsection\label{\detokenize{source/pysubgroup:pysubgroup.subgroup.SelectorBase}}\pysigline{\sphinxbfcode{\sphinxupquote{class }}\sphinxbfcode{\sphinxupquote{SelectorBase}}}~\index{set\_descriptions() (SelectorBase method)@\spxentry{set\_descriptions()}\spxextra{SelectorBase method}}

\begin{fulllineitems}
\phantomsection\label{\detokenize{source/pysubgroup:pysubgroup.subgroup.SelectorBase.set_descriptions}}\pysiglinewithargsret{\sphinxbfcode{\sphinxupquote{abstract }}\sphinxbfcode{\sphinxupquote{set\_descriptions}}}{\emph{*args}, \emph{**kwargs}}{}
\end{fulllineitems}


\end{fulllineitems}

\index{Subgroup (class in pysubgroup.subgroup)@\spxentry{Subgroup}\spxextra{class in pysubgroup.subgroup}}

\begin{fulllineitems}
\phantomsection\label{\detokenize{source/pysubgroup:pysubgroup.subgroup.Subgroup}}\pysiglinewithargsret{\sphinxbfcode{\sphinxupquote{class }}\sphinxbfcode{\sphinxupquote{Subgroup}}}{\emph{target}, \emph{subgroup\_description}}{}~\index{calculate\_statistics() (Subgroup method)@\spxentry{calculate\_statistics()}\spxextra{Subgroup method}}

\begin{fulllineitems}
\phantomsection\label{\detokenize{source/pysubgroup:pysubgroup.subgroup.Subgroup.calculate_statistics}}\pysiglinewithargsret{\sphinxbfcode{\sphinxupquote{calculate\_statistics}}}{\emph{data}, \emph{weighting\_attribute=None}}{}
\end{fulllineitems}

\index{covers() (Subgroup method)@\spxentry{covers()}\spxextra{Subgroup method}}

\begin{fulllineitems}
\phantomsection\label{\detokenize{source/pysubgroup:pysubgroup.subgroup.Subgroup.covers}}\pysiglinewithargsret{\sphinxbfcode{\sphinxupquote{covers}}}{\emph{instance}}{}
\end{fulllineitems}

\index{get\_base\_statistics() (Subgroup method)@\spxentry{get\_base\_statistics()}\spxextra{Subgroup method}}

\begin{fulllineitems}
\phantomsection\label{\detokenize{source/pysubgroup:pysubgroup.subgroup.Subgroup.get_base_statistics}}\pysiglinewithargsret{\sphinxbfcode{\sphinxupquote{get\_base\_statistics}}}{\emph{data}, \emph{weighting\_attribute=None}}{}
\end{fulllineitems}


\end{fulllineitems}

\index{create\_nominal\_selectors() (in module pysubgroup.subgroup)@\spxentry{create\_nominal\_selectors()}\spxextra{in module pysubgroup.subgroup}}

\begin{fulllineitems}
\phantomsection\label{\detokenize{source/pysubgroup:pysubgroup.subgroup.create_nominal_selectors}}\pysiglinewithargsret{\sphinxbfcode{\sphinxupquote{create\_nominal\_selectors}}}{\emph{data}, \emph{ignore=None}}{}
\end{fulllineitems}

\index{create\_nominal\_selectors\_for\_attribute() (in module pysubgroup.subgroup)@\spxentry{create\_nominal\_selectors\_for\_attribute()}\spxextra{in module pysubgroup.subgroup}}

\begin{fulllineitems}
\phantomsection\label{\detokenize{source/pysubgroup:pysubgroup.subgroup.create_nominal_selectors_for_attribute}}\pysiglinewithargsret{\sphinxbfcode{\sphinxupquote{create\_nominal\_selectors\_for\_attribute}}}{\emph{data}, \emph{attribute\_name}, \emph{dtypes=None}}{}
\end{fulllineitems}

\index{create\_numeric\_selector\_for\_attribute() (in module pysubgroup.subgroup)@\spxentry{create\_numeric\_selector\_for\_attribute()}\spxextra{in module pysubgroup.subgroup}}

\begin{fulllineitems}
\phantomsection\label{\detokenize{source/pysubgroup:pysubgroup.subgroup.create_numeric_selector_for_attribute}}\pysiglinewithargsret{\sphinxbfcode{\sphinxupquote{create\_numeric\_selector\_for\_attribute}}}{\emph{data}, \emph{attr\_name}, \emph{nbins=5}, \emph{intervals\_only=True}, \emph{weighting\_attribute=None}}{}
\end{fulllineitems}

\index{create\_numeric\_selectors() (in module pysubgroup.subgroup)@\spxentry{create\_numeric\_selectors()}\spxextra{in module pysubgroup.subgroup}}

\begin{fulllineitems}
\phantomsection\label{\detokenize{source/pysubgroup:pysubgroup.subgroup.create_numeric_selectors}}\pysiglinewithargsret{\sphinxbfcode{\sphinxupquote{create\_numeric\_selectors}}}{\emph{data}, \emph{nbins=5}, \emph{intervals\_only=True}, \emph{weighting\_attribute=None}, \emph{ignore=None}}{}
\end{fulllineitems}

\index{create\_selectors() (in module pysubgroup.subgroup)@\spxentry{create\_selectors()}\spxextra{in module pysubgroup.subgroup}}

\begin{fulllineitems}
\phantomsection\label{\detokenize{source/pysubgroup:pysubgroup.subgroup.create_selectors}}\pysiglinewithargsret{\sphinxbfcode{\sphinxupquote{create\_selectors}}}{\emph{data}, \emph{nbins=5}, \emph{intervals\_only=True}, \emph{ignore=None}}{}
\end{fulllineitems}

\index{remove\_target\_attributes() (in module pysubgroup.subgroup)@\spxentry{remove\_target\_attributes()}\spxextra{in module pysubgroup.subgroup}}

\begin{fulllineitems}
\phantomsection\label{\detokenize{source/pysubgroup:pysubgroup.subgroup.remove_target_attributes}}\pysiglinewithargsret{\sphinxbfcode{\sphinxupquote{remove\_target\_attributes}}}{\emph{selectors}, \emph{target}}{}
\end{fulllineitems}



\subsection{pysubgroup.utils module}
\label{\detokenize{source/pysubgroup:module-pysubgroup.utils}}\label{\detokenize{source/pysubgroup:pysubgroup-utils-module}}\index{pysubgroup.utils (module)@\spxentry{pysubgroup.utils}\spxextra{module}}
Created on 02.05.2016

@author: lemmerfn
\index{SubgroupDiscoveryResult (class in pysubgroup.utils)@\spxentry{SubgroupDiscoveryResult}\spxextra{class in pysubgroup.utils}}

\begin{fulllineitems}
\phantomsection\label{\detokenize{source/pysubgroup:pysubgroup.utils.SubgroupDiscoveryResult}}\pysiglinewithargsret{\sphinxbfcode{\sphinxupquote{class }}\sphinxbfcode{\sphinxupquote{SubgroupDiscoveryResult}}}{\emph{results}, \emph{task}}{}~\index{supportSetVisualization() (SubgroupDiscoveryResult method)@\spxentry{supportSetVisualization()}\spxextra{SubgroupDiscoveryResult method}}

\begin{fulllineitems}
\phantomsection\label{\detokenize{source/pysubgroup:pysubgroup.utils.SubgroupDiscoveryResult.supportSetVisualization}}\pysiglinewithargsret{\sphinxbfcode{\sphinxupquote{supportSetVisualization}}}{\emph{in\_order=True}, \emph{drop\_empty=True}}{}
\end{fulllineitems}

\index{to\_dataframe() (SubgroupDiscoveryResult method)@\spxentry{to\_dataframe()}\spxextra{SubgroupDiscoveryResult method}}

\begin{fulllineitems}
\phantomsection\label{\detokenize{source/pysubgroup:pysubgroup.utils.SubgroupDiscoveryResult.to_dataframe}}\pysiglinewithargsret{\sphinxbfcode{\sphinxupquote{to\_dataframe}}}{\emph{include\_info=False}}{}
\end{fulllineitems}

\index{to\_descriptions() (SubgroupDiscoveryResult method)@\spxentry{to\_descriptions()}\spxextra{SubgroupDiscoveryResult method}}

\begin{fulllineitems}
\phantomsection\label{\detokenize{source/pysubgroup:pysubgroup.utils.SubgroupDiscoveryResult.to_descriptions}}\pysiglinewithargsret{\sphinxbfcode{\sphinxupquote{to\_descriptions}}}{}{}
\end{fulllineitems}

\index{to\_subgroups() (SubgroupDiscoveryResult method)@\spxentry{to\_subgroups()}\spxextra{SubgroupDiscoveryResult method}}

\begin{fulllineitems}
\phantomsection\label{\detokenize{source/pysubgroup:pysubgroup.utils.SubgroupDiscoveryResult.to_subgroups}}\pysiglinewithargsret{\sphinxbfcode{\sphinxupquote{to\_subgroups}}}{}{}
\end{fulllineitems}


\end{fulllineitems}

\index{add\_if\_required() (in module pysubgroup.utils)@\spxentry{add\_if\_required()}\spxextra{in module pysubgroup.utils}}

\begin{fulllineitems}
\phantomsection\label{\detokenize{source/pysubgroup:pysubgroup.utils.add_if_required}}\pysiglinewithargsret{\sphinxbfcode{\sphinxupquote{add\_if\_required}}}{\emph{result}, \emph{sg}, \emph{quality}, \emph{task}, \emph{check\_for\_duplicates=False}}{}
\end{fulllineitems}

\index{as\_df() (in module pysubgroup.utils)@\spxentry{as\_df()}\spxextra{in module pysubgroup.utils}}

\begin{fulllineitems}
\phantomsection\label{\detokenize{source/pysubgroup:pysubgroup.utils.as_df}}\pysiglinewithargsret{\sphinxbfcode{\sphinxupquote{as\_df}}}{\emph{data}, \emph{result}, \emph{statistics\_to\_show=(\textquotesingle{}size\_sg\textquotesingle{}}, \emph{\textquotesingle{}size\_dataset\textquotesingle{}}, \emph{\textquotesingle{}positives\_sg\textquotesingle{}}, \emph{\textquotesingle{}positives\_dataset\textquotesingle{}}, \emph{\textquotesingle{}size\_complement\textquotesingle{}}, \emph{\textquotesingle{}relative\_size\_sg\textquotesingle{}}, \emph{\textquotesingle{}relative\_size\_complement\textquotesingle{}}, \emph{\textquotesingle{}coverage\_sg\textquotesingle{}}, \emph{\textquotesingle{}coverage\_complement\textquotesingle{}}, \emph{\textquotesingle{}target\_share\_sg\textquotesingle{}}, \emph{\textquotesingle{}target\_share\_complement\textquotesingle{}}, \emph{\textquotesingle{}target\_share\_dataset\textquotesingle{}}, \emph{\textquotesingle{}lift\textquotesingle{})}, \emph{autoround=False}, \emph{weighting\_attribute=None}, \emph{include\_target=False}}{}
\end{fulllineitems}

\index{conditional\_invert() (in module pysubgroup.utils)@\spxentry{conditional\_invert()}\spxextra{in module pysubgroup.utils}}

\begin{fulllineitems}
\phantomsection\label{\detokenize{source/pysubgroup:pysubgroup.utils.conditional_invert}}\pysiglinewithargsret{\sphinxbfcode{\sphinxupquote{conditional\_invert}}}{\emph{val}, \emph{invert}}{}
\end{fulllineitems}

\index{count\_bits() (in module pysubgroup.utils)@\spxentry{count\_bits()}\spxextra{in module pysubgroup.utils}}

\begin{fulllineitems}
\phantomsection\label{\detokenize{source/pysubgroup:pysubgroup.utils.count_bits}}\pysiglinewithargsret{\sphinxbfcode{\sphinxupquote{count\_bits}}}{\emph{bitset\_as\_int}}{}
\end{fulllineitems}

\index{effective\_sample\_size() (in module pysubgroup.utils)@\spxentry{effective\_sample\_size()}\spxextra{in module pysubgroup.utils}}

\begin{fulllineitems}
\phantomsection\label{\detokenize{source/pysubgroup:pysubgroup.utils.effective_sample_size}}\pysiglinewithargsret{\sphinxbfcode{\sphinxupquote{effective\_sample\_size}}}{\emph{weights}}{}
\end{fulllineitems}

\index{equal\_frequency\_discretization() (in module pysubgroup.utils)@\spxentry{equal\_frequency\_discretization()}\spxextra{in module pysubgroup.utils}}

\begin{fulllineitems}
\phantomsection\label{\detokenize{source/pysubgroup:pysubgroup.utils.equal_frequency_discretization}}\pysiglinewithargsret{\sphinxbfcode{\sphinxupquote{equal\_frequency\_discretization}}}{\emph{data}, \emph{attribute\_name}, \emph{nbins=5}, \emph{weighting\_attribute=None}}{}
\end{fulllineitems}

\index{find\_set\_bits() (in module pysubgroup.utils)@\spxentry{find\_set\_bits()}\spxextra{in module pysubgroup.utils}}

\begin{fulllineitems}
\phantomsection\label{\detokenize{source/pysubgroup:pysubgroup.utils.find_set_bits}}\pysiglinewithargsret{\sphinxbfcode{\sphinxupquote{find\_set\_bits}}}{\emph{bitset\_as\_int}}{}
\end{fulllineitems}

\index{float\_formatter() (in module pysubgroup.utils)@\spxentry{float\_formatter()}\spxextra{in module pysubgroup.utils}}

\begin{fulllineitems}
\phantomsection\label{\detokenize{source/pysubgroup:pysubgroup.utils.float_formatter}}\pysiglinewithargsret{\sphinxbfcode{\sphinxupquote{float\_formatter}}}{\emph{x}, \emph{digits=2}}{}
\end{fulllineitems}

\index{intersect\_of\_ordered\_list() (in module pysubgroup.utils)@\spxentry{intersect\_of\_ordered\_list()}\spxextra{in module pysubgroup.utils}}

\begin{fulllineitems}
\phantomsection\label{\detokenize{source/pysubgroup:pysubgroup.utils.intersect_of_ordered_list}}\pysiglinewithargsret{\sphinxbfcode{\sphinxupquote{intersect\_of\_ordered\_list}}}{\emph{list\_1}, \emph{list\_2}}{}
\end{fulllineitems}

\index{is\_categorical\_attribute() (in module pysubgroup.utils)@\spxentry{is\_categorical\_attribute()}\spxextra{in module pysubgroup.utils}}

\begin{fulllineitems}
\phantomsection\label{\detokenize{source/pysubgroup:pysubgroup.utils.is_categorical_attribute}}\pysiglinewithargsret{\sphinxbfcode{\sphinxupquote{is\_categorical\_attribute}}}{\emph{data}, \emph{attribute\_name}}{}
\end{fulllineitems}

\index{is\_numerical\_attribute() (in module pysubgroup.utils)@\spxentry{is\_numerical\_attribute()}\spxextra{in module pysubgroup.utils}}

\begin{fulllineitems}
\phantomsection\label{\detokenize{source/pysubgroup:pysubgroup.utils.is_numerical_attribute}}\pysiglinewithargsret{\sphinxbfcode{\sphinxupquote{is\_numerical\_attribute}}}{\emph{data}, \emph{attribute\_name}}{}
\end{fulllineitems}

\index{minimum\_required\_quality() (in module pysubgroup.utils)@\spxentry{minimum\_required\_quality()}\spxextra{in module pysubgroup.utils}}

\begin{fulllineitems}
\phantomsection\label{\detokenize{source/pysubgroup:pysubgroup.utils.minimum_required_quality}}\pysiglinewithargsret{\sphinxbfcode{\sphinxupquote{minimum\_required\_quality}}}{\emph{result}, \emph{task}}{}
\end{fulllineitems}

\index{overlap() (in module pysubgroup.utils)@\spxentry{overlap()}\spxextra{in module pysubgroup.utils}}

\begin{fulllineitems}
\phantomsection\label{\detokenize{source/pysubgroup:pysubgroup.utils.overlap}}\pysiglinewithargsret{\sphinxbfcode{\sphinxupquote{overlap}}}{\emph{sg}, \emph{another\_sg}, \emph{data}}{}
\end{fulllineitems}

\index{perc\_formatter() (in module pysubgroup.utils)@\spxentry{perc\_formatter()}\spxextra{in module pysubgroup.utils}}

\begin{fulllineitems}
\phantomsection\label{\detokenize{source/pysubgroup:pysubgroup.utils.perc_formatter}}\pysiglinewithargsret{\sphinxbfcode{\sphinxupquote{perc\_formatter}}}{\emph{x}}{}
\end{fulllineitems}

\index{powerset() (in module pysubgroup.utils)@\spxentry{powerset()}\spxextra{in module pysubgroup.utils}}

\begin{fulllineitems}
\phantomsection\label{\detokenize{source/pysubgroup:pysubgroup.utils.powerset}}\pysiglinewithargsret{\sphinxbfcode{\sphinxupquote{powerset}}}{\emph{{[}1,2,3{]}) \sphinxhyphen{}\sphinxhyphen{}\textgreater{} () (1,) (2,) (3,) (1,2) (1,3) (2,3) (1,2,3}}{}
\end{fulllineitems}

\index{print\_result\_set() (in module pysubgroup.utils)@\spxentry{print\_result\_set()}\spxextra{in module pysubgroup.utils}}

\begin{fulllineitems}
\phantomsection\label{\detokenize{source/pysubgroup:pysubgroup.utils.print_result_set}}\pysiglinewithargsret{\sphinxbfcode{\sphinxupquote{print\_result\_set}}}{\emph{data}, \emph{result}, \emph{statistics\_to\_show}, \emph{weighting\_attribute=None}, \emph{print\_header=True}, \emph{include\_target=False}}{}
\end{fulllineitems}

\index{remove\_selectors\_with\_attributes() (in module pysubgroup.utils)@\spxentry{remove\_selectors\_with\_attributes()}\spxextra{in module pysubgroup.utils}}

\begin{fulllineitems}
\phantomsection\label{\detokenize{source/pysubgroup:pysubgroup.utils.remove_selectors_with_attributes}}\pysiglinewithargsret{\sphinxbfcode{\sphinxupquote{remove\_selectors\_with\_attributes}}}{\emph{selector\_list}, \emph{attribute\_list}}{}
\end{fulllineitems}

\index{result\_as\_table() (in module pysubgroup.utils)@\spxentry{result\_as\_table()}\spxextra{in module pysubgroup.utils}}

\begin{fulllineitems}
\phantomsection\label{\detokenize{source/pysubgroup:pysubgroup.utils.result_as_table}}\pysiglinewithargsret{\sphinxbfcode{\sphinxupquote{result\_as\_table}}}{\emph{data}, \emph{result}, \emph{statistics\_to\_show}, \emph{weighting\_attribute=None}, \emph{print\_header=True}, \emph{include\_target=False}}{}
\end{fulllineitems}

\index{results\_as\_df() (in module pysubgroup.utils)@\spxentry{results\_as\_df()}\spxextra{in module pysubgroup.utils}}

\begin{fulllineitems}
\phantomsection\label{\detokenize{source/pysubgroup:pysubgroup.utils.results_as_df}}\pysiglinewithargsret{\sphinxbfcode{\sphinxupquote{results\_as\_df}}}{\emph{data}, \emph{result}, \emph{statistics\_to\_show=(\textquotesingle{}size\_sg\textquotesingle{}}, \emph{\textquotesingle{}size\_dataset\textquotesingle{}}, \emph{\textquotesingle{}positives\_sg\textquotesingle{}}, \emph{\textquotesingle{}positives\_dataset\textquotesingle{}}, \emph{\textquotesingle{}size\_complement\textquotesingle{}}, \emph{\textquotesingle{}relative\_size\_sg\textquotesingle{}}, \emph{\textquotesingle{}relative\_size\_complement\textquotesingle{}}, \emph{\textquotesingle{}coverage\_sg\textquotesingle{}}, \emph{\textquotesingle{}coverage\_complement\textquotesingle{}}, \emph{\textquotesingle{}target\_share\_sg\textquotesingle{}}, \emph{\textquotesingle{}target\_share\_complement\textquotesingle{}}, \emph{\textquotesingle{}target\_share\_dataset\textquotesingle{}}, \emph{\textquotesingle{}lift\textquotesingle{})}, \emph{autoround=False}, \emph{weighting\_attribute=None}, \emph{include\_target=False}}{}
\end{fulllineitems}

\index{results\_df\_autoround() (in module pysubgroup.utils)@\spxentry{results\_df\_autoround()}\spxextra{in module pysubgroup.utils}}

\begin{fulllineitems}
\phantomsection\label{\detokenize{source/pysubgroup:pysubgroup.utils.results_df_autoround}}\pysiglinewithargsret{\sphinxbfcode{\sphinxupquote{results\_df\_autoround}}}{\emph{df}}{}
\end{fulllineitems}

\index{to\_bits() (in module pysubgroup.utils)@\spxentry{to\_bits()}\spxextra{in module pysubgroup.utils}}

\begin{fulllineitems}
\phantomsection\label{\detokenize{source/pysubgroup:pysubgroup.utils.to_bits}}\pysiglinewithargsret{\sphinxbfcode{\sphinxupquote{to\_bits}}}{\emph{list\_of\_ints}}{}
\end{fulllineitems}

\index{to\_latex() (in module pysubgroup.utils)@\spxentry{to\_latex()}\spxextra{in module pysubgroup.utils}}

\begin{fulllineitems}
\phantomsection\label{\detokenize{source/pysubgroup:pysubgroup.utils.to_latex}}\pysiglinewithargsret{\sphinxbfcode{\sphinxupquote{to\_latex}}}{\emph{data}, \emph{result}, \emph{statistics\_to\_show}}{}
\end{fulllineitems}



\subsection{pysubgroup.visualization module}
\label{\detokenize{source/pysubgroup:module-pysubgroup.visualization}}\label{\detokenize{source/pysubgroup:pysubgroup-visualization-module}}\index{pysubgroup.visualization (module)@\spxentry{pysubgroup.visualization}\spxextra{module}}\index{compare\_distributions\_numeric() (in module pysubgroup.visualization)@\spxentry{compare\_distributions\_numeric()}\spxextra{in module pysubgroup.visualization}}

\begin{fulllineitems}
\phantomsection\label{\detokenize{source/pysubgroup:pysubgroup.visualization.compare_distributions_numeric}}\pysiglinewithargsret{\sphinxbfcode{\sphinxupquote{compare\_distributions\_numeric}}}{\emph{sgs}, \emph{data}, \emph{bins}}{}
\end{fulllineitems}

\index{plot\_distribution\_numeric() (in module pysubgroup.visualization)@\spxentry{plot\_distribution\_numeric()}\spxextra{in module pysubgroup.visualization}}

\begin{fulllineitems}
\phantomsection\label{\detokenize{source/pysubgroup:pysubgroup.visualization.plot_distribution_numeric}}\pysiglinewithargsret{\sphinxbfcode{\sphinxupquote{plot\_distribution\_numeric}}}{\emph{sg}, \emph{data}, \emph{bins}}{}
\end{fulllineitems}

\index{plot\_npspace() (in module pysubgroup.visualization)@\spxentry{plot\_npspace()}\spxextra{in module pysubgroup.visualization}}

\begin{fulllineitems}
\phantomsection\label{\detokenize{source/pysubgroup:pysubgroup.visualization.plot_npspace}}\pysiglinewithargsret{\sphinxbfcode{\sphinxupquote{plot\_npspace}}}{\emph{result\_df}, \emph{data}, \emph{annotate=True}, \emph{fixed\_limits=False}}{}
\end{fulllineitems}

\index{plot\_roc() (in module pysubgroup.visualization)@\spxentry{plot\_roc()}\spxextra{in module pysubgroup.visualization}}

\begin{fulllineitems}
\phantomsection\label{\detokenize{source/pysubgroup:pysubgroup.visualization.plot_roc}}\pysiglinewithargsret{\sphinxbfcode{\sphinxupquote{plot\_roc}}}{\emph{result\_df}, \emph{data}, \emph{qf=\textless{}pysubgroup.binary\_target.StandardQF object\textgreater{}}, \emph{levels=40}, \emph{annotate=False}}{}
\end{fulllineitems}

\index{plot\_sgbars() (in module pysubgroup.visualization)@\spxentry{plot\_sgbars()}\spxextra{in module pysubgroup.visualization}}

\begin{fulllineitems}
\phantomsection\label{\detokenize{source/pysubgroup:pysubgroup.visualization.plot_sgbars}}\pysiglinewithargsret{\sphinxbfcode{\sphinxupquote{plot\_sgbars}}}{\emph{result\_df}, \emph{\_}, \emph{ylabel=\textquotesingle{}target share\textquotesingle{}}, \emph{title=\textquotesingle{}Discovered Subgroups\textquotesingle{}}, \emph{dynamic\_widths=False}, \emph{\_suffix=\textquotesingle{}\textquotesingle{}}}{}
\end{fulllineitems}

\index{similarity\_dendrogram() (in module pysubgroup.visualization)@\spxentry{similarity\_dendrogram()}\spxextra{in module pysubgroup.visualization}}

\begin{fulllineitems}
\phantomsection\label{\detokenize{source/pysubgroup:pysubgroup.visualization.similarity_dendrogram}}\pysiglinewithargsret{\sphinxbfcode{\sphinxupquote{similarity\_dendrogram}}}{\emph{result}, \emph{data}}{}
\end{fulllineitems}

\index{similarity\_sgs() (in module pysubgroup.visualization)@\spxentry{similarity\_sgs()}\spxextra{in module pysubgroup.visualization}}

\begin{fulllineitems}
\phantomsection\label{\detokenize{source/pysubgroup:pysubgroup.visualization.similarity_sgs}}\pysiglinewithargsret{\sphinxbfcode{\sphinxupquote{similarity\_sgs}}}{\emph{sgd\_results}, \emph{data}, \emph{color=True}}{}
\end{fulllineitems}



\subsection{Module contents}
\label{\detokenize{source/pysubgroup:module-pysubgroup}}\label{\detokenize{source/pysubgroup:module-contents}}\index{pysubgroup (module)@\spxentry{pysubgroup}\spxextra{module}}

\chapter{Indices and tables}
\label{\detokenize{index:indices-and-tables}}\begin{itemize}
\item {} 
\DUrole{xref,std,std-ref}{genindex}

\item {} 
\DUrole{xref,std,std-ref}{modindex}

\item {} 
\DUrole{xref,std,std-ref}{search}

\end{itemize}


\renewcommand{\indexname}{Python Module Index}
\begin{sphinxtheindex}
\let\bigletter\sphinxstyleindexlettergroup
\bigletter{p}
\item\relax\sphinxstyleindexentry{pysubgroup}\sphinxstyleindexpageref{source/pysubgroup:\detokenize{module-pysubgroup}}
\item\relax\sphinxstyleindexentry{pysubgroup.algorithms}\sphinxstyleindexpageref{source/pysubgroup:\detokenize{module-pysubgroup.algorithms}}
\item\relax\sphinxstyleindexentry{pysubgroup.boolean\_expressions}\sphinxstyleindexpageref{source/pysubgroup:\detokenize{module-pysubgroup.boolean_expressions}}
\item\relax\sphinxstyleindexentry{pysubgroup.fi\_target}\sphinxstyleindexpageref{source/pysubgroup:\detokenize{module-pysubgroup.fi_target}}
\item\relax\sphinxstyleindexentry{pysubgroup.gp\_growth}\sphinxstyleindexpageref{source/pysubgroup:\detokenize{module-pysubgroup.gp_growth}}
\item\relax\sphinxstyleindexentry{pysubgroup.measures}\sphinxstyleindexpageref{source/pysubgroup:\detokenize{module-pysubgroup.measures}}
\item\relax\sphinxstyleindexentry{pysubgroup.model\_target}\sphinxstyleindexpageref{source/pysubgroup:\detokenize{module-pysubgroup.model_target}}
\item\relax\sphinxstyleindexentry{pysubgroup.numeric\_target}\sphinxstyleindexpageref{source/pysubgroup:\detokenize{module-pysubgroup.numeric_target}}
\item\relax\sphinxstyleindexentry{pysubgroup.refinement\_operator}\sphinxstyleindexpageref{source/pysubgroup:\detokenize{module-pysubgroup.refinement_operator}}
\item\relax\sphinxstyleindexentry{pysubgroup.representations}\sphinxstyleindexpageref{source/pysubgroup:\detokenize{module-pysubgroup.representations}}
\item\relax\sphinxstyleindexentry{pysubgroup.subgroup}\sphinxstyleindexpageref{source/pysubgroup:\detokenize{module-pysubgroup.subgroup}}
\item\relax\sphinxstyleindexentry{pysubgroup.utils}\sphinxstyleindexpageref{source/pysubgroup:\detokenize{module-pysubgroup.utils}}
\item\relax\sphinxstyleindexentry{pysubgroup.visualization}\sphinxstyleindexpageref{source/pysubgroup:\detokenize{module-pysubgroup.visualization}}
\end{sphinxtheindex}

\renewcommand{\indexname}{Index}
\printindex
\end{document}